\begin{tabella}{!{\VRule}c!{\VRule}p{8cm}!{\VRule}c!{\VRule}c!{\VRule}}
\color{white} \bold{Codice test} & \color{white} \bold{Descrizione} & \color{white} \bold{Requisito} & \color{white} \bold{Stato}\\
\endfirsthead
TV1 & Un nuovo utente vuole verificare la possibilità di creare un proprio account.
\newline \newline 
Operazioni:
{\begin{enumerate}
\item entrare nella pagina di registrazione;
\item inserire il nome;
\item inserire il cognome;
\item inserire l'email;
\item inserire l'email di conferma;
\item inserire la password;
\item inserire la password di conferma;
\item verificare la presenza di un avviso qualora vi siano errori;
\item in caso di assenza di errori verificare che la creazione dell'account sia avvenuta correttamente eseguendo un login.
\end{enumerate}
} & ROF1 & Non superato\\
TV1.3 & L'utente desidera verificare che l'inserimento della propria email non provochi alcun problema.
\newline \newline
Operazioni:
{\begin{enumerate}
\item inserire l'email nell'apposito campo;
\item verificare che l'indirizzo email sia univoco all'interno del sistema;
\item verificare che l'indirizzo email abbia un formato valido come descritto nell'\AdRdoc;
\item verificare la segnalazione di errori attraverso degli avvisi qualora uno dei punti precedenti non sia stato eseguito in modo corretto.
\end{enumerate}
} & ROF1.3 & Non superato\\
TV1.3.3 & L'utente deve verificare che l'email di conferma inserita sia uguale a quella precedentemente inserita nel campo sopra.
\newline \newline
Operazioni:
{\begin{enumerate}
\item inserire l'email per la seconda volta;
\item verificare che l'email appena inserita sia uguale a quella inserita nel campo soprastante;
\item verificare la presenza di un avviso qualora si verifichi un errore con la email appena inserita.
\end{enumerate}
} & ROF1.3.3 & Non superato\\
TV1.4 & L'utente desidera verificare che l'inserimento della password non provochi problemi.
 \newline \newline
 Operazioni:
 {\begin{enumerate}
 \item inserire la password nell'apposito campo;
 \item verificare che la password rispetti le caratteristiche descritte nell'\AdRdoc;
 \item verificare la presenza di un avviso d'errore qualora ci sia stato un problema.
\end{enumerate}
 } & ROF1.4 & Non superato\\
TV1.4.2 & L'utente deve verificare che la password di conferma inserita sia uguale a quella precedentemente digitata nel campo sopra.
\newline \newline
Operazioni:
{\begin{enumerate}
\item inserire la password per la seconda volta;
\item verificare che la password appena inserita sia uguale a quella inserita nel campo soprastante;
\item verificare la presenza di un avviso qualora si verifichi un errore con la password appena inserita.
\end{enumerate}
} & ROF1.4.2 & Non superato\\
TV2 & L'utente deve verificare la possibilità di autenticarsi una volta registrato.
\newline \newline
Operazioni:
{\begin{enumerate}
\item inserire l'email;
\item inserire la password;
\item cliccare sul pulsante di accesso;
\item verificare la presenza di un avviso in caso ci sia stato un errore durante l'autenticazione;
\item verificare la corretta autenticazione nel caso in cui i dati inseriti siano validi e non abbiano causato alcun errore.
\end{enumerate}
} & ROF2 & Non superato\\
TV3 & L'utente vuole verificare la possibilità di recuperare la propria password utilizzando l' email usata al momento della registrazione.
\newline \newline
Operazioni:
{\begin{enumerate}
\item entrare nella pagina dedicata al recupero della password;
\item inserire la propria email nell'appropriato campo;
\item cliccare sull'apposito pulsante per eseguire l'operazione;
\item verificare che, in caso di errore nel recupero della password, l'utente sia avvisato;
\item verificare, qualora non sia avvenuto alcun errore, di aver ricevuto una email con una password temporanea al suo interno.
\end{enumerate}
} & ROF3 & Non superato\\
TV8 & L'utente, che sia studente, docente oppure responsabile, deve verificare che il sistema funzioni correttamente su una piattaforma desktop.
\newline \newline
Operazioni:
{\begin{enumerate}
\item l'utente deve disporre di una connessione internet;
\item entrare nel sito di Quizzipedia da un computer desktop;
\item verificarne il suo corretto funzionamento.
\end{enumerate}
} & ROV8 & Non superato\\
TV9 & Lo studente vuole verificare la possibilità di utilizzare le funzionalità del sistema da dispositivi mobile.
\newline \newline
Operazioni:
{\begin{enumerate}
\item l'utente deve disporre di una connessione ad internet sul proprio dispositivo mobile; 
\item entrare nel sito di Quizzipedia da un \gl{browser} del dispositivo;
\item verificarne il corretto funzionamento.
\end{enumerate}
} & ROV9 & Non superato\\
TV10 & L'utente vuole verificare il corretto funzionamento del sistema con l'uso di \gl{JavaScript} e \gl{cookies}.
\newline \newline
Operazioni:
{\begin{enumerate}
\item l'utente deve disporre di una connessione ad internet;
\item l'utente deve aprire il \gl{browser} con cui ha intenzione di accedere a Quizzipedia;
\item verificare che nelle impostazioni del \gl{browser} Javascript e \gl{cookies} siano abilitati;
\item navigare nel sito di Quizzipedia; 
\end{enumerate}
} & ROV10 & Non superato\\
TV11 & L'utente vuole verificare la possibilità di svolgere i quiz pubblici all'interno del sistema.
\newline \newline
Operazioni:
{\begin{enumerate}
\item entrare nella pagina d'archivio dei quiz pubblici;
\item selezionare il quiz desiderato tramite le operazioni di ricerca descritte nell'\AdRdoc;
\item confermare la selezione del quiz desiderato cliccandoci sopra;
\item svolgere il quiz rispondendo alle domande secondo le modalità descritte nell'\AdRdoc;
\item confermare la risoluzione del quiz cliccando il pulsante apposito;
\item visualizzare l'esito del quiz svolto.
\end{enumerate}
} & ROF11 & N.I.\\
TV12.1 & L'utente vuole verificare la possibilità di eseguire una ricerca tra i quiz e visualizzare i risultati ottenuti.
\newline \newline
Operazioni:
{\begin{enumerate}
\item entrare nella pagina dedicata alla ricerca;
\item selezionare i parametri di ricerca (argomento, livello di difficoltà, parole chiave, autore ) per i quiz solamente pubblici;
\item premere il pulsante 'Cerca';
\item verificare la corretta visualizzazione dei risultati.
\end{enumerate}
} & ROF12.1 & N.I.\\
TV12.1.1.5 & Lo studente vuole verificare la possibilità di eseguire una ricerca tra i quiz solamente privati.
\newline \newline
Operazioni:
{\begin{enumerate}
\item entrare nella pagina dedicata alla ricerca;
\item selezionare i parametri di ricerca (argomento, livello di difficoltà, parole chiave, autore) per i quiz;
\item selezionare la voce 'Privati' per restringere il campo di ricerca ai quiz privati legati alla classe a cui lo studente appartiene;
\item premere il pulsante 'Cerca';
\item verificare la corretta visualizzazione dei risultati.
\end{enumerate}
} & ROF12.1.1.5 & N.I.\\
TV12.2 & Il docente vuole verificare la possibilità di eseguire una ricerca tra le domande presenti nel sistema.
\newline \newline
Operazioni:
{\begin{enumerate}
\item il docente deve essere autenticato;
\item entrare nella pagina dedicata alla ricerca;
\item selezionare i parametri di ricerca (argomento, livello di difficoltà, parole chiave, autore) per le domande;
\item premere il pulsante 'Cerca';
\item verificare la corretta visualizzazione dei risultati.
\end{enumerate}
} & ROF12.2 & N.I.\\
TV13 & L'utente può verificare l'uscita dal sistema eseguendo il logout.
\newline \newline
Operazioni:
{\begin{enumerate}
\item l'utente deve essere autenticato;
\item premere il pulsante di logout;
\item verificare di essere realmente fuori dall'area di autenticazione.
\end{enumerate}
} & ROF13 & Non superato\\
TV14.1 & L'utente vuole verificare di poter visualizzare le proprie informazioni personali.
\newline \newline
Operazioni:
{\begin{enumerate}
\item l'utente si deve autenticare;
\item entrare nella pagina dedicata al profilo;
\item visualizzare i propri dati presenti nella pagina come descritto nell'\AdRdoc.
\end{enumerate}
} & ROF14.1 & Non superato\\
TV14.2 & L'utente vuole verificare la possibilità di visualizzare lo storico dei propri quiz.
\newline \newline
Operazioni:
{\begin{enumerate}
\item l'utente deve essere autenticato;
\item entrare nella pagina dedicata allo storico dei quiz;
\item verificare la presenza corretta dei propri quiz svolti in precedenza.
\end{enumerate}
} & ROF14.2 & N.I.\\
TV14.3 & L'utente vuole verificare la disponibilità di modificare i propri dati personali.
\newline \newline
Operazioni:
{\begin{enumerate}
\item l'utente si deve autenticare;
\item entrare nella pagina dedicata al profilo;
\item selezionale il tasto dedicato alla modifica delle proprie informazioni personali;
\item confermare le proprie modifiche;
\item verificare la corretta modifica del campo/i dati cambiato/i.
\end{enumerate}
} & ROF14.3 & Non superato\\
TV14.3.3 & L'utente vuole verificare la possibilità di modificare la propria password.
\newline \newline
Operazioni:
{\begin{enumerate}
\item L'utente si deve autenticare;
\item entrare nella pagina dedicata al profilo;
\item selezionare il tasto 'Cambia password';
\item nello spazio dedicato alla modifica della password, inserire la password corrente o una password temporanea fornita dal sistema;
\item inserire la nuova password;
\item inserire nuovamente la password nuova;
\item confermare la modifica della password;
\item verificare che il cambio password sia avvenuto correttamente eseguendo un login;
\item verificare la presenza di un avviso in caso sia avvenuto un errore durante il cambio della password.
\end{enumerate}
} & ROF14.3.3 & Non superato\\
TV14.4.1 & Il responsabile vuole verificare la possibilità di eliminare dal proprio ente un account da lui gestito.
\newline \newline
Operazioni:
{\begin{enumerate}
\item il responsabile deve essere autenticato;
\item entrare nella pagina dedicata alla gestione account;
\item selezionare l'account da eliminare dall'ente, tra quelli gestiti da lui stesso;
\item premere il pulsante 'Elimina';
\item verificare che l'account sia stato eliminato dall'ente correttamente.
\end{enumerate}
} & ROF14.4.1 & Non superato\\
TV15.1 & Il docente vuole verificare la possibilità di creare un quiz.
\newline \newline
Operazioni:
{\begin{enumerate}
\item il docente deve essere autenticato;
\item entrare nella pagina dedicata ai quiz;
\item premere il pulsante 'Crea quiz';
\item inserire nel campo apposito il titolo del quiz;
\item selezionare l'argomento del quiz;
\item inserire la descrizione del quiz;
\item specificare, se desiderato, delle parole chiave da associare al quiz;
\item selezionare il permesso del quiz (pubblico o privato);
\item nel caso sia stato selezionato il permesso 'privato', selezionare le classi a cui il quiz è destinato;
\item aggiungere le domande desiderate;
\item confermare il salvataggio del quiz;
\item verificare che il docente sia attribuito automaticamente come autore del quiz in questione;
\item verificare la corretta creazione del quiz.
\end{enumerate}
} & ROF15.1 & N.I.\\
TV15.2 & Il docente vuole modificare un quiz precedentemente
creato.
\newline \newline
Operazioni: {\begin{enumerate}
\item il docente deve essere autenticato;
\item entrare nella pagina dedicata ai quiz;
\item selezionare il quiz che si desidera modificare;
\item premere il pulsante 'Modifica';
\item modificare il campo desiderato (titolo, argomento, descrizione, permesso, parole chiave);
\item aggiungere le domande desiderate;
\item rimuovere le domande indesiderate;
\item confermare la modifica;
\item verificare che la modifica sia stata attuata in modo corretto;
\item verificare che venga presentato un avviso in caso si verifichi un errore durante la modifica del
quiz. \end{enumerate}} & RZF15.2 & N.I.\\
TV15.2.5 & Il docente vuole verificare la possibilità di inserire
una domanda all'interno di un quiz. \newline \newline
Operazioni:
{\begin{enumerate}
\item il docente deve essere autenticato;
\item entrare nella pagina dedicata ai quiz;
\item selezionare il quiz che si desidera modificare;
\item premere il pulsante 'Modifica';
\item selezionare la voce 'Aggiungi domanda';
\item selezionare una domanda già esistente oppure
crearne una nuova;
\item confermare l'aggiunta della domanda;
\item confermare la modifica del quiz;
\item verificare che la modifica sia stata attuata in
modo corretto;
\item verificare che venga presentato un avviso in
caso di errore durante la modifica di un quiz. \end{enumerate}} & RZF15.2.5 & N.I.\\
TV15.2.6 & Il docente vuole verificare la possibilità di eliminare
una domanda all'interno di un quiz. \newline \newline
Operazioni:
{\begin{enumerate}
\item il docente deve essere autenticato;
\item entrare nella pagina dedicata ai quiz;
\item selezionare il quiz che si desidera modificare;
\item premere il pulsante modifica;
\item selezionare le domande che si vogliono eliminare dal quiz;
\item selezionare la voce 'Elimina domanda';
\item confermare la modifica del quiz;
\item verificare che la modifica sia stata attuata in modo corretto;
\item verificare che venga presentato un avviso in
caso di errore durante la modifica del quiz. \end{enumerate}} & RZF15.2.6 & N.I.\\
TV15.3 & Il docente vuole testare la possibilità di eliminare un quiz esistente.
\newline \newline
Operazioni:
{\begin{enumerate}
\item il docente deve essere autenticato;
\item entrare nella pagina dedicata ai quiz;
\item selezionare il quiz che si desidera eliminare;
\item premere il pulsante 'Elimina';
\item confermare l'eliminazione del quiz;
\item verificare che l'eliminazione sia stata effettuata in modo corretto.
\end{enumerate}
} & ROF15.3 & N.I.\\
TV16.1 & L'utente senza ruolo vuole verificare la possibilità di effettuare una richiesta per ricoprire il ruolo di studente all'interno di un ente.
\newline \newline
Operazioni:
{\begin{enumerate}
\item l'utente senza ruolo deve essere autenticato;
\item inviare la richiesta per il ruolo di studente all'interno dell'ente desiderato;
\item verificare che la richiesta sia stata accettata oppure negata tramite un avviso;
\item verificare che avvenga un messaggio di errore se l'utente possiede già un ruolo all'interno di tale ente e cerca comunque di eseguire una richiesta per un nuovo ruolo per tale ente.
\end{enumerate}
} & ROF16.1 & Non superato\\
TV16.2 & L'utente senza ruolo vuole verificare la possibilità di effettuare una richiesta per ricoprire il ruolo di docente all'interno di un ente.
\newline \newline
Operazioni:
{\begin{enumerate}
\item l'utente senza ruolo deve essere autenticato;
\item inviare la richiesta per il ruolo di docente all'interno dell'ente desiderato;
\item verificare che la richiesta sia stata accettata oppure negata tramite un avviso;
\item verificare che avvenga un messaggio di errore se l'utente possiede già un ruolo all'interno di tale ente e cerca comunque di eseguire una richiesta per un nuovo ruolo per tale ente.
\end{enumerate}
} & ROF16.2 & Non superato\\
TV16.3 & Il docente vuole verificare di poter effettuare una richiesta di inserimento in una classe esistente.
\newline \newline
Operazioni:
{\begin{enumerate}
\item il docente deve essere autenticato;
\item selezionare l'ente contenente la classe alla quale ci si vuole iscrivere;
\item selezionare la classe a cui ci si vuole iscrivere;
\item premere il pulsante 'Invia richiesta' per richiedere il proprio inserimento in tale classe;
\item verificare l'accettazione o la negazione della richiesta inviata;
\item verificare che venga segnalato un errore qualora un docente appartenente ad una classe richieda di essere inserito nella medesima classe.
\end{enumerate}
} & ROF16.3 & Non superato\\
TV16.4 & Lo studente vuole verificare di poter effettuare una richiesta di inserimento in una classe esistente.
\newline \newline
Operazioni:
{\begin{enumerate}
\item lo studente deve essere autenticato;
\item selezionare l'ente contenente la classe alla quale ci si vuole iscrivere;
\item selezionare la classe in cui ci si vuole inserire;
\item premere il pulsante 'Invia richiesta' per richiedere il proprio inserimento in tale classe;
\item verificare l'accettazione o la negazione della richiesta inviata;
\item verificare che venga segnalato un errore qualora uno studente appartenente ad una classe richieda di essere inserito nella medesima classe.
\end{enumerate}
} & ROF16.4 & Non superato\\
TV17.1 & Il responsabile vuole verificare la possibilità di gestire le richieste eseguite dai docenti per l'inserimento in una classe.
\newline \newline
Operazioni:
{\begin{enumerate}
\item il responsabile deve essere autenticato;
\item entrare nella pagina contenente la lista delle richieste in sospeso;
\item selezionare la richiesta ricevuta dal docente;
\item accettare o negare la richiesta;
\item se accettata, verificare che il docente sia stato inserito correttamente nella classe da lui richiesta.
\end{enumerate}
} & ROF17.1 & Non superato\\
TV17.2 & Il responsabile vuole verificare la possibilità di gestire le richieste di attribuzione del ruolo di docente ad un utente senza ruolo.
\newline \newline
Operazioni:
{\begin{enumerate}
\item il responsabile deve essere autenticato;
\item entrare nella pagina contenente la lista delle richieste in sospeso;
\item selezionare la richiesta di attribuzione del ruolo di docente;
\item accettare o rifiutare la richiesta;
\item se accettata, verificare che l'utente senza ruolo ora sia in possesso del ruolo di docente all'interno dell'ente gestito da tale responsabile.
\end{enumerate}
} & ROF17.2 & Non superato\\
TV17.3 & Il docente vuole verificare la possibilità di gestire le richieste effettuate da uno studente per il suo inserimento.
\newline \newline
Operazioni:
{\begin{enumerate}
\item il docente deve essere autenticato;
\item entrare nella pagina contenente la lista delle richieste in sospeso;
\item selezionare la richiesta espressa dallo studente per l'inserimento in una classe;
\item accettare o negare la richiesta;
\item se accettata, verificare il corretto inserimento dello studente nella classe desiderata.
\end{enumerate}
} & ROF17.3 & Non superato\\
TV17.4 & Il responsabile vuole verificare la possibilità di gestire le richieste di attribuzione del ruolo di studente ad un utente senza ruolo.
\newline \newline
Operazioni:
{\begin{enumerate}
\item il responsabile deve essere autenticato;
\item entrare nella pagina contenente la lista delle richieste in sospeso;
\item selezionare la richiesta di attribuzione del ruolo di studente;
\item accettare o rifiutare la richiesta;
\item se accettata, verificare che l'utente senza ruolo ora sia in possesso del ruolo di studente.
\end{enumerate}
} & ROF17.4 & Non superato\\
TV18.1 & Il responsabile vuole verificare la possibilità di creare un nuovo argomento all'interno del sistema.
\newline \newline
Operazioni:
{\begin{enumerate}
\item il responsabile deve essere autenticato;
\item entrare nella pagina dedicata agli argomenti;
\item premere il pulsante 'Nuovo argomento';
\item inserire il nome dell'argomento;
\item verificare che l'argomento sia stato creato correttamente;
\item verificare che venga segnalato un errore qualora il responsabile cerchi di creare un argomento con un nome già assegnato ad un altro argomento.
\end{enumerate}
} & ROF18.1 & Non superato\\
TV18.2 & Il responsabile vuole verificare la disponibilità di eliminare dal sistema un argomento esistente.
\newline \newline
Operazioni:
{\begin{enumerate}
\item il responsabile deve essere autenticato;
\item entrare nella pagina dedicata agli argomenti;
\item selezionare l'argomento che si desidera eliminare;
\item premere il pulsante elimina;
\item confermare l'eliminazione dell'argomento;
\item verificare che l'argomento sia stato eliminato correttamente.
\end{enumerate}
} & ROF18.2 & Non superato\\
TV19 & Il responsabile verifica la possibilità di modificare il proprio ente.
\newline \newline
Operazioni:
{\begin{enumerate}
\item il responsabile deve essere autenticato;
\item entrare nella pagina dedicata al proprio ente;
\item selezionare 'Modifica ente';
\item modificare il nome del proprio ente;
\item confermare la modifica;
\item verificare la corretta modifica.
\end{enumerate}
} & ROF19 & Non superato\\
TV20.1 & Il docente vuole verificare la possibilità di visualizzare le statistiche relative ai quiz.
\newline \newline
Operazioni:
{\begin{enumerate}
\item il docente deve essere autenticato;
\item entrare nella pagina dedicata ai quiz;
\item selezionare la voce statistiche al suo interno;
\item verificare la corretta visualizzazione delle statistiche come descritto nell'\AdRdoc.
\end{enumerate}
} & ROF20.1 & N.I.\\
TV20.2 & Il docente vuole verificare la possibilità di visualizzare le statistiche relative alle domande.
\newline \newline
Operazioni:
{\begin{enumerate}
\item il docente deve essere autenticato;
\item entrare nella pagina dedicata alle domande;
\item selezionare la voce statistiche al suo interno;
\item verificare la corretta visualizzazione delle statistiche come descritto nell'\AdRdoc.
\end{enumerate}
} & ROF20.2 & N.I.\\
TV20.3 & Il docente vuole visualizzare le statistiche relative agli studenti.
\newline \newline
Operazioni:
{\begin{enumerate}
\item il docente deve essere autenticato; 
\item entrare nella pagina dedicata agli studenti;
\item eseguire una ricerca per trovare lo studente desiderato; 
\item selezionare lo studente di cui si vuole vedere le statistiche;
\item verificare la corretta visualizzazione delle statistiche come definito nell'\AdRdoc.
\end{enumerate}
} & ROF20.3 & N.I.\\
TV20.4 & Il docente vuole verificare di poter visualizzare le statistiche relative ai docenti.
\newline \newline
Operazioni:
{\begin{enumerate}
\item il docente deve essere autenticato;
\item entrare nella pagina dedicata ai docenti;
\item selezionare la voce 'Statistiche';
\item visualizzare correttamente la lista dei quiz creati da un docente e la lista delle domande create da ogni docente (docenti che appartengono allo stesso ente di colui che cerca le statistiche).
\end{enumerate}
} & RDF20.4 & N.I.\\
TV21.1 & Il responsabile vuole verificare la possibilità di creare una classe associata al proprio ente.
\newline \newline
Operazioni:
{\begin{enumerate}
\item il responsabile deve essere autenticato;
\item entrare nella pagina dedicata alle classi del proprio ente;
\item selezionare il pulsante 'Nuova classe' dal suo interno;
\item inserire il nome della classe;
\item inserire la descrizione della classe;
\item inserire l'anno scolastico;
\item confermare la creazione della classe;
\item verificare la corretta creazione della classe all'interno del proprio ente.

\end{enumerate}
} & ROF21.1 & Non superato\\
TV21.2 & Il responsabile vuole verificare la possibilità di modificare una classe esistente all'interno del proprio ente.
\newline \newline
Operazioni:
{\begin{enumerate}
\item il responsabile deve essere autenticato;
\item entrare nella pagina dedicata alle classi del proprio ente;
\item selezionare la classe che si desidera modificare;
\item premere il pulsante 'Modifica';
\item eseguire la modifica della descrizione;
\item confermare la modifica;
\item verificare la corretta modifica della classe.
\end{enumerate}
} & ROF21.2 & Non superato\\
TV21.3 & Il responsabile verifica la possibilità di eliminare una classe appartenente al proprio ente.
\newline \newline
Operazioni:
{\begin{enumerate}
\item il responsabile deve essere autenticato;
\item entrare nella pagina dedicata alle classi del proprio ente;
\item selezionare la classe che si desidera eliminare;
\item premere il pulsante 'Elimina';
\item confermare l'eliminazione;
\item verificare la corretta rimozione della classe dal sistema.
\end{enumerate}
} & ROF21.3 & Non superato\\
TV21.4 & Il docente deve verificare di poter visualizzare la lista di studenti associati ad una classe.
\newline \newline
Operazioni:
{\begin{enumerate}
\item il responsabile deve essere autenticato;
\item entrare nella pagina dedicata alle classi del proprio ente;
\item selezionare la classe di cui si desidera conoscere il contenuto;
\item visualizzare la lista degli studenti di cui ne fanno parte.
\end{enumerate}
} & ROF21.4 & Non superato\\
TV21.5 & Il responsabile deve verificare di poter visualizzare la lista di docenti associati ad una classe.
\newline \newline
Operazioni:
{\begin{enumerate}
\item il responsabile deve essere autenticato;
\item entrare nella pagina dedicata alle classi del proprio ente;
\item selezionare la classe di cui si desidera conoscere il contenuto;
\item visualizzare la lista dei docenti di cui ne fanno parte.
\end{enumerate}
} & ROF21.5 & Non superato\\
TV21.6 & Il responsabile vuole verificare la possibilità di gestire una classe esistente all'interno del proprio ente.
\newline \newline
Operazioni:
{\begin{enumerate}
\item il responsabile deve essere autenticato;
\item entrare nella pagina dedicata alle classi del proprio ente;
\item selezionare la classe che si desidera gestire;
\item inserire/rimuovere uno studente;
\item inserire/rimuovere un docente;
\item verificare che l'operazione non abbia prodotto alcun errore.
\end{enumerate}
} & ROF21.6 & Non superato\\
TV23 & Lo studente vuole verificare la possibilità di commentare i quiz e le domande.
\newline \newline
Operazioni:
{\begin{enumerate}
\item lo studente deve essere autenticato;
\item entrare nella pagina dedicata al forum;
\item selezionare la voce 'Domande' oppure 'Quiz';
\item lasciare un proprio commento relativo ad una domanda o a un quiz;
\item verificare che il commento sia visibile nel forum.
\end{enumerate}
} & RZF23 & N.I.\\
TV31.2 & Il docente vuole verificare la possibilità di interrompere la creazione di una domanda uscendo dal sistema.
\newline \newline
Operazioni:
{\begin{enumerate}
\item il docente deve essere autenticato;
\item entrare nella pagina dedicata alle domande;
\item a seconda della tipologia di domanda che si desidera creare premere il corrispettivo pulsante 'Nuova domanda';
\item durante la creazione della domanda uscire cliccando qualsiasi voce presente nella pagina;
\item confermare il desiderio di uscita dalla creazione della domanda premendo il pulsante relativo sull'avviso mostrato dal sistema per chiedere conferma dell'interruzione;
\item verificare di essere fuori dalla creazione della domanda e verificare che la domanda non sia mai stata creata.
\end{enumerate}
} & ROF31.2 & Non superato\\
TV31.3 & Il docente vuole verificare la creazione di domande vero o falso.
\newline \newline
Operazioni:
{\begin{enumerate}
\item il docente deve essere autenticato;
\item entrare nella pagina dedicata alle domande;
\item premere il pulsante 'Nuova domanda vero o falso';
\item inserire il titolo della domanda;
\item inserire una descrizione;
\item selezionare l'argomento della domanda;
\item selezionare il livello di difficoltà;
\item inserire le parole chiave;
\item inserire facoltativamente un allegato;
\item selezionare la risposta vera;
\item confermare la creazione della domanda;
\item verificare la corretta creazione della domanda.
\end{enumerate}
} & ROF31.3 & Non superato\\
TV31.4 & Il docente vuole verificare la possibilità di creare una domanda a risposta aperta.
\newline \newline
Operazioni:
{\begin{enumerate}
\item il docente deve essere autenticato;
\item entrare nella pagina dedicata alle domande;
\item premere il pulsante 'Nuova domanda a risposta aperta';
\item inserire il titolo della domanda;
\item inserire una descrizione;
\item selezionare l'argomento della domanda;
\item selezionare il livello di difficoltà;
\item inserire le parole chiave;
\item inserire facoltativamente un allegato;
\item inserire la risposta corretta;
\item confermare la creazione della domanda;
\item verificare la corretta creazione della domanda.
\end{enumerate}
} & ROF31.4 & Non superato\\
TV31.5 & Il docente vuole verificare la possibilità di creare una domanda a risposta multipla.
\newline \newline
Operazioni:
{\begin{enumerate}
\item il docente deve essere autenticato;
\item entrare nella pagina dedicata alle domande;
\item premere il pulsante 'Nuova domanda a risposta multipla';
\item inserire il titolo della domanda;
\item inserire una descrizione;
\item selezionare l'argomento della domanda;
\item selezionare il livello di difficoltà;
\item inserire le parole chiave;
\item inserire facoltativamente un allegato;
\item inserire diverse risposte (di tipo multimediale o testuale);
\item selezionare le risposte corrette;
\item confermare la creazione della domanda;
\item verificare la corretta creazione della domanda.
\end{enumerate}
} & ROF31.5 & Non superato\\
TV31.6 & Il docente vuole verificare la possibilità di creare una domanda a completamento testuale.
\newline \newline
Operazioni:
{\begin{enumerate}
\item il docente deve essere autenticato;
\item entrare nella pagina dedicata alle domande;
\item premere il pulsante 'Nuova domanda a completamento testo';
\item inserire il titolo della domanda;
\item inserire una descrizione;
\item selezionare l'argomento della domanda;
\item selezionare il livello di difficoltà;
\item inserire le parole chiave;
\item inserire facoltativamente un allegato;
\item inserire il testo incompleto della domanda;
\item specificare un insieme di parole corrette e sbagliate;
\item confermare la creazione della domanda;
\item verificare la corretta creazione della domanda.
\end{enumerate}
} & ROF31.6 & Non superato\\
TV31.7 & Il docente vuole verificare la creazione di domande a collegamento.
\newline \newline
Operazioni:
{\begin{enumerate}
\item il docente deve essere autenticato;
\item entrare nella pagina dedicata alle domande;
\item premere il pulsante 'Nuova domanda a collegamenti';
\item inserire il titolo della domanda;
\item inserire una descrizione;
\item selezionare l'argomento della domanda;
\item selezionare il livello di difficoltà;
\item inserire le parole chiave;
\item inserire facoltativamente un allegato;
\item inserire la risposta sotto forma di ennupla secondo le modalità stabilite nell'\AdRdoc;
\item confermare la creazione della domanda;
\item verificare la corretta creazione della domanda.
\end{enumerate}
} & ROF31.7 & Non superato\\
TV31.7.1 & Il docente vuole verificare la possibilità di creare una ennupla per le domande a collegamenti.
\newline \newline
Operazioni:
{\begin{enumerate}
\item inserire nella parte iniziale dell'ennupla un file multimediale oppure del testo;
\item inserire nella parte finale dell'ennupla un file multimediale oppure del testo.
\end{enumerate}
} & ROF31.7.1 & Non superato\\
TV31.9 & Il docente vuole verificare la possibilità di interrompere la modifica di una domanda uscendo dal sistema.
\newline \newline
Operazioni:
{\begin{enumerate}
\item il docente deve essere autenticato;
\item entrare nella pagina dedicata alle domande;
\item selezionare la domanda che si vuole modificare;
\item premere il pulsante 'Modifica';
\item durante la modifica della domanda uscire cliccando qualsiasi voce presente nella pagina;
\item confermare il desiderio di uscita dalla modifica della domanda premendo il pulsante relativo sull'avviso mostrato dal sistema per chiedere conferma dell'interruzione;
\item verificare di essere fuori dalla modifica della domanda e verificare che la domanda non abbia subito alcuna modifica.
\end{enumerate}
} & RZF31.9 & N.I.\\
TV31.10 & Il docente vuole verificare la possibilità di modificare una domanda di tipo vero o falso esistente.
\newline \newline
Operazioni:
{\begin{enumerate}
\item il docente deve essere autenticato;
\item entrare nella pagina dedicata alle domande;
\item selezionare la domanda che si desidera modificare tra quelle di tipologia vero o falso;
\item premere il pulsante 'Modifica';
\item eseguire una modifica alle caratteristiche comuni (titolo, descrizione, argomento, livello di difficoltà, allegato, parole chiave); 
\item eseguire una modifica alla veridicità delle risposta;
\item confermare la modifica;
\item verificare la corretta attuazione della modifica.
\end{enumerate}
} & RZF31.10 & N.I.\\
TV31.11 & Il docente vuole verificare la possibilità di modificare una domanda a risposta multipla esistente.
\newline \newline
Operazioni:
{\begin{enumerate}
\item il docente deve essere autenticato;
\item entrare nella pagina dedicata alle domande;
\item selezionare la domanda che si desidera modificare tra quelle a risposta multipla;
\item premere il pulsante 'Modifica';
\item eseguire una modifica alle caratteristiche comuni (titolo, descrizione, argomento, livello di difficoltà, allegato, parole chiave);
\item inserire una nuova risposta (testuale o multimediale) e definire la propria veridicità;
\item modificare una risposta già esistente secondo le modalità descritte nell'\AdRdoc;
\item eliminare una risposta già esistente;
\item confermare la modifica;
\item verificare la corretta attuazione della modifica.
\end{enumerate}
} & RZF31.11 & N.I.\\
TV31.12 & Il docente vuole verificare la possibilità di modificare una domanda a risposta aperta esistente.
\newline \newline
Operazioni:
{\begin{enumerate}
\item il docente deve essere autenticato;
\item entrare nella pagina dedicata alle domande;
\item selezionare la domanda che si desidera modificare tra quelle a risposta aperta;
\item premere il pulsante 'Modifica';
\item eseguire una modifica alle caratteristiche comuni (titolo, descrizione, argomento, livello di difficoltà, allegato, parole chiave);
\item modificare il testo della risposta esistente;
\item confermare la modifica;
\item verificare la corretta attuazione della modifica.
\end{enumerate}
} & RZF31.12 & N.I.\\
TV31.13 & Il docente vuole verificare la possibilità di modificare una domanda a completamento testo esistente.
\newline \newline
Operazioni:
{\begin{enumerate}
\item il docente deve essere autenticato;
\item entrare nella pagina dedicata alle domande;
\item selezionare la domanda che si desidera modificare tra quelle a completamento testo;
\item premere il pulsante 'Modifica';
\item eseguire una modifica alle caratteristiche comuni (titolo, descrizione, argomento, livello di difficoltà, allegato, parole chiave);
\item modificare il testo incompleto della domanda;
\item modificare l'insieme delle risposte possibili;
\item eliminare alcune parole tra le risposte possibili;
\item confermare la modifica;
\item verificare la corretta attuazione della modifica.
\end{enumerate}
} & RZF31.13 & N.I.\\
TV31.14 & Il docente vuole verificare la possibilità di modificare una domanda a collegamenti esistente.
\newline \newline
Operazioni:
{\begin{enumerate}
\item il docente deve essere autenticato;
\item entrare nella pagina dedicata alle domande;
\item selezionare la domanda che si desidera modificare tra quelle a collegamenti;
\item premere il pulsante 'Modifica';
\item eseguire una modifica alle caratteristiche comuni (titolo, descrizione, argomento, livello di difficoltà, allegato, parole chiave);
\item aggiungere una risposta secondo le modalità descritte nell'\AdRdoc;
\item eliminare una risposta;
\item confermare la modifica;
\item verificare la corretta attuazione della modifica.
\end{enumerate}
} & RZF31.14 & N.I.\\
TV31.15 & Il docente vuole verificare la possibilità di eliminare una domanda dal sistema.
\newline \newline
Operazioni:
{\begin{enumerate}
\item il docente deve essere autenticato;
\item entrare nella pagina dedicata alle domande;
\item selezionare la domanda che si desidera eliminare;
\item premere il pulsante 'Elimina';
\item confermare l'eliminazione;
\item verificare la corretta eliminazione della domanda dal sistema.
\end{enumerate}
} & ROF31.15 & Non superato\\
\rowcolor{white}
\caption{Test di validazione}
\end{tabella}
