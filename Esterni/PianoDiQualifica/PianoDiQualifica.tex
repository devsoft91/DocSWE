% Nome del file: documento.tex
% Percorso: \gl{template}
% Autore: Vault-Tech
% Data creazione: 27.12.2016
% E-mail: vaulttech.swe@gmail.comcom
%
% Diario delle modifiche: interno al file.

\documentclass[a4paper, titlepage]{article}

\usepackage[margin=3cm]{geometry}
\usepackage{../../Stile}
\usepackage{../../Comandi}

\setcounter{secnumdepth}{5}
\setcounter{tocdepth}{5}

\def\NOME{Piano di Qualifica}
\def\VERSIONE{3.0}
\def\DATA{06.04.2016}
\def\REDATTORE{Filippo Tesser \\ & Michela De Bortoli \\ & Vassilikì Menarin \\ & Rudy Berton \\ & Giacomo Beltrame}
\def\VERIFICATORE{Giacomo Beltrame \\ & Rudy Berton}
\def\RESPONSABILE{Michela De Bortoli}
\def\USO{Esterno}
\def\DISTRIBUZIONE{\COMMITTENTE \\ & \CARDIN \\ & \PROPONENTE}

\usepackage{graphicx}
\begin{document}
\pagestyle{fancy}	
\pagenumbering{Roman}
\rfoot{Pagina \thepage{} di \pageref{lastromanpage}}

\maketitle

\begin{diario}
	\recap{Approvazione del documento}{Michela De Bortoli}{Responsabile}{06.04.2016}{3.0}
	\recap{Correzione errori individuati}{Michela De Bortoli}{Analista}{06.04.2016}{2.10}
	\recap{Verifica dell'intero documento}{Rudy Berton}{Verificatore}{05.04.2016}{2.9}
	\recap{Stesura appendice D}{Giacomo Beltrame}{Analista}{04.04.2016}{2.8}
	\recap{Verifica appendici A e B}{Giacomo Beltrame}{Verificatore}{03.04.2016}{2.7}
	\recap{Stesura test di integrazione}{Rudy Berton}{Amministratore}{02.04.2016}{2.6}
	\recap{Stesura test di sistema}{Vassilikì Menarin}{Progettista}{02.04.2016}{2.5}
	\recap{Modifica della sezione A.3.0.3 dell'appendice}{Filippo Tesser}{Analista}{01.04.2016}{2.4}
	\recap{Incremento test di accettazione}{Michela De Bortoli}{Progettista}{01.04.2016}{2.3}
	\recap{Inizio stesura specifica dei test (appendice B)}{Michela De Bortoli}{Progettista}{31.03.2016}{2.2}
	\recap{Incremento dell'appendice A}{Filippo Tesser}{Analista}{31.03.2016}{2.1}
	\recap{Approvazione documento}{Miki Violetto}{Responsabile}{23.02.2016}{2.0}
	\recap{Verifica delle sezioni modificate}{Rudy Berton}{Verificatore}{22.02.2016}{1.2}
	\recap{Revisione correttiva dei contenuti rispetto alle segnalazioni del committente}{Giacomo Beltrame}{Analista}{20.02.2016}{1.1}
	\recap{Approvazione documento}{Vassilikì Menarin}{Responsabile}{20.01.2016}{1.0}
	\recap{Verifica del documento}{Simone Boccato}{Verificatore}{19.01.2016}{0.9}
	\recap{Stesura appendice D}{Rudy Berton}{Analista}{18.01.2016}{0.8}
	\recap{Correzione errori segnalati}{Rudy Berton}{Analista}{16.01.2016}{0.7}
	\recap{Verifica del documento}{Filippo Tesser}{Verificatore}{15.01.2016}{0.6}
	\recap{Stesura appendici A, B e C}{Rudy Berton}{Analista}{11.01.2016}{0.5}
	\recap{Fine stesura Gestione della qualità e stesura sezione Gestione amministrativa della revisione}{Rudy Berton}{Analista}{08.01.2016}{0.4}
	\recap{Inizio stesura Gestione della qualità}{Rudy Berton}{Analista}{05.01.2016}{0.3}
	\recap{Stesura sezione Obiettivi di qualità}{Rudy Berton}{Analista}{03.01.2016}{0.2}
	\recap{Stesura sezione Introduzione}{Rudy Berton}{Analista}{02.01.2016}{0.1}
\end{diario}

\newpage
%N.B. Spostate comando "\label{lastromanpage}" di fianco all'ultimo indice che usate (contents, figures o tables)
\tableofcontents
%\newpage
%\listoffigures 
\newpage
\listoftables
\label{lastromanpage}

\newpage
\clearpage	
\pagenumbering{arabic}
\rfoot{Pagina \thepage{} di \pageref*{LastPage}}

\hypersetup{linkcolor=blue}
\section{Introduzione}
\subsection{Scopo del documento}
Il presente documento ha lo scopo di descrivere gli obiettivi di qualità, di processo e di prodotto che si intendono perseguire nella realizzazione del progetto, e la descrizione di una strategia di verifica e validazione adottata dal \gl{team} per il raggiungimento di tali obiettivi.

\subsection{Scopo del prodotto}
\SCOPO

\subsection{Glossario}
\GLOSSARIO

\subsection{Riferimenti}
\subsubsection{Riferimenti normativi}
\begin{itemize}
\item \bold{Capitolato d'appalto C5:} Quizzipedia: \gl{software} per la gestione di questionari
\newline \url{http://www.math.unipd.it/~tullio/IS-1/2015/Progetto/C5.pdf};
\item \bold{Norme di progetto: } \NdPdoc;
\item \bold{Analisi dei requisiti: } \ARdoc;
\item \bold{Piano di progetto: } \PdPdoc;
\item \bold{Specifica tecnica:} \STdoc.
\end{itemize}

\subsubsection{Riferimenti informativi}
\label{rifinfo}
\begin{itemize}
\item \bold{Qualità del \gl{software} - Ercole F. Colonese:}
	\newline \url{http://www.colonese.it/00-Manuali_Pubblicatii/06-Qualit%C3%A0Software_v2.pdf};
\item \bold{La qualità del \gl{software} secondo il lo ISO/IEC9126 - Ercole F. Colonese:} 
	\newline \url{http://www.colonese.it/00-Manuali_Pubblicatii/07-ISO-IEC9126_v2.pdf};
\item \bold{Metriche del \gl{software} G - Ercole F. Colonese:}
	\newline \url{http://www.colonese.it/00-Manuali_Pubblicatii/08-Metriche%20del%20software_v1.0.pdf};
\item \bold{The Guide to the \gl{Software} Engineering Body of Knowledge (SWEBOK Guide v3.0) - IEEE Computer Society:} Chapter 10 - \gl{Software} Quality
	\newline \url{http://www.computer.org/web/swebok/;jsessionid=e21ab6073d3593a8523de74bddd2};
\item \bold{PDCA:}
	\newline \url{https://en.wikipedia.org/wiki/PDCA};
\item \bold{Capability Maturity :}
	\newline \url{https://en.wikipedia.org/wiki/Capability_Maturity_};
\item \bold{ISO 9126:}
	\newline \url{https://en.wikipedia.org/wiki/ISO/IEC_9126};
\item \bold{Indice Gulpease:}
	\newline \url{https://it.wikipedia.org/wiki/Indice_Gulpease};
\item \bold{Slide del corso Ingegneria del Software - Misurazioni del Software:}
	\newline \url{http://www.math.unipd.it/~rcardin/sweb/B02.pdf};
\item \bold{Complessità ciclomatica:}
	\newline \url{https://it.wikipedia.org/wiki/Complessit%C3%A0_ciclomatica}.2
\end{itemize}

\newpage
\section{Obiettivi di qualità}
In questa sezione vengono presentate le caratteristiche che definiscono la qualità di prodotto e di processo che il \gl{team} si impegna a soddisfare nella realizzazione del progetto. 
\newline Per ogni caratteristica viene stabilita un'unità di misura che possa quantificarla ed una soglia di accettabilità che il gruppo si prefigge di raggiungere, e possibilmente superare con l'obiettivo di un miglioramento continuo della qualità.

\subsection{Qualità di processo}
Non è possibile distinguere completamente la qualità di processo dalla qualità di prodotto in quanto i risultati dei processi comprendono a loro volta prodotti. Determinare se un processo ha pertanto la capacità di realizzare costantemente prodotti di qualità desiderata non è semplice. 
\newline Per cercare di arginare questa difficoltà vengono stabilite delle caratteristiche da seguire nel tentativo di perseguire una qualità di processo ottimale durante l'applicazione congiunta dei li \gl{PDCA} e \gl{CMM}.
\newline Le caratteristiche considerate sono:

\subsubsection{Pianificazione temporale}
Rispettare la pianificazione temporale stabilita nel \doc{Piano di Progetto} è indice di un lavoro che si sta svolgendo nel migliore dei modi; qualora si verifichi un ritardo invece, un campanello d'allarme indicherà che i processi non ancora conclusi non disporranno del grado di qualità atteso per tale scadenza.
\begin{itemize}
\item \bold{Metrica:} l'unità di misura per valutare questa proprietà è la Schedule Variance.
\item \bold{Soglia di accettabilità:} si riterrà accettabile un ritardo al massimo di 4 giorni rispetto a quanto pianificato nel \doc{Piano di Progetto}.
\end{itemize}
Per maggiori dettagli sulla metrica si veda la sezione \hyperref[par:SV]{Schedule Variance} in appendice.

\subsubsection{Miglioramento costante}
Per valutare il grado di formalità e ottimizzazione dei processi attuali, con l'obiettivo di apportarne un ulteriore miglioramento alla qualità, viene assunto il lo \gl{CMM} (Capability Maturity ).
\begin{itemize}
\item \bold{Metrica:} l'unità considerata è la struttura a livelli di cui è rappresentata la Maturità (Maturity) all'interno del lo \gl{CMM}.
\item \bold{Soglia di accettabilità:} si ritiene sufficiente raggiungere il livello 2 (Repeatable), con l'intenzione di migliorare ulteriormente i processi per arrivare anche ad un livello superiore.
\end{itemize}
Per maggiori dettagli sulla metrica si veda la sezione \hyperref[par:cmm]{Capability Maturity } in appendice. 

\subsubsection{Stima del costo}
Una caratteristica, a lato economico, che stabilisce la qualità dei processi è data dall'aumento dei costi prefissati nel \doc{Piano di Progetto}: qualora un processo non abbia la qualità ottimale richiesta necessita di un miglioramento ulteriore, causando l'aumento dei costi per farlo.
\newline Questo innalzamento del budget non dovrà verificarsi all'interno del progetto se ad ogni processo sarà stata garantita sempre la massima qualità possibile.
\begin{itemize}
\item \bold{Metrica:} l'unità di misura per valutare l'aumento dei costi stabiliti è la Cost Variance.
\item \bold{Soglia di accettabilità:} sarà accettabile quando i costi saranno al massimo superiori del 10\% rispetto a quanto stimato; si cercherà in ogni modo che le stime presentate nel \doc{Piano di Progetto} siano rispettate, senza aumento alcuno.
\end{itemize}
Per maggiori dettagli sulla metrica si veda la sezione \hyperref[par:CV]{Cost Variance} in appendice.

\subsection{Qualità di prodotto}
\label{sec:qualprod}
I prodotti che vengono generati nello sviluppo di questo progetto sono principalmente due: i documenti ed il \gl{software}. Vengono pertanto stabiliti gli obiettivi da raggiungere per la diversa tipologia di prodotto considerato.

\subsubsection{Qualità nei documenti}
La stesura dei documenti all'interno di un progetto ricopre un aspetto principale per chiunque collabori nella sua realizzazione come fonte d'informazione, resoconto evolutivo del lavoro svolto, etc. Proprio per la loro importanza i documenti devono presentare un livello di qualità alto, non solo a progetto concluso, ma ad ogni passo dello sviluppo.
\newline Seguendo le caratteristiche riportate in questa sezione il \gl{team} cercherà di redigere della documentazione corretta dal punto di vista grammaticale e contenutistico.

\myparagraph{Correttezza ortografica}
I documenti per essere perfetti in primis devono essere privi di errori grammaticali e ortografici. Sarà effettuata una prima correzione manuale al documento da parte del \italics{Verificatore} ed una successiva verifica automatizzata di supporto attraverso l'uso di uno strumento (\gl{Hunspell}) per garantire una perfetta correttezza ortografica (non è possibile affidarsi unicamente ad un verificatore automatico dal momento che non riesce ad assicurare l'individuazione degli errori al 100\%).  
\begin{itemize}
\item \bold{Metrica:} l'unità di misura è il numero di errori ortografici riscontrati una seconda volta dopo esser stati già segnalati (ma non corretti) durante una precedente verifica del documento, da parte dello strumento automatico e del \italics{Verificatore}.
\item \bold{Soglia di accettabilità:} si accetta di trovare, durante una successiva verifica del documento, al massimo il 5\% degli errori ortografici segnalati durante la verifica precedente (su 100 errori segnalati è accettabile che 5 non siano stati corretti e pertanto risultino segnalati nella verifica successiva).
\end{itemize}
Per maggiori dettagli sulla metrica si veda la sezione \hyperref[par:errort]{Errori ortografici} in appendice.

\myparagraph{Leggibilità e comprensibilità}
Un documento poiché risulti utile al proprio fine deve essere leggibile e comprensibile da tutti coloro che possono usufruirne (si consideri come utente tipo un individuo con almeno licenza di istruzione superiore).
\begin{itemize}
\item \bold{Metrica:} l'unità di riferimento è data dall'\gl{Indice Gulpease}.
\item \bold{Soglia di accettabilità:} si riterrà sufficiente un valore superiore a 40 dato dal calcolo dell'\gl{Indice Gulpease}.
\end{itemize}
Per maggiori dettagli sulla metrica si veda la sezione \hyperref[par:IG]{\gl{Indice Gulpease}} in appendice.

\myparagraph{Correttezza dei contenuti}
Oltre ad essere esatto dal punto di vista grammaticale un ottimo documento deve presentare del contenuto che sia corretto, coerente e preciso. Sarà compito del \italics{Verificatore} individuare e segnalare gli errori di contenuto presenti all'interno del documento.
\begin{itemize}
\item \bold{Metrica:} il numero di errori concettuali precedentemente segnalati ma non corretti è l'unità di misura che verrà presa in considerazione.
\item \bold{Soglia di accettabilità:} la soglia di accettabilità è inferiore al 5\%; anche se il \gl{team} si prefiggerà il compito di correggere immediatamente qualsiasi errore concettuale segnalato. 
\end{itemize}
Per maggiori dettagli sulla metrica si veda la sezione \hyperref[par:errcon]{Errori concettuali} in appendice.

\myparagraph{Adesione alle norme interne}
Ogni documento deve esser redatto secondo le norme definite nel documento \doc{Norme di Progetto}, in tal modo si evitano fraintendimenti sulla lettura da parte dei membri all'interno di un gruppo di sviluppo.
\begin{itemize}
\item \bold{Metrica:} l'unità considerata riguarda il numero di errori  segnalati, ma non corretti, non rispettanti le norme stabilite ad inizio progetto (mancanza del pedice per i vocaboli del \doc{Glossario}, mancanza dei riferimenti ad altri documenti, stile tipografico delle parole sbagliato, etc.). Sarà compito del \italics{Verificatore} individuare tali errori manualmente.
\item \bold{Soglia di accettabilità:} la soglia di accettabilità massima corrisponde al 5\% di errori; si cercherà ad ogni modo di rispettare in modo preciso le norme stabilite.
\end{itemize}
Per maggiori dettagli sulla metrica si veda la sezione \hyperref[par:errfor]{Errori di forma} in appendice.

\subsubsection{Qualità nel software}
La qualità di processo, come già detto, influenza direttamente la qualità di un prodotto \gl{software}: se si riesce ad ottenere una buona qualità di processo sarà possibile realizzare un prodotto di qualità ottimale; al contrario un prodotto scarso indicherà che alla base la qualità dei processi non era adeguata.
\newline Il lo che rappresenta in modo ottimale le qualità di un prodotto \gl{software} è lo standard \iso{ISO/IEC 9126}.
\newline Le sue caratteristiche sono:

\myparagraph{Funzionalità}
Tale caratteristica rappresenta la capacità del \gl{software} di fornire le funzioni, espresse ed implicite, necessarie per soddisfare nel modo più completo possibile i requisiti dichiarati nel documento \ARdoc, in modo tale che garantiscano la sicurezza del prodotto, dei suoi componenti e si adegui alle norme.
\begin{itemize}
\item \bold{Metrica:} l'unità di misura per valutare questa caratteristica si basa sul numero di requisiti soddisfatti in modo completo.
\item \bold{Soglia di accettabilità:} la soglia minima di sufficienza è ottenuta qualora almeno tutti i requisiti di natura obbligatoria siano stati soddisfatti.
\end{itemize}
Per maggiori dettagli sulla metrica si veda la sezione \hyperref[par:req]{Requisiti rispettati} in appendice.

\newpage
\myparagraph{Affidabilità}
L'affidabilità rappresenta la capacità di un prodotto \gl{software} di mantenere il livello di prestazione in caso di variazioni dell'ambiente circostante. Deve pertanto essere robusto, di facile ripristino e recupero in caso di errori ed aderire alle norme.
\begin{itemize}
\item \bold{Metrica:} la quantità di esecuzioni dell'applicazione avvenute con successo.
\item \bold{Soglia di accettabilità:} la soglia di accettabilità viene raggiunta qualora almeno il 98\% delle esecuzioni produca il risultato atteso.
\end{itemize}
Per maggiori dettagli sulla metrica si veda la sezione \hyperref[par:out]{Output attesi} in appendice.

\myparagraph{Usabilità}
È la capacità per un prodotto di essere facilmente comprensibile, apprendibile ed utilizzabile da parte degli utenti finali, in modo tale che soddisfi le aspettative dell'utente e al tempo stesso aderisca alle norme. 
\begin{itemize}
\item \bold{Metrica:} non è facile per questa caratteristica stabilire un'unità di misura oggettiva. Dipende da molti fattori esterni al sistema come le facoltà psicofisiche dell'utente, gli strumenti a disposizione per interagire col sistema, etc.
\item \bold{Soglia di accettabilità:} non esistono metriche obiettive riguardanti l’usabilità. Si cercherà comunque di garantire l'aspetto dell'accessibilità soddisfando gli standard \gl{web} descritti dal \gl{W3C}, portando in questo modo benefici all'usabilità del prodotto.
\end{itemize}
Per maggiori dettagli sulla metrica si veda la sezione \hyperref[par:web]{Validazione \gl{web}} in appendice.

\myparagraph{Efficienza}
L'efficienza è la capacità di un prodotto \gl{software} di realizzare le funzioni richieste nel minor tempo possibile e con l'uso minimo di risorse necessarie.
\begin{itemize}
\item \bold{Metrica:} l'unità di misura riguarda il tempo di attesa nella visualizzazione e nel funzionamento dell'applicativo sui diversi dispositivi.
\item \bold{Soglia di accettabilità:} si riterranno accettabili tempi di attesa dell'applicativo non superiori ai 5 secondi durante una transizione eseguita dall'utente.
\end{itemize}
Per maggiori dettagli sulla metrica si vedano le sezioni \hyperref[par:web]{Validazione \gl{web}} e \hyperref[par:greff]{Grado di efficienza} in appendice.

\myparagraph{Manutenibilità}
Rappresenta la capacità di un prodotto \gl{software} di essere modificato (senza provocare in seguito effetti indesiderati), corretto in caso di errori o adattato ai cambiamenti dell'ambiente, sempre con l'accortezza che rimanga verificabile.
\begin{itemize}
\item \bold{Metrica:} come unità di misura si considera il numero di correzioni di anomalie non andate a buon fine (che richiedono, cioè, un secondo intervento correttivo) rispetto al numero totale di anomalie risolte.
\item \bold{Soglia di accettabilità:} per rispettare questa qualità si riterranno accettabili valori inferiori al 25\% (su quattro anomalie risolte è accettabile che una sola debba esser risolta una seconda volta).
\end{itemize}
Per maggiori dettagli sulla metrica si veda la sezione \hyperref[par:anins]{Anomalie insufficienti} in appendice.

\myparagraph{Portabilità}
L'applicazione dev'essere di facile installazione, esecuzione, adattamento e compatibilità in ambienti \gl{hardware}/\gl{software} diversificati.
\begin{itemize}
\item \bold{Metrica:} numero di \gl{browser} che supportano l'applicazione in modo eccellente, qualsiasi sia il dispositivo utilizzato. 
\item \bold{Soglia di accettabilità:} La soglia di sufficienza per garantire tale caratteristica si avrà quando l'applicativo sarà supportato almeno dalle ultime versioni dei seguenti \gl{browser}: \gl{Google Chrome} , \gl{Mozilla Firefox}, \gl{Safari}, \gl{Opera} ed \gl{Internet Explorer/Edge} (tutti per la versione desktop mentre sono sufficienti i primi tre per dispositivi mobile/tablet).
\end{itemize}
Per maggiori dettagli sulla metrica si veda la sezione \hyperref[par:web]{Validazione \gl{web}} in appendice.
\\ 
\newline Le qualità descritte sopra sono quelle interne (proprietà intrinseche del \gl{software}) ed esterne (proprietà che hanno rilevanza solo per l’utente) di un prodotto \gl{software} che il gruppo si prefigge di garantire nello svolgimento dell'applicazione. Queste due categorie vanno ad incidere in una terza tipologia di qualità: la qualità in uso che rappresenta il punto di vista dell'utente finale sulla qualità, una volta che il prodotto sarà completato e in fase di utilizzo. Di questa il \gl{team} non se ne occuperà dal momento che il progetto terminerà prima del rilascio effettivo del \gl{software}.

Per maggiori informazioni sullo standard ISO/IEC 9126 si rimanda alla visione dei \hyperref[rifinfo] {riferimenti informativi}.

\section{Gestione della qualità: visione generale di strategia}

\subsection{Tecniche di controllo qualità di processo}
La qualità di un processo viene raffinata attraverso un miglioramento continuo grazie al lo \gl{PDCA} (Plan-Do-Check-Act) che in modo incrementale apporta ulteriore qualità a quella precedente. Per rendere effettivo il miglioramento è necessario che ogni attività del lo abbia una scrupolosa pianificazione (possibilmente) proattiva ed un'ottima strategia di verifica che, svolta fin dall'inizio, dovrà permettere una serie di iterazioni all'interno di una determinata fase fino al raggiungimento di un grado di qualità accettabile per poter procedere ad un ulteriore incremento.
\newline La valutazione della qualità di un processo, per poter parlare di un suo possibile miglioramento, dev'essere pertanto quantificata studiando le sue caratteristiche d'interesse; ciò sarà possibile basandosi sul lo \gl{CMM} e sul confronto di performance diverse in momenti differenti.
\newline Un ulteriore indice di valutazione per i processi può esser assunto dalla qualità dei loro prodotti, dal momento che esiste una forte correlazione diretta tra processo e prodotto.

Si vedano i \hyperref[rifinfo]{riferimenti informativi} riguardanti il \gl{PDCA} e il \gl{Capability Maturity } per maggiori chiarimenti sui li adottati.

\subsection{Tecniche di controllo qualità di prodotto}
Il controllo della qualità del prodotto è assicurato da due tipi di processi, quello di verifica e quello di validazione e dal rispetto delle norme.
\begin{itemize}
\item \bold{Verifica:} questo processo si occupa di accertare che l'esecuzione delle attività di processo siano corrette, senza alcun errore. È pertanto un'azione che dev'esser svolta costantemente su ogni attività o risultato che fa progredire il progetto da una \gl{baseline} a quella successiva per poterne assicurare l'assenza di errori.
\\ Il resoconto delle attività di verifica sono descritte in \hyperref[app:valtest]{appendice}.
\\
\item \bold{Validazione:} tale attività viene svolta come atto conclusivo per accertare che il prodotto risultante rispecchi le aspettative sfruttando un metodo sistematico, disciplinato e quantificabile. Sarà pertanto compito del \gl{team} effettuare alla fine dei test per validare il prodotto da consegnare al committente.
\\
\item \bold{Norme:} il prodotto dovrà rispettare le norme e l'uso degli strumenti elencati all'interno del documento \doc{Norme di Progetto}, redatto dal gruppo fin dall'inizio del progetto.
\end{itemize}

\subsection{Responsabilità} 
\label{sec:repo}
La qualità del \gl{software} realizzato è responsabilità di tutti, nessuno escluso. Ogni membro di un gruppo coinvolto in un progetto \gl{software} contribuisce con il proprio lavoro a costruire (positivamente o negativamente) la qualità del prodotto finale e dei processi che ne permettono la realizzazione.
\newline Proprio per questo, qualunque ruolo ricopra, ogni individuo all'interno del \gl{team} deve svolgere le proprie attività con la massima cura al fine di ottenere un grado di qualità alto.
\newline Le responsabilità definite per la qualità a cui ogni ruolo deve adempiere vengono di seguito elencate.
\begin{description}

\item \bold{\italics{Responsabile}}
\begin{itemize}
\item[-]Assicurare che ogni processo sia valutato in maniera oggettiva ed opportunamente migliorato per renderlo sempre più semplice, efficace e conforme alle necessità.
\item[-]Assegnare le responsabilità relative all’assicurazione della qualità a persone indipendenti dallo sviluppo in modo tale che possano verificare, validare e valutare la qualità raggiunta.
\end{itemize}
\ 
\item \bold{\italics{Amministratore}}
\begin{itemize}
\item[-]Assicurare la definizione del processo per la gestione della qualità che ne preveda una fase di pianificazione ed una di controllo.
\item[-]Pianificare la qualità del progetto assicurando la disponibilità delle risorse necessarie sia realizzative che di verifica e validazione.
\item[-] Incentivare nella realizzazione di un processo di verifica sempre più automatizzabile (aumentando il grado d'efficienza).
\item[-]Controllare l’esecuzione di tutte le attività pianificate che assicurino il raggiungimento del livello qualitativo atteso.
\end{itemize}
\ 
\item \bold{\italics{Analista}}
\begin{itemize}
\item[-] Definire e documentare i requisiti funzionali e quelli qualitativi (non funzionali).
\item[-] Deve assicurarsi di aderire agli standard e alle norme riguardanti la documentazione prodotta.
\end{itemize}
\ 
\item \bold{\italics{Progettista}}
\begin{itemize}
\item[-] Indirizzare nelle specifiche tecniche e funzionali anche i requisiti di qualità.
\item[-] Realizzare la progettazione in modo da indirizzare completamente, correttamente ed efficacemente anche i requisiti di qualità.
\item[-] Aderire a tutti gli standard applicabili (standard di programmazione, di documentazione, ...).
\end{itemize}
\ 
\item \bold{\italics{Programmatore}}
\begin{itemize}
\item[-] Sviluppare il codice secondo le norme redatte all'interno del \gl{team} con alto livello di qualità.
\item[-] Aderire a tutti gli standard applicabili (standard di programmazione, di documentazione, ...).
\item[-] Deve fornire test necessari ad eseguire verifiche sulle singole unità prodotte, in modo da verificare il reale livello
qualitativo raggiunto in linea con la criticità e complessità dell’applicazione.
\end{itemize}
\ 
\item \bold{\italics{Verificatore}}
\begin{itemize}
\item[-] Tracciare gli errori rilevati in ciascuna fase del ciclo di sviluppo per poter essere risolti nella stessa fase, verificandone la corretta rimozione.
\item[-] Controllare la corretta esecuzione dei test verificandone lo stato di completamento e stilando un rapporto periodico. 
\item[-] Eseguire le attività di verifica presenti in questo documento valutando processi e prodotti secondo le misure e metriche stabilite.
\end{itemize}

\end{description}
\ 
\newline Per maggiori dettagli circa i ruoli e i compiti assegnati si rimanda al documento delle \doc{Norme di Progetto}.

\newpage
\appendix
\section{Misure e metriche}
\label{sec:metr}
In questa sezione vengono presentate le metriche (criteri consolidati nel tempo) che permettano di quantificare, misurare e valutare in modo preciso la qualità dei processi e dei prodotti realizzati durante i processi di verifica.

\subsection{Misure}
Per ogni misura effettuata attraverso l'utilizzo di una specifica metrica viene eseguita una valutazione sul valore ottenuto facendolo rientrare in uno dei seguenti giudizi:
\begin{itemize}
	\item valore \bold{negativo}: un valore che venga considerato negativo non deve essere accettato, qualunque sia l'ambito di provenienza.
	\item valore \bold{accettabile}: un valore con giudizio accettabile deve esser preso in considerazione positivamente poiché avrà raggiunto almeno la soglia minima di accettabilità.
	\item valore \bold{ottimale}: se il risultato dell'applicazione di una metrica viene considerato ottimale vuol dire che l'attività generatrice non necessita di ulteriori verifiche poiché è stato raggiunto il massimo valore possibile.
\end{itemize} 

\subsection{Metriche per i processi}
Le metriche utilizzate dal gruppo per la valutazione della qualità dei processi sono descritte in questa sezione.

\myparagraph{Schedule Variance}
\label{par:SV}
La Schedule Variance (SV) è un indicatore di efficacia dei processi utilizzato per valutare l'andamento temporale delle attività: se la loro conclusione avviene in ritardo, in anticipo oppure in concordanza coi tempi prefissati nel documento \doc{Piano di Progetto}.
\newline Formula della Schedule Variance per una attività:
\begin{displaymath}
SV= \mbox{data conclusione reale} - \mbox{data conclusione pianificata}
\end{displaymath}
\\Giudizio dei risultati ottenibili:
\begin{itemize}
\item Valore negativo: un valore maggiore di 4 giorni rispetto al tempo pianificato.
\item Valore accettabile: un valore inferiore o uguale a 4 giorni rispetto al tempo pianificato.
\item Valore ottimale: un valore minore o uguale a 0. Indica che l'attività si è conclusa perfettamente nei tempi stabiliti o addirittura in anticipo. 
\end{itemize}

\myparagraph{Capability Maturity }
\label{par:cmm}
Questa metrica viene utilizzata come indice principale per la valutazione di un processo che verrà classificato in uno dei cinque livelli previsti dal lo sulla base della Maturity.
\par Giudizio dei risultati ottenibili:
\begin{itemize}
\item Valore negativo: livello 1 (Initial).
\item Valore accettabile: livello 2 (Repeatable) e livello 3 (Defined).
\item Valore ottimale: livello 4 (Managed) e livello 5 (Optimizing), anche se quest'ultimo è difficilmente raggiungibile.
\end{itemize}

\myparagraph{Cost Variance}
\label{par:CV}
La Cost Variance (CV) è un indicatore del costo dei processi. Qualora il costo effettivo sia maggiore del preventivo stilato nel documento \doc{Piano di Progetto} si evidenzia che sono state necessarie maggiori risorse nello svolgere le attività rispetto a quanto era stato quantificato inizialmente.
\\Formula della Cost Variance per un processo:
\begin{displaymath}
CV= \mbox{costo effettivo} - \mbox{costo preventivato}
\end{displaymath}
\\ Giudizio dei risultati ottenibili:
\begin{itemize}
\item Valore negativo: un valore maggiore (o uguale) del 10\% dei costi pianificati
\item Valore accettabile: un valore minore del 10\% rispetto ai costi pianificati.
\item Valore ottimale: un valore pari allo 0\%. In questo caso il valore indica come il costo preventivato di un processo sia realmente concorde a quanto pianificato. 
\end{itemize}

\subsection{Metriche per i prodotti}

\myparagraph{Per i documenti}
Vengono di seguito presentate le metriche che verranno utilizzate nei processi di verifica dei documenti prodotti.

\mysubparagraph{Errori ortografici}
\label{par:errort}
Questa metrica serve per identificare quanto un documento sia corretto dal punto di vista ortografico. Il suo giudizio si avvale di due controlli: uno di tipo automatico attraverso uno strumento \gl{software} e l'altro di tipo manuale da parte del \italics{Verificatore}, necessario conoscendo la natura fallibile dei correttori automatici.
\newline Questa metrica misura il numero di errori riscontrati, attraverso le due modalità di verifica, ma non corretti immediatamente.
\newline Formula:
\begin{displaymath}
\mbox{Errori ortografici}= \frac{\mbox{numero errori non corretti}}{\mbox{numero totale errori segnalati}}*100
\end{displaymath}
\par Giudizio dei risultati ottenibili:
\begin{itemize}
\item Valore negativo: un valore superiore al 5\%. 
\item Valore accettabile: un valore inferiore o uguale al 5\%.
\item Valore ottimale: un valore pari a 0.
\end{itemize}

\mysubparagraph{Indice Gulpease}
\label{par:IG}
L'\gl{Indice Gulpease} è un indice di leggibilità di un testo tarato sulla lingua italiana che presenta il vantaggio di utilizzare la lunghezza delle parole per facilitarne il calcolo automatico. Si considerano due variabili linguistiche: la lunghezza della parola e la lunghezza della frase rispetto al numero delle lettere.
\newline I risultati sono compresi tra 0 (leggibilità bassa) e 100 (leggibilità alta); in generale risulta che testi con un indice
\begin{itemize}
\item[-]inferiori a 80 sono difficili da leggere per chi ha la
licenza elementare;
\item[-]inferiore a 60 sono difficili da leggere per chi ha la
licenza media;
\item[-]inferiore a 40 sono difficili da leggere per chi ha un
diploma superiore.
\end{itemize}
Nella realizzazione del progetto, e nella valutazione dei risultati ottenuti da tale metrica, considereremo che la documentazione scritta sia indirizzata a persone colte e competenti con un alto livello di istruzione.
\\
\newline Formula:
\begin{displaymath}
\mbox{\gl{Indice Gulpease}}= 89+\frac{300*\mbox{(numero delle frasi)}-10*\mbox{(numero delle lettere)}}{\mbox{numero delle parole}}
\end{displaymath}
\\
\newline Giudizio dei risultati ottenibili:
\begin{itemize}
\item Valore negativo: un valore inferiore a 40. 
\item Valore accettabile: un valore superiore o uguale a 40.
\item Valore ottimale: un valore compreso tra 70 e 100.
\end{itemize}

\mysubparagraph{Errori concettuali}
\label{par:errcon}
Questa metrica serve ad indicare la correttezza di un documento dal punto di vista del proprio contenuto, se i concetti espressi sono corretti e coerenti in tutto il documento.
\newline Il valore ottenuto da questa metrica rappresenta il numero di errori concettuali che non sono stati corretti dopo esser stati segnalati dal \italics{Verificatore} durante la precedente verifica del documento.
\newline Formula:
\begin{displaymath}
\mbox{Errori concettuali}=\frac{\mbox{numero errori non corretti}}{\mbox{numero totale errori segnalati}}*100
\end{displaymath}
\\
\newline Giudizio dei risultati ottenibili:
\begin{itemize}
\item Valore negativo: un valore superiore al 5\%. 
\item Valore accettabile: un valore inferiore o uguale al 5\%.
\item Valore ottimale: un valore uguale allo 0\%.
\end{itemize}

\mysubparagraph{Errori di forma}
\label{par:errfor}
Viene utilizzata questa unità di misura per verificare quanto un documento rispetti le regole strutturali descritte nelle \doc{Norme di Progetto}. 
\newline La metrica si basa sul numero di errori segnalati dal \italics{Verificatore} che non sono stati corretti successivamente.
\newline Formula:
\begin{displaymath}
\mbox{Errori di forma}=\frac{\mbox{numero errori non corretti}}{\mbox{numero totale errori segnalati}}*100
\end{displaymath}
\\
\newline Giudizio dei risultati ottenibili:
\begin{itemize}
\item Valore negativo: un valore superiore al 5\%. 
\item Valore accettabile: un valore inferiore o uguale al 5\%.
\item Valore ottimale: un valore uguale allo 0\%.
\end{itemize}

\myparagraph{Per i prodotti software}
In questa sezione si presentano alcune delle metriche di cui il \gl{team} si avvalerà per eseguire le verifiche dei prodotti \gl{software} durante lo sviluppo del progetto. Tale sezione potrà subire nelle attività successive un ulteriore incremento di contenuti grazie all'effettiva realizzazione di un qualche prodotto \gl{software}. 

\mysubparagraph{Requisiti rispettati}
\label{par:req}
Tale metrica permette di identificare il numero di requisiti che sono stati realizzati all'interno del progetto.
\newline I requisiti si dividono in funzionali, vincolanti, qualitativi e prestazionali; ogni tipo di requisito a sua volta può essere marcato obbligatorio oppure facoltativo. Il gruppo cercherà come primo obiettivo di svolgere tutti i requisiti obbligatori per poi passare a quelli opzionali.
\newline Formula per calcolare il numero di requisiti realizzati a seconda della tipologia di appartenenza:
\begin{displaymath}
\mbox{Requisiti rispettati}=\frac{\mbox{numero requisiti svolti}}{\mbox{numero totale requisiti}}*100
\end{displaymath}
\
\newline Giudizio dei risultati ottenibili:
\begin{itemize}
\item Valore negativo: il numero di requisiti obbligatori svolti è inferiore al 100\%. 
\item Valore accettabile: il 100\% dei requisiti obbligatori sono stati svolti correttamente.
\item Valore ottimale: il 100\% dei requisiti, di natura obbligatoria e opzionale, sono stati realizzati.
\end{itemize}

\mysubparagraph{Output attesi}
\label{par:out}
La valutazione di questa metrica permette di giudicare a che livello di affidabilità sia il prodotto durante la fase di verifica.
Misura pertanto il numero di esecuzioni avvenute con successo, ovvero che hanno prodotto un output corretto concorde alle attese.
\newline Formula:
\begin{displaymath}
\mbox{Output attesi}=\frac{\mbox{esecuzioni con output corretto}}{\mbox{numero totale di esecuzioni}}*100
\end{displaymath}
\
\
\newline Giudizio dei risultati ottenibili:
\begin{itemize}
\item Valore negativo: meno del 98\% delle esecuzioni producono l'output atteso.
\item Valore accettabile: qualora almeno il 98\% delle esecuzioni producano l'output atteso.
\item Valore ottimale: il 100\% delle esecuzioni producono l'output atteso. Non sarà comunque possibile garantire l'affidabilità completa del sistema in ogni circostanza, sarebbe altrimenti troppo oneroso in qualità di tempo e costi.
\end{itemize}

\mysubparagraph{Validazione web}
\label{par:web}
Tale metrica viene usufruita, all'interno del progetto, per migliorare la qualità nell'usabilità, dal momento che questa risulta un fattore molto soggettivo, l'efficienza e la portabilità di un prodotto.
Si avvale degli strumenti messi a disposizione dal \gl{W3C} per valutare l'accessibilità, il codice HTML prodotto, etc. segnalando la quantità di errori riscontrati.
\
\newline Giudizio dei risultati ottenibili:
\begin{itemize}
\item Valore negativo: un numero di errori maggiore (o uguale) di 10.
\item Valore accettabile: un numero di errori minore di 10.
\item Valore ottimale: 0 errori riscontrati durante la verifica attraverso gli strumenti forniti.
\end{itemize}

\mysubparagraph{Grado di efficienza}
\label{par:greff}
Questa metrica permette di valutare il grado di efficienza di un prodotto attraverso la misurazione di tempo impiegato nello svolgere una transizione eseguita da un utente su tale prodotto.
\
\newline Giudizio dei risultati ottenibili:
\begin{itemize}
\item Valore negativo: attesa maggiore di 5 secondi dall'inizio di una transizione.
\item Valore accettabile: attesa massima entro 5 secondi di un output (che sia corretto).
\item Valore ottimale: attesa massima di 1 secondo. È infatti difficile garantire un'attesa nulla poiché ci sono altri fattori, oltre al prodotto \gl{software}, che possono incidere su tale valore temporale (sistema operativo, risorse \gl{hardware}, etc.). 
\end{itemize}

\mysubparagraph{Anomalie insufficienti}
\label{par:anins}
Per valutare la manutenibilità si fa uso di tale metrica che permette di identificare il numero di anomalie che sono state risolte una prima volta ma che necessitano di una seconda correzione per essere archiviate.
\newline Formula:
\begin{displaymath}
\mbox{Anomalie insufficienti}=\frac{\mbox{numero di anomalie da ricorreggere}}{\mbox{numero totale di anomalie risolte}}*100
\end{displaymath}
\
\
\newline Giudizio dei risultati ottenibili:
\begin{itemize}
\item Valore negativo: un valore maggiore del 25\%.
\item Valore accettabile: un valore inferiore o uguale al 25\%.
\item Valore ottimale: un valore inferiore all'1\%.
\end{itemize}


\myparagraph{Per il codice}
Nella seguente sezione vengono proposte delle metriche che possano valutare al meglio la qualità del codice che verrà scritto nelle attività successive alla Progettazione architetturale.

\mysubparagraph{Linee di codice per metodo}
\label{par:statement}
Questa metrica valuta il numero di statement presenti in ogni metodo (escluse le righe di commento). Maggiore sarà tale numero allora maggiore risulterà la complessità di comprensione e di verifica del codice scritto per tale metodo. 
\\ Pertanto se un metodo deve svolgere diverse azioni complesse, per non appesantirlo eccessivamente è buona norma suddividerlo in un numero maggiore di metodi che possano ad ogni modo portare al medesimo risultato.
\
\newline Giudizio dei risultati ottenibili:
\begin{itemize}
\item Valore negativo: un numero maggiore di 60 statement.
\item Valore accettabile: un numero di statement pari o inferiore a 60. 
\item Valore ottimale: un numero di statement inferiore a 30.
\end{itemize}

\mysubparagraph{Parametri formali per metodo}
La complessità di un metodo è determinata, oltre che da altri elementi, anche dal numero di parametri che deve ricevere in ingresso per eseguire nel proprio corpo determinate operazioni. 
\\ Un metodo ottimale dovrebbe perciò permettere il minor numero possibile di parametri formali al momento della dichiarazione in modo tale che il metodo non debba gestirne troppi al suo interno.
\
\newline Giudizio dei risultati ottenibili:
\begin{itemize}
\item Valore negativo: un numero di parametri formali maggiore di 6. 
\item Valore accettabile: un numero di parametri pari o inferiore a 6.
\item Valore ottimale: un numero inferiore a 4 parametri formali.
\end{itemize}

\mysubparagraph{Campi dati per classe}
Questa metrica determina il numero di campi dati presenti all'interno di una classe del programma. \\ Se una classe presenterà molti campi dati da gestire sarà allora maggiore la sua complessità e la difficoltà di riuso del codice in altre classi, oltre che alla scarsa manutenibilità del codice che ne consegue qualora si debbano apportare delle modifiche.
\
\newline Giudizio dei risultati ottenibili:
\begin{itemize}
\item Valore negativo: un numero di campi dati maggiore di 10. 
\item Valore accettabile: un numero di campi dati pari o inferiore a 10.
\item Valore ottimale: un numero inferiore a 5 campi dati.
\end{itemize}

\mysubparagraph{Complessità ciclomatica}
La complessità ciclomatica è la metrica che misura il numero di cammini linearmente indipendenti attraverso il grafo di controllo di flusso del programma, determinandone pertanto la sua complessità.
\\Il grafo di controllo di flusso, su cui avviene la misurazione, è composto da nodi ed archi. I primi corrispondono a gruppi indivisibili di istruzioni, mentre i secondi connettono due nodi se il secondo gruppo di istruzioni può essere eseguito immediatamente dopo il primo gruppo.
\\La complessità ciclomatica può essere
applicata a singole procedure, a moduli, a metodi oppure a classi di un programma.
Attraverso il suo impiego sarà possibile definire inoltre un limite superiore al numero di test necessari per raggiungere una completa copertura della componente considerata, a seconda della complessità del suo flusso di controllo.
\
\newline Giudizio dei risultati ottenibili:
\begin{itemize}
\item Valore negativo: un valore maggiore di 10. 
\item Valore accettabile: una complessità ciclomatica pari o inferiore a 10.
\item Valore ottimale: un valore minore di 5.
\end{itemize}

\mysubparagraph{Copertura del codice}
La copertura del codice è la percentuale degli statement eseguiti correttamente durante lo
svolgimento dei test. Se il valore risultante è alto ciò sta ad indicare che il codice analizzato potrà presentare eventuali errori con probabilità molto bassa, andando quindi a garantire un maggiore livello di affidabilità del software.
\
\newline Giudizio dei risultati ottenibili:
\begin{itemize}
\item Valore negativo: un valore minore dell'80\%. 
\item Valore accettabile: un valore maggiore dell'80\%.
\item Valore ottimale: un valore maggiore del 95\%.
\end{itemize}

\mysubparagraph{Livello di annidamento}
Tale metrica indica nel codice la quantità di volte in cui le strutture di controllo sono innestate una dentro l’altra. Se si ottiene un valore elevato di annidamento il programma considerato presenterà certamente un flusso interno complesso che renderà pertanto difficoltosa la copertura del codice durante l'esecuzione dei test. 
\
\newline Giudizio dei risultati ottenibili:
\begin{itemize}
\item Valore negativo: un valore maggiore di 5.  
\item Valore accettabile: un valore pari o inferiore a 5.
\item Valore ottimale: un valore inferiore a 3.
\end{itemize}

\mysubparagraph{Grado di accoppiamento}
Il grado di accoppiamento misura il grado con cui ogni componente di un programma fa affidamento su ciascuna delle altre componenti, dipendendo da esse.
Un valore basso di questa metrica permetterà al programma di essere maggiormente manutenibile poiché in caso di modifiche all'interno di una sua componente, le altre componenti non necessiteranno di modifiche consequenziali.
\\ Il grado di accoppiamento viene valutato attraverso il calcolo di due distinti indici:
\begin{description}
\item{ Accoppiamento Afferente (Ca):} indica il numero di classi esterne ad un package che dipendono da classi interne ad esso. 
\\ Un alto valore di tale accoppiamento è indice di un alto grado di dipendenza del resto del software dal package. D'altro canto un valore troppo basso indicherà che il package, con le sue funzionalità interne, non viene sufficientemente sfruttato dal resto del software.
\item{Accoppiamento Efferente (Ce):} indica il numero di classi interne ad un package che dipendono da classi esterne ad esso. 
\\Ottenere un basso valore di questo accoppiamento indica che la maggior parte delle funzionalità fornite dal package sono indipendenti dal resto del sistema.
\end{description}
\
\newline Giudizio dei risultati ottenibili:
\begin{itemize}
\item Valore negativo: per entrambi gli indici un valore maggiore di 8.   
\item Valore accettabile: per entrambi gli indici un valore minore (o uguale) di 8.
\item Valore ottimale: per entrambi gli indici un valore minore di 5.
\end{itemize}

\mysubparagraph{Grado di instabilità}
Attraverso i valori degli indici Ca e Ce (accoppiamento afferente ed efferente rispettivamente) ottenuti dalla misurazione della metrica Grado di accoppiamento, è possibile definire il grado di instabilità di una componente del programma. Questa metrica indica la possibilità di effettuare modifiche a tale componente senza influenzarne altre all’interno del programma.
\newline Formula:
\begin{displaymath}
\mbox{Instabilità (I)}=\frac{Ce}{Ca+Ce}*100
\end{displaymath}
\
\
\newline Giudizio dei risultati ottenibili:
\begin{itemize}
\item Valore negativo: un valore maggiore del 90\%.
\item Valore accettabile: un valore minore del 90\%.
\item Valore ottimale: un valore minore del 70\%.
\end{itemize}


\newpage
\section{Specifica dei test}
In questa sezione vengono descritti i test che nelle successive attività verranno implementati in modo tale che, con il loro superamento, possano garantire al sistema software livelli di qualità ottimali, un corretto funzionamento e un riscontro positivo con le aspettative del cliente.
\newline L'intera sezione potrà subire un ulteriore incremento nelle attività successive con l'aggiunta di nuovi test o la modifica della descrizione di alcuni di essi. La loro implementazione avverrà peraltro nelle attività successive per cui all'attività attuale tutti i test saranno nello stato N.I. (non implementato).
\newline Per comprendere la sintassi con cui saranno definiti i test è consigliabile consultare la sezione apposita nel documento \doc{Norme di Progetto}.

\subsection{Test di Validazione}
I test di validazione vengono utilizzati durante l'attività di collaudo finale, per accertarsi che il prodotto realizzato sia conforme alle attese del committente.
\newline Per ogni test viene specificato il proprio codice univoco, il requisito (o i requisiti) a cui fa riferimento, lo stato di implementazione attuale e la descrizione che indica le operazioni che l'utente deve svolgere per testare i requisiti associati a tale test. 
\newline Il tracciamento tra i test di validazione e i requisiti correlati è riportato nel documento \doc{Analisi dei Requisiti}.



\begin{tabella}{!{\VRule}c!{\VRule}p{8cm}!{\VRule}c!{\VRule}c!{\VRule}}
	\color{white} \bold{Codice test} & \color{white} \bold{Descrizione} & \color{white} \bold{Requisito} & \color{white} \bold{Stato}\\
	\endfirsthead
	TV1 & 
		Un nuovo utente vuole verificare la possibilità di creare un proprio account.
		\newline \newline 
		Operazioni:
		{\begin{enumerate}
			\item entrare nella pagina di registrazione;
			\item inserire il nome;
			\item inserire il cognome;
			\item inserire l'email;
			\item inserire l’email di conferma;
			\item inserire la password;
			\item inserire la password di conferma;
			\item verificare la presenza di  un avviso qualora vi siano errori;
			\item in caso di assenza di errori verificare che la creazione dell’account sia avvenuta correttamente eseguendo un login.
		\end{enumerate}
		} 
	& ROF1 & N.I. 
	\\
	TV1.3 & 
		L'utente desidera verificare che l'inserimento della propria email non provochi alcun problema.
		\newline \newline
		Operazioni:
		{\begin{enumerate}
			\item inserire l'email nell'apposito campo;
			\item verificare che l’indirizzo email sia univoco all’interno del sistema;
			\item verificare che l’indirizzo email abbia un formato valido come descritto nell' nell’\ARdoc;
			\item verificare la segnalazione di errori attraverso degli avvisi qualora uno dei punti precedenti non sia stato eseguito in modo corretto.
		\end{enumerate}
		}
	& ROF1.3 & N.I.
	\\
	TV1.3.3 &
		L’utente deve verificare che l’email di conferma inserita sia uguale a quella precedentemente inserita nel campo sopra.
		\newline \newline
		Operazioni:
		{\begin{enumerate}
			\item inserire l’email per la seconda volta;
			\item verificare che l’email appena inserita sia uguale a quella inserita nel campo soprastante;
			\item verificare la presenza di un avviso qualora si verifichi un errore con la email appena inserita.
		\end{enumerate}
		}
	& ROF1.3.3 & N.I.
	\\
	TV1.4 &
		L’utente desidera verificare che l’inserimento della password non provochi problemi.
		 \newline \newline
		 Operazioni:
		 {\begin{enumerate}
		 	\item inserire la password nell’apposito campo;
		 	\item verificare che la password rispetti le caratteristiche descritte nell'\ARdoc;
		 	\item verificare la presenza di un avviso d’errore qualora ci sia stato un problema.
		\end{enumerate}
		 }
	& ROF1.4 & N.I.
	\\
	TV1.4.2 &
		L’utente deve verificare che la password di conferma inserita sia uguale a quella precedentemente digitata nel campo sopra.
		\newline \newline
		Operazioni:
		{\begin{enumerate}
			\item inserire la password per la seconda volta;
			\item verificare che la password appena inserita sia uguale a quella inserita nel campo soprastante;
			\item verificare la presenza di un avviso qualora si verifichi un errore con la password appena inserita.
		\end{enumerate}
		}
	& ROF1.4.2 & N.I.
	\\
	TV2 &
		L’utente deve verificare la possibilità di autenticarsi una volta registrato.
		\newline \newline
		Operazioni:
		{\begin{enumerate}
			\item inserire l’email;
			\item inserire la  password;
			\item cliccare sul pulsante di accesso;
			\item verificare la presenza di un avviso in caso ci sia stato un errore durante l’autenticazione;
			\item verificare la corretta autenticazione nel caso in cui i dati inseriti siano validi e non abbiano causato alcun errore.
		\end{enumerate}
		}
	& ROF2 & N.I.
	\\
	TV3 &
		L’utente vuole verificare la possibilità di recuperare la propria password utilizzando l’ email usata al momento della registrazione.	
		\newline \newline
		Operazioni:
		{\begin{enumerate}
			\item entrare nella pagina dedicata al recupero della password;
			\item inserire la propria email nell’appropriato campo;
			\item cliccare sull’apposito pulsante per eseguire l’operazione;
			\item verificare che, in caso di errore nel recupero della password, l’utente sia avvisato;
			\item verificare, qualora non sia avvenuto alcun errore, di aver ricevuto una email con una password temporanea al suo interno.
		\end{enumerate}
		}
	& ROF3 & N.I.
	\\
	TV11 &
		L’utente vuole verificare la possibilità di svolgere i quiz pubblici all’interno del sistema.
		\newline \newline
		Operazioni:
		{\begin{enumerate}
			\item entrare nella pagina d’archivio dei quiz pubblici;
			\item selezionare il quiz desiderato tramite le operazioni di ricerca descritte nell’\ARdoc;
			\item confermare la selezione del quiz desiderato cliccandoci sopra;			
			\item svolgere il quiz rispondendo alle domande secondo le modalità descritte nell’\ARdoc;
			\item confermare la risoluzione del quiz cliccando il pulsante apposito;
			\item visualizzare l’esito del quiz svolto.	
		\end{enumerate}
		}
	& ROF11 & N.I.
	\\
	TV12.1 &
		L’utente vuole verificare la possibilità di eseguire una ricerca tra i quiz e visualizzare i risultati ottenuti.
		\newline \newline
		Operazioni:
		{\begin{enumerate}
			\item entrare nella pagina dedicata alla ricerca;
			\item selezionare i parametri di ricerca (argomento, livello di difficoltà, parole chiave, autore ) per i quiz solamente pubblici;
			\item premere il pulsante 'Cerca';
			\item verificare la corretta visualizzazione dei risultati.
		\end{enumerate}
		}
	& ROF12.1 & N.I.
	\\	
	TV12.1.1.5 &
		Lo studente vuole verificare la possibilità di eseguire una ricerca tra i quiz solamente privati.
		\newline \newline
		Operazioni:
		{\begin{enumerate}
			\item entrare nella pagina dedicata alla ricerca;
			\item selezionare i parametri di ricerca (argomento, livello di difficoltà, parole chiave, autore) per i quiz;
			\item selezionare la voce 'Privati' per restringere il campo di ricerca ai quiz privati legati alla classe a cui lo studente appartiene;
			\item premere il pulsante 'Cerca';
			\item verificare la corretta visualizzazione dei risultati.
		\end{enumerate}
		}
	& ROF12.1.1.5 & N.I.
	\\
	TV12.2 &
		Il docente vuole verificare la possibilità di eseguire una ricerca tra le domande presenti nel sistema.
		\newline \newline
		Operazioni:
		{\begin{enumerate}
				\item il docente deve essere autenticato;
				\item entrare nella pagina dedicata alla ricerca;
				\item selezionare i parametri di ricerca (argomento, livello di difficoltà, parole chiave, autore) per le domande;
				\item premere il pulsante 'Cerca';
				\item verificare la corretta visualizzazione dei risultati.
		\end{enumerate}
		}
	& ROF12.2 & N.I.
	\\
	TV13 &
		L’utente può verificare l’uscita dal sistema eseguendo il logout.
		\newline \newline
		Operazioni:
		{\begin{enumerate}
				\item l’utente deve essere autenticato;
				\item premere il pulsante di logout;
				\item verificare di essere realmente fuori dall’area di autenticazione.
		\end{enumerate}
		}
	& ROF13 & N.I.
	\\
	TV14.1 &
		L’utente vuole verificare di poter visualizzare le proprie informazioni personali.
		\newline \newline
		Operazioni:
		{\begin{enumerate}
				\item l’utente si deve autenticare;
				\item entrare nella pagina dedicata al profilo;
				\item visualizzare i propri dati presenti nella pagina come descritto nell’\ARdoc.
		\end{enumerate}
		}
	& ROF14.1 & N.I.
	\\
	TV14.2 & 
		L’utente vuole verificare la possibilità di visualizzare lo storico dei propri quiz.
		\newline \newline
		Operazioni:
		{\begin{enumerate}
				\item l’utente deve essere autenticato;
				\item entrare nella pagina dedicata allo storico dei quiz;
				\item verificare la presenza corretta dei propri quiz svolti in precedenza.
		\end{enumerate}
		}
	& ROF14.2 & N.I.
	\\
	TV14.3 & 
		L’utente vuole verificare la disponibilità di modificare i propri dati personali.
		\newline \newline
		Operazioni:
		{\begin{enumerate}
				\item l’utente si deve autenticare;
				\item entrare nella pagina dedicata al profilo;
				\item selezionale il tasto 'Modifica' per modificare le proprie informazioni personali;
				\item confermare le proprie modifiche;
				\item verificare la corretta modifica del campo/i dati cambiato/i.				
		\end{enumerate}
		}
	& ROF14.3 & N.I.
	\\
	TV14.3.3 &
		L’utente vuole verificare la possibilità di modificare la propria password.
		\newline \newline
		Operazioni:
		{\begin{enumerate}
				\item L’utente si deve autenticare;				
				\item entrare nella pagina dedicata al profilo;
				\item selezionare il tasto 'Modifica';
				\item nello spazio dedicato alla modifica della password, inserire la password corrente o una password temporanea fornita dal sistema;
				\item inserire la nuova password;
				\item inserire nuovamente la password nuova;
				\item confermare la modifica della password;
				\item verificare che il cambio password sia avvenuto correttamente eseguendo un login;
				\item verificare la presenza di un avviso in caso sia avvenuto un errore durante il cambio della password.
		\end{enumerate}
		}
	& ROF14.3.3 & N.I
	\\
	TV14.4.1 &
		Il responsabile vuole verificare la possibilità di eliminare dal proprio ente un account da lui gestito.
		\newline \newline
		Operazioni:
		{\begin{enumerate}
				\item il responsabile deve essere autenticato;				
				\item entrare nella pagina dedicata alla gestione account;
				\item selezionare l’account da eliminare dall’ente, tra quelli gestiti da lui stesso;
				\item premere il pulsante 'Elimina';
				\item verificare che l’account sia stato eliminato dall’ente correttamente.
		\end{enumerate}
		}
	& ROF14.4.1 & N.I.
	\\
	TV15.1 &
		Il docente vuole verificare la possibilità di creare un quiz.
		\newline \newline
		Operazioni:
		{\begin{enumerate}
				\item il docente deve essere autenticato;
				\item entrare nella pagina dedicata ai quiz;
				\item premere il pulsante 'Crea quiz';
				\item inserire nel campo apposito il titolo del quiz;
				\item selezionare l’argomento del quiz;
				\item inserire la descrizione del quiz;
				\item specificare, se desiderato, delle parole chiave da associare al quiz;
				\item selezionare il permesso del quiz (pubblico o privato);
				\item nel caso sia stato selezionato il permesso 'privato', selezionare le classi a cui il quiz è destinato;
				\item aggiungere le domande desiderate;
				\item confermare il salvataggio del quiz;
				\item verificare che il docente sia attribuito automaticamente come autore del quiz in questione;
				\item verificare la corretta creazione del quiz.
		\end{enumerate}
		}
	& ROF15.1 & N.I.
	\\
	TV15.2 &
		Il docente vuole modificare un quiz precedentemente creato.
		\newline \newline
		Operazioni:
		{\begin{enumerate}
				\item il docente deve essere autenticato;
				\item entrare nella pagina dedicata ai quiz;
				\item selezionare il quiz che si desidera modificare;
				\item premere il pulsante 'Modifica';
				\item modificare il campo desiderato (titolo, argomento, descrizione, permesso, parole chiave);
				\item aggiungere le domande desiderate;
				\item rimuovere le domande indesiderate;
				\item confermare la modifica;
				\item verificare che la modifica sia stata attuata in modo corretto;
				\item verificare che venga presentato un avviso in caso si verifichi un errore durante la modifica del quiz.
		\end{enumerate}
		}
	& ROF15.2 & N.I.
	\\
	TV15.2.5 &
		Il docente vuole verificare la possibilità di inserire una domanda all’interno di un quiz.
		\newline \newline
		Operazioni:
		{\begin{enumerate}
				\item il docente deve essere autenticato;
				\item entrare nella pagina dedicata ai quiz;
				\item selezionare il quiz che si desidera modificare;				
				\item premere il pulsante 'Modifica';
				\item selezionare la voce 'Aggiungi domanda';
				\item selezionare una domanda già esistente oppure crearne una nuova;
				\item confermare l’aggiunta della domanda;
				\item confermare la modifica del quiz;
				\item verificare che la modifica sia stata attuata in modo corretto;
				\item verificare che venga presentato un avviso in caso di errore durante la modifica di un quiz.
		\end{enumerate}
		}
	& ROF15.2.5 & N.I.
	\\
	TV15.2.6 &
		Il docente vuole verificare la possibilità di eliminare una domanda all’interno di un quiz.
		\newline \newline
		Operazioni:
		{\begin{enumerate}
				\item il docente deve essere autenticato;				
				\item entrare nella pagina dedicata ai quiz;
				\item selezionare il quiz che si desidera modificare;
				\item premere il pulsante modifica;
				\item selezionare le domande che si vogliono eliminare dal quiz;
				\item selezionare la voce 'Elimina domanda';
				\item confermare la modifica del quiz;
				\item verificare che la modifica sia stata attuata in modo corretto;
				\item verificare che venga presentato un avviso in caso di errore durante la modifica del quiz.
				
		\end{enumerate}
		}
	& ROF15.2.6 & N.I.
	\\
	TV15.3 &
		Il docente vuole testare la possibilità di eliminare un quiz esistente.
		\newline \newline
		Operazioni:
		{\begin{enumerate}
				\item il docente deve essere autenticato;
				\item entrare nella pagina dedicata ai quiz;
				\item selezionare il quiz che si desidera eliminare;
				\item premere il pulsante 'Elimina';
				\item confermare l’eliminazione del quiz;
				\item verificare che l’eliminazione sia stata effettuata in modo corretto.
			\end{enumerate}
		}
	& ROF15.3 & N.I.
	\\
	TV16.1/2 &
		L’utente senza ruolo vuole verificare la possibilità di effettuare una richiesta per ricoprire il ruolo di docente o studente all’interno di un ente.
		\newline \newline
		Operazioni:
		{\begin{enumerate}
				\item l’utente senza ruolo deve essere autenticato;
				\item inviare la richiesta per il ruolo di studente o docente all’interno dell’ente desiderato;
				\item verificare che la richiesta sia stata accettata oppure negata tramite un avviso;
				\item verificare che avvenga un messaggio di errore se l’utente possiede già un ruolo all’interno di tale ente e cerca comunque di eseguire una richiesta per un nuovo ruolo per tale ente.
		\end{enumerate}
		}
	&  ROF16.1 - ROF16.2  & N.I.
	\\
	TV16.3 &
		Il docente vuole verificare di poter effettuare una richiesta di inserimento in una classe esistente.
		\newline \newline
		Operazioni:
		{\begin{enumerate}
				\item il docente deve essere autenticato;
				\item selezionare l’ente contenente la classe alla quale ci si vuole iscrivere;
				\item selezionare la classe a cui ci si vuole iscrivere;
				\item premere il pulsante 'Invia richiesta' per richiedere il proprio inserimento in tale classe;
				\item verificare l’accettazione o la negazione della richiesta inviata;
				\item verificare che venga segnalato un errore qualora un docente appartenente ad una classe richieda di essere inserito nella medesima classe.
		\end{enumerate}
		}
	& ROF16.3 & N.I.
	\\
	TV16.4 &
		Lo studente vuole verificare di poter effettuare una richiesta di inserimento in una classe esistente.
		\newline \newline
		Operazioni:
		{\begin{enumerate}
				\item lo studente deve essere autenticato;
				\item selezionare l’ente contenente la classe alla quale ci si vuole iscrivere;
				\item selezionare la classe in cui ci si vuole inserire;
				\item premere il pulsante 'Invia richiesta' per richiedere il proprio inserimento in tale classe;
				\item verificare l’accettazione o la negazione della richiesta inviata;
				\item verificare che venga segnalato un errore qualora uno studente appartenente ad una classe richieda di essere inserito nella medesima classe.
		\end{enumerate}
		}
	& ROF16.4 & N.I.
	\\
	TV17.1 & 
		Il responsabile vuole verificare la possibilità di gestire le richieste eseguite dai docenti per l’inserimento in una classe.
		\newline \newline
		Operazioni:
		{\begin{enumerate}
				\item il responsabile deve essere autenticato;
				\item entrare nella pagina contenente la lista delle richieste in sospeso;
				\item selezionare la richiesta ricevuta dal docente;
				\item accettare o negare la richiesta;
				\item se accettata, verificare che il docente sia stato inserito correttamente nella classe da lui richiesta.
		\end{enumerate}
		}
	& ROF17.1 & N.I.
	\\
	TV17.2 & 
		Il responsabile vuole verificare la possibilità di gestire le richieste di attribuzione del ruolo di docente ad un utente senza ruolo.
		\newline \newline
		Operazioni:
		{\begin{enumerate}
				\item il responsabile deve essere autenticato;
				\item entrare nella pagina contenente la lista delle richieste in sospeso;
				\item selezionare la richiesta di attribuzione del ruolo di docente;
				\item accettare o rifiutare la richiesta;
				\item se accettata, verificare che l’utente senza ruolo ora sia in possesso del ruolo di docente all'interno dell'ente gestito da tale responsabile.
		\end{enumerate}
		}
	& ROF17.2 & N.I.
	\\
	TV17.3 &
		Il docente vuole verificare la possibilità di gestire le richieste effettuate da uno studente per il suo inserimento.
		\newline \newline
		Operazioni:
		{\begin{enumerate}
				\item il docente deve essere autenticato;
				\item entrare nella pagina contenente la lista delle richieste in sospeso;
				\item selezionare la richiesta espressa dallo studente per l’inserimento in una classe;
				\item accettare o negare la richiesta;
				\item se accettata, verificare il corretto inserimento dello studente nella classe desiderata.
		\end{enumerate}
		}
	& ROF17.3 & N.I.
	\\
	TV17.4 &
		Il responsabile vuole verificare la possibilità di gestire le richieste di attribuzione del ruolo di studente ad un utente senza ruolo.
		\newline \newline
		Operazioni:
		{\begin{enumerate}
				\item il responsabile deve essere autenticato;
				\item entrare nella pagina contenente la lista delle richieste in sospeso;
				\item selezionare la richiesta di attribuzione del ruolo di studente;
				\item accettare o rifiutare la richiesta;
				\item se accettata, verificare che l’utente senza ruolo ora sia in possesso del ruolo di studente.
		\end{enumerate}
		}
	& ROF17.4 & N.I.
	\\
	TV18.1 &
		Il responsabile vuole verificare la possibilità di creare un nuovo argomento all’interno del sistema.
		\newline \newline
		Operazioni:
		{\begin{enumerate}
				\item il responsabile deve essere autenticato;
				\item entrare nella pagina dedicata agli argomenti;
				\item premere il pulsante 'Nuovo argomento';
				\item inserire il nome dell’argomento;
				\item verificare che l’argomento sia stato creato correttamente;
				\item verificare che venga segnalato un errore qualora il responsabile cerchi di creare un argomento con un nome già assegnato ad un altro argomento.
		\end{enumerate}
		}
	& ROF18.1 & N.I.
	\\
	TV18.2 &
		Il responsabile vuole verificare la disponibilità di eliminare dal sistema un argomento esistente.
		\newline \newline
		Operazioni:
		{\begin{enumerate}
				\item il responsabile deve essere autenticato;
				\item entrare nella pagina dedicata agli argomenti;
				\item selezionare l’argomento che si desidera eliminare;
				\item premere il pulsante elimina;
				\item confermare l’eliminazione dell’argomento;
				\item verificare che l’argomento sia stato eliminato correttamente.
		\end{enumerate}
		}
	& ROF18.2 & N.I.
	\\
	TV19 &
		Il responsabile verifica la possibilità di modificare il proprio ente.
		\newline \newline
		Operazioni:
		{\begin{enumerate}
				\item il responsabile deve essere autenticato;	
				\item entrare nella pagina dedicata al proprio ente;
				\item selezionare 'Modifica ente';
				\item modificare il nome del proprio ente;
				\item confermare la modifica;
				\item verificare la corretta modifica.
		\end{enumerate}
		}
	& ROF19 & N.I.
	\\				
	TV20.1 &
		Il docente vuole verificare la possibilità di visualizzare le statistiche relative ai quiz.
		\newline \newline
		Operazioni:
		{\begin{enumerate}
				\item il docente deve essere autenticato;
				\item entrare nella pagina dedicata ai quiz;
				\item selezionare la voce statistiche al suo interno;
				\item verificare la corretta visualizzazione delle statistiche come descritto nell’\ARdoc.
		\end{enumerate}
		}
	& ROF20.1 & N.I.
	\\
	TV20.2 &
		Il docente vuole verificare la possibilità di visualizzare le statistiche relative alle domande.
		\newline \newline
		Operazioni:
		{\begin{enumerate}
				\item il docente deve essere autenticato;
				\item entrare nella pagina dedicata alle domande;
				\item selezionare la voce statistiche al suo interno;
				\item verificare la corretta visualizzazione delle statistiche come descritto nell’\ARdoc.
		\end{enumerate}
		}
	& ROF20.2 & N.I.
	\\ 
	TV20.3 &
		Il docente vuole visualizzare le statistiche relative agli studenti.
		\newline \newline
		Operazioni:
		{\begin{enumerate}
				\item il docente deve essere autenticato; 
				\item entrare nella pagina dedicata agli studenti;
				\item eseguire una ricerca per trovare lo studente desiderato; 
				\item selezionare lo studente di cui si vuole vedere le statistiche;
				\item verificare la corretta visualizzazione delle statistiche come definito nell’\ARdoc.
		\end{enumerate}
		}
	& ROF20.3 & N.I.
	\\
	TV21.1 &
		Il responsabile vuole verificare la possibilità di creare una classe associata al proprio ente.
		\newline \newline
		Operazioni:
		{\begin{enumerate}
				\item il responsabile deve essere autenticato;
				\item entrare nella pagina dedicata alle classi del proprio ente;
				\item selezionare il pulsante 'Nuova classe' dal suo interno;
				\item inserire il nome della classe;
				\item inserire la descrizione della classe;
				\item inserire l’anno scolastico;
				\item confermare la creazione della classe;
				\item verificare la corretta creazione della classe all’interno del proprio ente.
				  
		\end{enumerate}
		}
	& ROF21.1 & N.I.
	\\
	TV21.2 &
		Il responsabile vuole verificare la possibilità di modificare una classe esistente all'interno del proprio ente.
		\newline \newline
		Operazioni:
		{\begin{enumerate}
				\item il responsabile deve essere autenticato;
				\item entrare nella pagina dedicata alle classi del proprio ente;
				\item selezionare la classe che si desidera modificare;
				\item premere il pulsante 'Modifica';
				\item eseguire la modifica della descrizione;
				\item confermare la modifica;
				\item verificare la corretta modifica della classe.
		\end{enumerate}
		}
	& ROF21.2 & N.I.
	\\
	TV21.3 &
		Il responsabile verifica la possibilità di eliminare una classe appartenente al proprio ente.
		\newline \newline
		Operazioni:
		{\begin{enumerate}
				\item il responsabile deve essere autenticato;
				\item entrare nella pagina dedicata alle classi del proprio ente;
				\item selezionare la classe che si desidera eliminare;
				\item premere il pulsante 'Elimina';
				\item confermare l’eliminazione;
				\item verificare la corretta rimozione della classe dal sistema.
		\end{enumerate}
		}
	& ROF21.3 & N.I.
	\\
	TV21.4 &
		Il docente deve verificare di poter visualizzare la lista di studenti associati ad una classe.
		\newline \newline
		Operazioni:
		{\begin{enumerate}
				\item il responsabile deve essere autenticato;
				\item entrare nella pagina dedicata alle classi del proprio ente;
				\item selezionare la classe di cui si desidera conoscere il contenuto;
				\item visualizzare la lista degli studenti di cui ne fanno parte.
		\end{enumerate}
		}
	& ROF21.4 & N.I.
	\\
	TV21.5 &
		Il responsabile deve verificare di poter visualizzare la lista di  docenti associati ad una classe.
		\newline \newline
		Operazioni:
		{\begin{enumerate}
				\item il responsabile deve essere autenticato;
				\item entrare nella pagina dedicata alle classi del proprio ente;
				\item selezionare la classe di cui si desidera conoscere il contenuto;
				\item visualizzare la lista  dei docenti di cui ne fanno parte.
		\end{enumerate}
		}
	& ROF21.5 & N.I.
	\\
	TV21.6 &
		Il responsabile vuole verificare la possibilità di gestire una classe esistente all'interno del proprio ente.
		\newline \newline
		Operazioni:
		{\begin{enumerate}
				\item il responsabile deve essere autenticato;
				\item entrare nella pagina dedicata alle classi del proprio ente;
				\item selezionare la classe che si desidera gestire;
				\item inserire/rimuovere uno studente;
				\item inserire/rimuovere un docente;
				\item verificare che l’operazione non abbia prodotto alcun errore.
		\end{enumerate}
		}
	& ROF21.6 & N.I.
	\\
	TV31.3 &
		Il docente vuole verificare la creazione di domande vero o falso.
		\newline \newline
		Operazioni:
		{\begin{enumerate}
				\item il docente deve essere autenticato;
				\item entrare nella pagina dedicata alle domande;
				\item premere il pulsante 'Nuova domanda vero o falso';
				\item inserire il titolo della domanda;
				\item inserire una descrizione;
				\item selezionare l’argomento della domanda;
				\item selezionare il livello di difficoltà;
				\item inserire le parole chiave;
				\item inserire facoltativamente un allegato;
				\item selezionare la risposta vera;
				\item confermare la creazione della domanda;
				\item verificare la corretta creazione della domanda.
		\end{enumerate}
		}
	& ROF31.3 & N.I.
	\\
	TV31.4 & 
		Il docente vuole verificare la possibilità di creare una domanda a risposta aperta.
		\newline \newline
		Operazioni:
		{\begin{enumerate}
				\item il docente deve essere autenticato;
				\item entrare nella pagina dedicata alle domande;
				\item premere il pulsante 'Nuova domanda a risposta aperta';
				\item inserire il titolo della domanda;
				\item inserire una descrizione;
				\item selezionare l’argomento della domanda;
				\item selezionare il livello di difficoltà;
				\item inserire le parole chiave;
				\item inserire facoltativamente un allegato;
				\item inserire la risposta corretta;
				\item confermare la creazione della domanda;
				\item verificare la corretta creazione della domanda.
		\end{enumerate}
		}
	& ROF31.4 & N.I.
	\\
	TV31.5 &
		Il docente vuole verificare la possibilità di creare una domanda a risposta multipla.
		\newline \newline
		Operazioni:
		{\begin{enumerate}
				\item il docente deve essere autenticato;				  
				\item entrare nella pagina dedicata alle domande;
				\item premere il pulsante 'Nuova domanda a risposta multipla';
				\item inserire il titolo della domanda;
				\item inserire una descrizione;
				\item selezionare l’argomento della domanda;
				\item selezionare il livello di difficoltà;
				\item inserire le parole chiave;
				\item inserire facoltativamente un allegato;
				\item inserire diverse risposte (di tipo multimediale o testuale);
				\item selezionare le risposte corrette;
				\item confermare la creazione della domanda;
				\item verificare la corretta creazione della domanda.
		\end{enumerate}
		}
	& ROF31.5 & N.I.
	\\
	TV31.6 &
		Il docente vuole verificare la possibilità di creare una domanda a completamento testuale.
		\newline \newline
		Operazioni:
		{\begin{enumerate}
				\item il docente deve essere autenticato;
				\item entrare nella pagina dedicata alle domande;
				\item premere il pulsante 'Nuova domanda a completamento testo';
				\item inserire il titolo della domanda;
				\item inserire una descrizione;
				\item selezionare l’argomento della domanda;
				\item selezionare il livello di difficoltà;
				\item inserire le parole chiave;
				\item inserire facoltativamente un allegato;
				\item inserire il testo incompleto della domanda;
				\item specificare un insieme di parole corrette e sbagliate;
				\item confermare la creazione della domanda;
				\item verificare la corretta creazione della domanda.
				
				
		\end{enumerate}
		}
	& ROF31.6 & N.I
	\\
	TV31.7 &
		Il docente vuole verificare la creazione di domande a collegamento.
		\newline \newline
		Operazioni:
		{\begin{enumerate}
				\item il docente deve essere autenticato;
				\item entrare nella pagina dedicata alle domande;
				\item premere il pulsante 'Nuova domanda a collegamenti';
				\item inserire il titolo della domanda;
				\item inserire una descrizione;
				\item selezionare l’argomento della domanda;
				\item selezionare il livello di difficoltà;
				\item inserire le parole chiave;
				\item inserire facoltativamente un allegato;
				\item inserire la risposta sotto forma di ennupla secondo le modalità stabilite nell'\ARdoc;
				\item confermare la creazione della domanda;
				\item verificare la corretta creazione della domanda.
		\end{enumerate}
		}
	& ROF31.7 & N.I.
	\\
	TV31.7.1 &
		Il docente vuole verificare la possibilità di creare una ennupla per le domande a collegamenti.
		\newline \newline
		Operazioni:
		{\begin{enumerate}
				\item inserire nella parte iniziale dell’ennupla un file multimediale oppure del testo;
				\item inserire nella parte finale dell’ennupla un file multimediale oppure del testo.
		\end{enumerate}
		}
	& ROF31.7.1 & N.I.
	\\
	TV31.2 &
		Il docente vuole verificare la possibilità di interrompere la creazione di una domanda uscendo dal sistema.
		\newline \newline
		Operazioni:
		{\begin{enumerate}
				\item il docente deve essere autenticato;
				\item entrare nella pagina dedicata alle domande;
				\item a seconda della tipologia di domanda che si desidera creare premere il corrispettivo pulsante 'Nuova domanda';				
				\item durante la creazione della domanda uscire cliccando qualsiasi voce presente nella pagina;
				\item confermare il desiderio di uscita dalla creazione della domanda premendo il pulsante relativo sull’avviso mostrato dal sistema per chiedere conferma dell’interruzione;
				\item verificare di essere fuori dalla creazione della domanda e verificare che la domanda non sia mai stata creata.	
		\end{enumerate}
		}
	& ROF31.2 & N.I.
	\\
	TV 31.10 &
		Il docente vuole verificare la possibilità di modificare una domanda di tipo vero o falso esistente.	
		\newline \newline
		Operazioni:
		{\begin{enumerate}
				\item il docente deve essere autenticato;
				\item entrare nella pagina dedicata alle domande;
				\item selezionare la domanda che si desidera modificare tra quelle di tipologia vero o falso;
				\item premere il pulsante 'Modifica';
				\item eseguire una modifica alle caratteristiche comuni (titolo, descrizione, argomento, livello di difficoltà, allegato, parole chiave); 
				\item eseguire una modifica alla veridicità delle risposta;
				\item confermare la modifica;
				\item verificare la corretta attuazione della modifica.
		\end{enumerate}
		}
	& ROF31.10 & N.I.
	\\
	TV31.11 &
		Il docente vuole verificare la possibilità di modificare una domanda a risposta multipla esistente.
		\newline \newline
		Operazioni:
		{\begin{enumerate}
				\item il docente deve essere autenticato;
				\item entrare nella pagina dedicata alle domande;
				\item selezionare la domanda che si desidera modificare tra quelle a risposta multipla;
				\item premere il pulsante 'Modifica';
				\item eseguire una modifica alle caratteristiche comuni (titolo, descrizione, argomento, livello di difficoltà, allegato, parole chiave);
				\item inserire una nuova risposta (testuale o multimediale) e definire la propria veridicità;
				\item modificare una risposta già esistente secondo le modalità descritte nell’\ARdoc;
				\item eliminare una risposta già esistente;
				\item confermare la modifica;
				\item verificare la corretta attuazione della modifica.
		\end{enumerate}
		}
	& ROF31.11 & N.I.
	\\
	TV31.12 &
		Il docente vuole verificare la possibilità di modificare una domanda a risposta aperta esistente.
		\newline \newline
		Operazioni:
		{\begin{enumerate}
				\item il docente deve essere autenticato;
				\item entrare nella pagina dedicata alle domande;
				\item selezionare la domanda che si desidera modificare tra quelle a risposta aperta;
				\item premere il pulsante 'Modifica';
				\item eseguire una modifica alle caratteristiche comuni (titolo, descrizione, argomento, livello di difficoltà, allegato, parole chiave);
				\item modificare il testo della risposta esistente;
				\item confermare la modifica;
				\item verificare la corretta attuazione della modifica.
		\end{enumerate}
		}
	& ROF31.12 & N.I.
	\\
	TV31.13 & 
		Il docente vuole verificare la possibilità di modificare una domanda a completamento testo esistente.
		\newline \newline
		Operazioni:
		{\begin{enumerate}
				\item il docente deve essere autenticato;
				\item entrare nella pagina dedicata alle domande;
				\item selezionare la domanda che si desidera modificare tra quelle a completamento testo;
				\item premere il pulsante 'Modifica';
				\item eseguire una modifica alle caratteristiche comuni (titolo, descrizione, argomento, livello di difficoltà, allegato, parole chiave);
				\item modificare il testo incompleto della domanda;
				\item modificare l’insieme delle risposte possibili;
				\item eliminare alcune parole tra le risposte possibili;
				\item confermare la modifica;
				\item verificare la corretta attuazione della modifica.
		\end{enumerate}
		}
	& ROF31.13 & N.I.
	\\
	TV31.14 &
		Il docente vuole verificare la possibilità di modificare una domanda a collegamenti esistente.
		\newline \newline
		Operazioni:
		{\begin{enumerate}
				\item il docente deve essere autenticato;
				\item entrare nella pagina dedicata alle domande;
				\item selezionare la domanda che si desidera modificare tra quelle a collegamenti;
				\item premere il pulsante 'Modifica';
				\item eseguire una modifica alle caratteristiche comuni (titolo, descrizione, argomento, livello di difficoltà, allegato, parole chiave);
				\item aggiungere una risposta secondo le modalità descritte nell'\ARdoc;
				\item eliminare una risposta;
				\item confermare la modifica;
				\item verificare la corretta attuazione della modifica.				
		\end{enumerate}
		}
	& ROF31.14 & N.I.
	\\
	TV31.9 &
		Il docente vuole verificare la possibilità di interrompere la modifica di una domanda uscendo dal sistema.
		\newline \newline
		Operazioni:
		{\begin{enumerate}
				\item il docente deve essere autenticato;
				\item entrare nella pagina dedicata alle domande;
				\item selezionare la domanda che si vuole modificare;
				\item premere il pulsante 'Modifica';
				\item durante la modifica della domanda uscire cliccando qualsiasi voce presente nella pagina;
				\item confermare il desiderio di uscita dalla modifica della domanda premendo il pulsante relativo sull’avviso mostrato dal sistema per chiedere conferma dell’interruzione;
				\item verificare di essere fuori dalla modifica della domanda e verificare che la domanda non abbia subito alcuna modifica.
		\end{enumerate}
		}
	& ROF31.9 & N.I.
	\\
	TV31.15 &
		Il docente vuole verificare la possibilità di eliminare una domanda dal sistema.
		\newline \newline
		Operazioni:
		{\begin{enumerate}
				\item il docente deve essere autenticato;
				\item entrare nella pagina dedicata alle domande;
				\item selezionare la domanda che si desidera eliminare;
				\item premere il pulsante 'Elimina';
				\item confermare l’eliminazione;
				\item verificare la corretta eliminazione della domanda dal sistema.
		\end{enumerate}
		}
	& ROF31.15 & N.I.
	\\
	TV20.4 &
		Il docente vuole verificare di poter visualizzare le statistiche relative ai docenti.
		\newline \newline
		Operazioni:
		{\begin{enumerate}
				\item il docente deve essere autenticato;
				\item entrare nella pagina dedicata ai docenti;
				\item selezionare la voce 'Statistiche';
				\item visualizzare correttamente la lista dei quiz creati da un docente  e la lista delle domande create da ogni docente (docenti che appartengono allo stesso ente di colui che cerca le statistiche).
		\end{enumerate}
		}
	& RDF20.4 & N.I.
	\\
	TV23 &
		Lo studente vuole verificare la possibilità di commentare i quiz e le domande.
		\newline \newline
		Operazioni:
		{\begin{enumerate}
				\item lo studente deve essere autenticato;
				\item entrare nella pagina dedicata al forum;
				\item selezionare la voce 'Domande' oppure 'Quiz';
				\item lasciare un proprio commento relativo ad una domanda o a un quiz;
				\item verificare che il commento sia visibile nel forum.
		\end{enumerate}
		}
	& RZF23 & N.I.
	\\
	TV8 &
		L’utente, che sia studente, docente oppure responsabile, deve verificare che il sistema funzioni correttamente su una piattaforma desktop.
		\newline \newline
		Operazioni:
		{\begin{enumerate}
				\item l’utente deve disporre di una connessione internet;
				\item entrare nel sito di Quizzipedia da un computer desktop;
				\item verificarne il suo corretto funzionamento.
		\end{enumerate}
		}
	& ROV8 & N.I.
	\\
	TV9 &
		Lo studente vuole verificare la possibilità di utilizzare le funzionalità del sistema da dispositivi mobile.
		\newline \newline
		Operazioni:
		{\begin{enumerate}
				\item l’utente deve disporre di una connessione ad internet sul proprio dispositivo mobile;				 
				\item entrare nel sito di Quizzipedia da un browser del dispositivo;
				\item verificarne il corretto funzionamento.
		\end{enumerate}
		}
	& ROV9 & N.I.
	\\
	TV10 &
		L’utente vuole verificare il corretto funzionamento del sistema con l’uso di JavaScript e cookies.
		\newline \newline
		Operazioni:
		{\begin{enumerate}
				\item l’utente deve disporre di una connessione ad internet;
				\item l’utente deve aprire il browser con cui ha intenzione di accedere a Quizzipedia;
				\item verificare che nelle impostazioni  del browser Javascript e cookies siano abilitati;
				\item navigare nel sito di Quizzipedia; 
		\end{enumerate}
		}
	& ROV10 & N.I.
	\\
	\rowcolor{white}  
	\caption{Test di validazione}
\end{tabella}


\newpage
\subsection{Test di Sistema}
I seguenti test di sistema servono per verificare il comportamento dinamico complessivo dell'intero sistema in riferimento ai requisiti dichiarati nel documento \doc{Analisi dei Requisiti}.
\newline Per ogni test viene specificato il proprio codice univoco, il requisito (o i requisiti) a cui fa riferimento, lo stato di implementazione attuale e la sua descrizione.
\newline Il tracciamento tra i test di sistema e i requisiti correlati è riportato nel documento \doc{Analisi dei Requisiti}.


\begin{tabella}{!{\VRule}c!{\VRule}p{8cm}!{\VRule}c!{\VRule}c!{\VRule}}
	\color{white} \bold{Codice test} & \color{white} \bold{Descrizione} & \color{white} \bold{Requisito} & \color{white} \bold{Stato}\\
	\endfirsthead
	TS1 & Viene verificato che il sistema conceda la
	creazione di un account permettendo di
	usufruire delle funzionalità del sistema. & ROF1 & N.I.
	\\
	TS2 & Viene verificato che il sistema permetta ad ogni utente correttamente registrato di autenticarsi. & ROF2 & N.I.
	\\
	TS3 & Viene verificato che il sistema permetta all’utente di recuperare la propria password, attraverso l’invio di una email contenente una password temporanea. & ROF3 & N.I.
	\\
	TS11 & Viene verificato che il sistema permetta a qualsiasi utente di svolgere i quiz pubblici all’interno del sistema. & ROF11 & N.I.
	\\
	TS12 & Viene verificato che il sistema permetta all’utente di eseguire determinate funzioni di ricerca su quiz e domande. & ROF12 & N.I.
	\\
	TS13 & Viene verificato che il sistema garantisca all’utente di uscire dalla navigazione. & ROF13 & N.I.
	\\
	TS14 & Viene verificato che il sistema garantisca all’utente delle funzionalità per la gestione dell’account. & ROF14 & N.I.
	\\
	TS15 & Viene verificato che il sistema permetta all’utente con il ruolo di docente di gestire i quiz. & ROF15 & N.I.
	\\
	TS16 & Viene verificato che il sistema permetta all’utente di eseguire delle richieste di ruolo al responsabile di un ente all’interno del sistema. & ROF16 & N.I.
	\\
	TS17 & Viene verificato che sistema permetta ai responsabili e ai docenti di gestire le richieste. & ROF17 & N.I.
	\\
	TS18 & Viene verificato che il sistema garantisca al responsabile di poter gestire gli argomenti presenti nel sistema. & ROF18 & N.I.
	\\
	TS19 & Viene verificato che il sistema permetta al responsabile di modificare il proprio ente all’interno del sistema. & ROF19 & N.I.
	\\
	TS20 & Viene verificato che il sistema permetta la visualizzazione delle statistiche relative ai quiz, alle domande, agli studenti. & ROF20 & N.I.
	\\
	TS21 & Viene verificato che il sistema permetta al responsabile di gestire le classi creandole, modificandole oppure eliminandole. & ROF21 & N.I.
	\\
	TS31 & Viene verificato che il sistema permetta la gestione delle domande da parte del docente. & ROF31 & N.I.
	\\
	TS31.2 & Viene verificato che il sistema permetta l’uscita volontaria del docente dalla creazione di una domanda. & ROF31.2 & N.I.
	\\
	TS31.9 & Viene verificato che il sistema permetta al docente l’uscita volontaria dalla modifica in corso di una domanda. & ROF31.9 & N.I.
	\\
	TS22 & Viene verificato che il sistema permetta la creazione dei quiz in modo dinamico attraverso la valutazione delle risposte date. & RZF22 & N.I.
	\\
	TS23 & Viene verificato che il sistema permetta l’uso di un forum legato ai quiz e alle domande. & RZF23 & N.I.
	\\
	TS24 & Viene verificato che il sistema permetta lo svolgimento dei quiz in modalità lezione. & RZF24 & N.I.
	\\
	TS25 & Viene verificato che il sistema conceda ad uno studenti di proporre nuove domande da inserire nei questionari. & RZF25 & N.I.
	\\
	TS7 & Viene verificato che il sistema conceda ad uno studente di proporre nuove domande da inserire nei questionari. & ROV7 & N.I.
	\\
	TS8/9 & Viene verificato che il sistema funzioni correttamente sia su dispositivi mobile sia su piattaforme desktop. & ROV8 - ROV9 & N.I.
	\\
	TS10 & Viene verificato che il sistema funzioni correttamente con JavaScript e  cookies abilitati. & ROV10 & N.I.
	\\
	TS27 & Viene verificato che il sistema funzioni correttamente sul browser Google Chrome versione 49 e future. & ROV27 & N.I.
	\\
	TS28 & Viene verificato che il sistema funzioni correttamente sul browser Mozilla Firefox versione 44. & ROV28 & N.I.
	\\
	TS29 & Viene verificato che il sistema funzioni correttamente sul browser Internet Explorer versione 11. & ROV29 & N.I.
	\\
	TS30 & Viene verificato che il sistema funzioni correttamente sul browser Safari versione 9. & ROV30 & N.I.
	\\
	\rowcolor{white}  
	\caption{Test di sistema}
\end{tabella}

\newpage
\subsection{Test di Integrazione}
I test di integrazione, che vengono di seguito riportati, servono per verificare la corretta comunicazione tra le diverse componenti del sistema software in modo tale che il funzionamento sia quello atteso.
\newline Tali test sono stati decisi seguendo una strategia bottom-up, basata sulla logica di realizzare il prodotto conclusivo a partire dalle singole componenti necessarie per realizzare  prima di tutto le funzionalità obbligatorie che sono state richieste.
\newline Per ogni test viene specificato il proprio codice univoco, la componente o le componenti a cui fa riferimento, lo stato di implementazione attuale e la sua descrizione. 
\newline Il tracciamento tra i test di integrazione e le componenti ad essi associate è riportato nel documento \doc{Specifica Tecnica}.

\begin{tabella}{!{\VRule}l!{\VRule}p{3cm}!{\VRule}p{6cm}!{\VRule}c!{\VRule}}
	
	%{!{\VRule}p{7cm}!{\VRule}p{7cm}!{\VRule}!}
	\color{white} \bold{Codice test} & \color{white} \bold{Descrizione} & \color{white} \bold{Componente} & \color{white} \bold{Stato}\\
	\endfirsthead
	TIQuizzipedia & Viene verificato che le comunicazioni
	tra client, server e database funzionino a dovere e
	i dati siano passati in modo corretto e sicuro. & Quizzipedia & N.I.  
	\\
	TIClient & Viene verificato che il sistema gestisca correttamente il
	front end del prodotto tra le tre componenti Model, Controller e View. & Quizzipedia::Client & N.I.
	\\
	TIClient & Viene verificato che il sistema gestisca correttamente la componente Model relativa al design pattern MVC. Più precisamente, viene verificato che la memorizzazione della struttura dei dati avvenga correttamente. & Quizzipedia::Client::Client & N.I.
	\\
	TIOrganization & Viene verificato che il sistema gestisca correttamente la memorizzazione della struttura dei dati associati agli enti e alle classi. & Quizzipedia::Client::Client::\-Organization & N.I.
	\\
	TIRequests & Viene verificato che il sistema gestisca a dovere la memorizzazione della struttura delle diverse richieste in arrivo dal controller. & Quizzipedia::Client::Client::\-Requests & N.I.
	\\
	TIServicesClient & Viene verificato che il sistema gestisca in modo corretto la memorizzazione della struttura dei dati per i diversi servizi offerti dal prodotto (domande, quiz, argomenti). & Quizzipedia::Client::Client::\-Services & N.I.
	\\
	TIQuestionsClient & Viene verificato che il sistema gestisca in modo corretto la memorizzazione della struttura dei dati relativi alle diverse tipologie di domande che possono comparire all’interno del sistema. & Quizzipedia::Client::Client::\-Services::Questions & N.I.
	\\
	TIStatistics & Viene verificato che il sistema gestisca in modo corretto la memorizzazione della struttura dei dati relativi alle statistiche. & Quizzipedia::Client::Client::\-Statistics & N.I.
	\\
	TIUsers & Viene verificato che il sistema gestisca correttamente la
	memorizzazione della struttura dei dati associati ai diversi utenti. & Quizzipedia::Client::Client::\-Users & N.I.
	\\
	TIViewClient & Viene verificato che il sistema gestisca correttamente la
	componente View relativa al pattern MVC. Più precisamente, viene verificato che venga gestita correttamente la rappresentazione del prodotto al cliente. & Quizzipedia::Client::ViewClient & N.I.
	\\
	TIViewOrgManager & Viene verificato che il sistema gestisca in modo corretto le pagine dedicate alla gestione delle classi e degli enti. & Quizzipedia::Client::ViewClient::\-ViewOrgManager & N.I.
	\\
	TIViewQuestionManager & Viene verificato che il sistema gestisca correttamente le pagine dedicate alla gestione delle domande. & Quizzipedia::Client::ViewClient::\-ViewQuestionManager & N.I.
	\\
	TIViewQuizManager & Viene verificato che il sistema gestisca correttamente le pagine dedicate alla gestione dei quiz. & Quizzipedia::Client::ViewClient::\-ViewQuizManager & N.I.
	\\
	TIViewQuizSolver & Viene verificato che il sistema gestisca correttamente le pagine dedicate allo svolgimento dei quiz e alla visualizzazione dei loro risultati. & Quizzipedia::Client::ViewClient::\-ViewQuizSolver & N.I.
	\\
	TIViewQuestionSolver & Viene verificato che il sistema gestisca in modo appropriato le pagine dedicate allo svolgimento delle singole domande, a seconda della loro tipologia. & Quizzipedia::Client::ViewClient::\-ViewQuizSolver::ViewQuestionSolver & N.I.
	\\
	TIViewRequest & Viene verificato che il sistema gestisca correttamente le pagine rappresentanti la gestione delle richieste di ruolo e di classe. & Quizzipedia::Client::ViewClient::\-ViewRequests & N.I.
	\\
	TIViewSearch & Viene verificato che il sistema gestisca in modo corretto le pagine per l’esecuzione della ricerca di classi, quiz e domande. & Quizzipedia::Client::ViewClient::\-ViewSearch & N.I.
	\\
	TIViewStatistics & Viene verificata la corretta gestione da parte del sistema delle pagine per la visualizzazione delle statistiche. & Quizzipedia::Client::ViewClient::\-ViewStatistics & N.I.
	\\
	TIViewTopicManager & Viene verificato che il sistema gestisca adeguatamente le pagine dedicate alla gestione degli argomenti. & Quizzipedia::Client::ViewClient::\-ViewTopicManager & N.I.
	\\
	TIViewUsers & Viene verificato che il sistema gestisca in modo corretto le pagine riguardanti le informazioni di ogni utente e le loro le funzioni principali. & Quizzipedia::Client::ViewClient::\-ViewUsers & N.I.
	\\
	TIViewErrors & Viene verificato che il sistema gestisca correttamente la visualizzazione degli errori ricevuti. & Quizzipedia::Client::ViewClient::\-ViewErrors & N.I.
	\\
	TIControllerClient & Viene verificato che il sistema gestisca correttamente la componente Controller relativa al design pattern MVC . Più precisamente, viene verificato che il
	sistema gestisca le azioni dell’utente, collegando in modo adeguato View e Model. & Quizzipedia::Client::ControllerClient & N.I.
	\\
	TICtrlOrganization & Viene verificato che il sistema gestisca ottimamente la comunicazione tra la componente della View relativa alla creazione delle classi e degli enti e la componente Model a cui vengono inviate le informazioni da immagazzinare nel database.  & Quizzipedia::Client::ControllerClient::\-CtrlOrganization & N.I.
	\\
	TICtrlRequests & Viene verificato che il sistema gestisca correttamente la comunicazione tra View e Model per la gestione delle richieste eseguite dall’utente. & Quizzipedia::Client::ControllerClient::\-CtrlRequests & N.I.
	\\
	TICtrlServices & Viene verificato che il sistema gestisca correttamente la comunicazione tra View e Model del client per la creazione, lo svolgimento e la ricerca di quiz e domande. Deve funzionare correttamente anche la comunicazione con il database attraverso il server per il salvataggio/caricamento delle informazioni. & Quizzipedia::Client::ControllerClient::\-CtrlServices & N.I.
	\\
	TICtrlStatistics & Viene verificato che sia gestita correttamente dal sistema la comunicazione tra Model e View per la gestione delle statistiche richieste dall’utente. & Quizzipedia::Client::ControllerClient::\-CtrlStatistics & N.I.
	\\
	TICtrlUsers & Viene verificato che sia gestita correttamente dal sistema la comunicazione tra la View, il Model e il server per quanto riguarda le informazioni dell’utente e la loro gestione. & Quizzipedia::Client::ControllerClient::\-CtrlUsers & N.I.
	\\
	TIServer & Viene verificato che il sistema
	gestisca correttamente l’interazione tra le componenti del back-end. & Quizzipedia::Server & N.I.
	\\
	TIControllerServer & Viene verificato che il sistema gestisca in modo corretto le richieste eseguite lato client. & Quizzipedia::Server::ControllerServer & N.I.
	\\
	TIAuthenticationManager & Viene verificato che il sistema gestisca correttamente i servizi di autenticazioni richiesti dal client. & Quizzipedia::Server::ControllerServer::\-AuthenticationManager & N.I.
	\\
	TIClassManager & Viene verificato che il sistema gestisca correttamente le richieste di gestione delle classi provenienti dal client. & Quizzipedia::Server::ControllerServer::\-ClassManager & N.I.
	\\
	TICompanyManager & Viene verificato che il sistema gestisca appropriatamente le richieste di gestione degli enti ricevute dal client. & Quizzipedia::Server::ControllerServer::\-CompanyManager & N.I.
	\\
	TIProfileManager & Viene verificato che il sistema gestisca appropriatamente le richieste del client per la gestione delle informazioni riguardanti i profili degli utenti autenticati. & Quizzipedia::Server::ControllerServer::\-ProfileManager & N.I.
	\\
	TIQuestionsManager & Viene verificato che il sistema gestisca in modo corretto la gestione delle domande secondo le richieste ricevute dal client. & Quizzipedia::Server::ControllerServer::\-QuestionsManager & N.I.
	\\
	TIQuizManager & Viene verificato che il sistema gestisca in modo corretto la gestione dei quiz secondo le richieste ricevute dal client. & Quizzipedia::Server::ControllerServer::\-QuizManager & N.I.
	\\
	TIQMLAgent & gestisca in modo corretto la traduzione delle informazioni inviate dal client, da linguaggio QML ad un formato corretto per il sistema. Deve essere verificato che avvenga anche il contrario. & Quizzipedia::Server::ControllerServer::\-QuizManager::QMLAgent & N.I.
	\\
	TIRequestsManager & Viene verificato che il sistema gestisca in modo appropriato le richieste degli utenti (di ruolo o di entrata in una classe) inviate dal client. & Quizzipedia::Server::ControllerServer::\-RequestsManager & N.I.
	\\
	TISearchManager & Viene verificato che il sistema gestisca correttamente le informazioni inviate dal client qualora un utente abbia eseguito una ricerca lato client. Deve essere verificata anche la comunicazione della risposta al client. &  Quizzipedia::Server::ControllerServer::\-SearchManager & N.I.
	\\
	TIStatisticsManager & Viene verificato che il sistema gestisca correttamente le richieste delle statistiche ricevute dal client, riguardanti quiz, domande e utenti. &  Quizzipedia::Server::ControllerServer::\-StatisticsManager & N.I.
	\\
	TITopicManager & Viene verificato che il sistema gestisca correttamente le richieste di creazione e di eliminazione di un argomento da parte del client. & Quizzipedia::Server::ControllerServer::\-TopicManager & N.I.
	\\
	TIServer & Viene verificato che il sistema gestisca correttamente lato server tutti i dati ottenuti dal client.  &  Quizzipedia::Server::Server & N.I.
	\\
	TIServicesServer & Viene verificato che il sistema gestisca correttamente lato server tutti i dati riguardanti le domande e i quiz. &  Quizzipedia::Server::Server::\-Services & N.I.
	\\
	TIQuestionsServer & Viene verificato che il sistema gestisca correttamente lato server i dati relativi alle diverse tipologie di domande presenti nel sistema. & Quizzipedia::Server::Server::\-Services::Questions & N.I.
	\\
	TIRoutingManager & Viene verificato che il sistema gestisca in modo adeguato la comunicazione con il client attraverso le API e il socket per indirizzare correttamente le richieste dell’utente ai corretti services. & Quizzipedia::Server::RoutingManager & N.I.	
	\\
	\rowcolor{white}  
	\caption{Test di integrazione}
\end{tabella}


\subsection{Test di Unità}
I test di unità servono per verificare il corretto funzionamento delle singole unità che compongono il sistema, ovvero le classi. 
\newline Tali test saranno definiti nelle attività successive quando l'architettura del sistema sarà definita in maggior dettaglio.

\newpage
\section{Attività di Analisi requisiti utente \\\large{resoconto~delle~sottoattività~di~verifica}}
\label{app:valtest}

\subsection{Verifica dei processi}
\subsubsection{Documentazione}
\myparagraph{Schedule Variance}
Vengono riportati di seguito i valori ottenuti calcolando la Schedule Variance sui tempi di stesura di ogni documento:
\begin{tabella}{l!{\VRule}c!{\VRule}c!{\VRule}}
	\color{white} \bold{Documento} & \color{white} \bold{SV in giorni} &\color{white} \bold{Giudizio} \\
	\endfirsthead
	Analisi dei Requisiti v1.0 & +2 & Accettabile \\
	Norme di Progetto v1.0 & 0 & Ottimale \\
    Studio di Fattibilità v1.0 &  -1 &  Ottimale \\
    Piano di Progetto v1.0 &  0 &  Ottimale\\
    Piano di Qualifica v1.0 & -2 & Ottimale \\
    Glossario v1.0 & +2 & Accettabile\\	
	\rowcolor{white}  
	\caption{Esiti della Schedule Variance - Attività di Analisi requisiti utente}	    	
\end{tabella}

\bold{Considerazioni finali:}
\begin{description}
\item{\bold{Schedule Variance finale:}} +1 giorno.
\\In generale per il processo di documentazione il \gl{team} ha ritardato la stesura di 1 giorno rispetto a quanto pianificato nel \doc{Piano di Progetto}. Il risultato ottenuto pertanto rientra nella soglia di accettabilità prevista.
\end{description}


\myparagraph{Cost Variance}
Vengono riportati di seguito i valori ottenuti calcolando la Cost Variance sui tempi di stesura di ogni documento:
\begin{tabella}{l!{\VRule}c!{\VRule}c!{\VRule}}
	
	\color{white} \bold{Processo} & \color{white} \bold{CV} &\color{white} \bold{Giudizio} \\
	\endfirsthead
	Processo di documentazione & 0\% & Ottimale\\
	\rowcolor{white}  
	\caption{Esiti della Cost Variance - Attività di Analisi requisiti utente}	   	
\end{tabella}

\bold{Considerazioni finali:} Per il processo di documentazione il \gl{team} non ha necessitato di ulteriori risorse che potessero aumentare i costi pianificati precedentemente.

\myparagraph{Capability Maturity }
Si è cercato di valutare la qualità del processo di documentazione secondo le metriche stabilite dal lo \gl{CMM} . Ovviamente, quando il progetto è cominciato si era al livello 1 dei livelli di Maturity: le procedure e le norme erano completamente informali, non era presente alcun tipo di documentazione e lo stato risultava caotico, non permettendo alcun tipo di ripetibilità di procedure in modo certo.
\newline In seguito alla redazione del documento \NdPdoc sono state definite  norme valide per ogni tipo di documentazione, strumenti comuni
da poter utilizzare e procedure da seguire per effettuare determinate attività. Il processo di documentazione ha in questo modo guadagnato ripetibilità come viene richiesto al livello 2 del lo \gl{CMM}.
\newline La soglia di accettabilità, descritta nella sezione \hyperref[sec:metr]{Misure e metriche}, pertanto è stata raggiunta per tale processo, ma ciò che è stato fatto non è sufficiente per raggiungere anche il livello 3.
\newline Nelle prossime attività si cercherà di apportare un incremento di qualità al processo di documentazione.

\subsubsection{Verificazione}
\myparagraph{Schedule Variance}
Vengono riportati di seguito i valori ottenuti calcolando la Schedule Variance sui tempi di verifica per ogni documento:
\begin{tabella}{l!{\VRule}c!{\VRule}c!{\VRule}}
	
	\color{white} \bold{Documento} & \color{white} \bold{SV in giorni} &\color{white} \bold{Giudizio} \\
	\endfirsthead
	Analisi dei Requisiti v1.0 & +2 & Accettabile \\
	Norme di Progetto v1.0 & -1 & Ottimale \\
    Studio di Fattibilità v1.0 &  0 &  Ottimale \\
    Piano di Progetto v1.0 &  0 &  Ottimale\\
    Piano di Qualifica v1.0 & -1 & Ottimale \\
    Glossario v1.0 & 0 & Ottimale\\	
	\rowcolor{white}  
	\caption{Esiti della Schedule Variance - Attività di Analisi requisiti utente}	    	
\end{tabella}

\bold{Considerazioni finali:}
\begin{description}
\item{\bold{Schedule Variance finale:}} 0 giorni.
\\Per il processo di verificazione, in generale, il \gl{team} ha ritardato di 0 giorni rispetto a quanto pianificato nel \doc{Piano di Progetto}. Pur essendoci documenti che sono stati verificati in ritardo, il valore generale ottenuto rientra nella soglia di accettabilità prevista.
\end{description}

\myparagraph{Cost Variance}
Vengono riportati di seguito i valori ottenuti calcolando la Cost Variance sui tempi impiegati nel processo di verificazione:
\begin{tabella}{l!{\VRule}c!{\VRule}c!{\VRule}}
	
	\color{white} \bold{Processo} & \color{white} \bold{CV} &\color{white} \bold{Giudizio} \\
	\endfirsthead
	Processo di verifica & 0\% & Ottimale\\
	\rowcolor{white}  
	\caption{Esiti della Cost Variance - Attività di Analisi requisiti utente}	  
\end{tabella}

\bold{Considerazioni finali:} Per il processo di verificazione il \gl{team} non ha necessitato di ulteriori risorse che potessero aumentare i costi pianificati precedentemente.

\myparagraph{Capability Maturity }
Il processo di verificazione è da considerarsi uno dei più importanti poiché deve risultare efficiente ed efficace per tutta la durata del progetto, dal momento che dev'essere eseguito molteplici volte ed ha un costo elevato.
\newline Come per il processo di documentazione, anche per quello di verificazione si è partiti dal livello 1 della classificazione di Maturity presente nel lo \gl{CMM}. Il modo di procedere era caotico e non ripetibile, attraverso l'individuazione di strumenti e tecniche per eseguire il processo di verifica è stato possibile renderlo più disciplinato, controllato e ripetibile (anche attraverso l'uso di strumenti automatici).
\newline Si è pertanto raggiunto il livello 2 (Ripetibile) come ci si attendeva, rispettando la soglia di accettabilità, descritta nell'appendice \hyperref[sec:metr]{Misure e metriche}.

\subsection{Verifica dei prodotti}
\subsubsection{Documenti}
\myparagraph{Errori ortografici}
Il gruppo, attraverso un processo di verifica automatica e manuale, ha individuato parecchi errori di battitura all'interno dei documenti.
\newline Vengono riportati di seguito i valori ottenuti riguardanti la verifica degli errori ortografici:
\begin{tabella}{l!{\VRule}c!{\VRule}c!{\VRule}}
	
	\color{white} \bold{Documento} & \color{white} \bold{Errori ortografici} &\color{white} \bold{Giudizio} \\
	\endfirsthead
	Analisi dei Requisiti v1.0 & 2\% & Accettabile\\
	Norme di Progetto v1.0 & 1\% & Accettabile\\
    Studio di Fattibilità v1.0 & 2\% &  Accettabile \\
    Piano di Progetto v1.0 & 2\% & Accettabile \\
    Piano di Qualifica v1.0 & 3\% & Accettabile\\
    Glossario v1.0 & 1\% & Accettabile\\	
	\rowcolor{white}  
	\caption{Esiti degli Errori Ortografici - Attività di Analisi requisiti utente}	    	
\end{tabella}

\bold{Considerazioni finali:}
Per ogni documento il \gl{team} ha svolto un buon lavoro nella correzione degli errori ortografici, anche se non tutti sono stati corretti subito. In ogni caso la percentuale di errori individuati in ogni documento rientra nella soglia di accettabilità prevista dalla metrica.

\myparagraph{Indice Gulpease}
Ogni documento è stato sottoposto ad uno \gl{script} che ne calcolasse l'\gl{Indice Gulpease} per valutarne il grado di leggibilità.
\newline Di seguito vengono presentati i dati raccolti:
\begin{tabella}{l!{\VRule}c!{\VRule}c!{\VRule}}
	
	\color{white} \bold{Documento} & \color{white} \bold{\gl{Indice Gulpease}} &\color{white} \bold{Giudizio} \\
	\endfirsthead
	Analisi dei Requisiti v1.0 &  51 & Accettabile \\
	Norme di Progetto v1.0 & 53 & Accettabile\\
    Studio di Fattibilità v1.0 & 59 & Accettabile \\
    Piano di Progetto v1.0 & 49 & Accettabile \\
    Piano di Qualifica v1.0 & 45 & Accettabile\\
    Glossario v1.0 & 59 & Accettabile\\	
	\rowcolor{white}  
	\caption{Esiti dell'\gl{Indice Gulpease} - Attività di Analisi requisiti utente}	    	
\end{tabella}
\bold{Considerazioni finali:}
Dai risultati ottenuti è possibile stabilire che ogni documento presenta un buon grado di leggibilità. Ogni valore rientra appieno nella soglia di accettabilità indicata da tale metrica.

\myparagraph{Errori Concettuali}
Il gruppo ha posto molta attenzione nell'uso di termini, concetti teorici e contenuti più generali in modo da non cadere in contraddizioni, nell'uso di terminologie sbagliate o di concetti teorici errati. Durante il processo di verificazione dei documenti pertanto non è stato riscontrato nessun tipo di errore concettuale a cui sia stato necessario apportarne una modifica. 
\newline La soglia di accettabilità in tal modo viene rispettata e addirittura viene raggiunto il valore ottimale per tale metrica, descritta nell'appendice \hyperref[sec:metr]{Misure e metriche}.

\myparagraph{Errori di forma}
Sono stati segnalati, dai \italics{Verificatori}, diversi errori di forma che non rispettavano le norme descritte nel documento \italics{Norme di Progetto}. La maggior parte sono stati corretti, ma una minima parte è sfuggita durante l'azione di correzione e pertanto sono stati segnalati durante una nuova azione di verifica.
\newline In ogni caso la percentuale di errori di forma, per come è descritta tale metrica nell'appendice \hyperref[sec:metr]{Misure e metriche}, risulta pari all'1\%; valore che rientra pienamente all'interno della soglia di accettabilità prevista.


\subsection {Sommario del resoconto delle sottoattività di verifica}

\begin{tabella}{l!{\VRule}>{\centering\arraybackslash}p{6 cm}!{\VRule}>{\centering\arraybackslash}p{2 cm}!{\VRule}>{\centering\arraybackslash}p{2 cm}}

		
	
	\color{white} \bold{Verifica} & \color{white} \bold{Metrica} & \color{white} \bold{Giudizio finale} \\
	\endfirsthead
	
	\cellcolor{P} & Schedule Variance & Accettabile\\
	\cellcolor{P} & Cost Variance & Ottimale \\
	\multirow{-3}{*}{\cellcolor{P}Processo di Documentazione}	& Capability Maturity  & Accettabile \\
	\hline
	
	\cellcolor{D} & Schedule Variance & Accettabile \\
	\cellcolor{D} & Cost Variance & Ottimale \\
	\multirow{-2}{*}{\cellcolor{D}Processo di Verificazione} & Capability Maturity  & Accettabile \\
	\hline
	
	\cellcolor{P} & Errori Ortografici & Accettabile \\
	\cellcolor{P} & \gl{Indice Gulpease} & Accettabile \\
	\multirow{-2}{*}{\cellcolor{P} Documenti} & Errori Concettuali & Ottimale \\ & Errori di Forma & Accettabile \\
	\hline
		

	\caption{Riassunto del Resoconto delle sottoattività di verifica - Attività di Analisi requisiti utente}	    	
	
\end{tabella}

\newpage
\section{Attività di Progettazione architetturale \\\large{resoconto~delle~sottoattività~di~verifica}}
\label{app:valtest}


\subsection{Verifica dei processi}
\subsubsection{Documentazione}
\myparagraph{Schedule Variance}
Vengono riportati di seguito i valori ottenuti calcolando la Schedule Variance sui tempi di stesura di ogni documento:
\begin{tabella}{l!{\VRule}c!{\VRule}c!{\VRule}}
	\color{white} \bold{Documento} & \color{white} \bold{SV in giorni} &\color{white} \bold{Giudizio} \\
	\endfirsthead
	Analisi dei Requisiti v3.0 & -2 & Ottimale \\
	Norme di Progetto v3.0 & 0 & Ottimale \\
    Piano di Progetto v3.0 &  +2 &  Accettabile\\
    Piano di Qualifica v3.0 & 0 & Ottimale \\
    Glossario v2.0 & -1 & Ottimale\\	
    Specifica Tecnica v1.0 & +6 & Negativo \\
	\rowcolor{white}  
	\caption{Esiti della Schedule Variance - Attività di Progettazione architetturale}	    	
\end{tabella}

\bold{Considerazioni finali:}
\begin{description}
\item{\bold{Schedule Variance finale:}} +5 giorni.
\\Durante l'attività di Progettazione architetturale il team ha concluso la stesura della documentazione con un ritardo di 5 giorni da quanto riportato nel documento \doc{Piano di Progetto}. Pur essendo un valore non accettabile che fuoriesce dal range assunto in tale metrica, il gruppo non considera necessario assumere delle azioni correttive dal momento che il problema della malattia prolungata di parte del team viene considerato un evento eccezionale che difficilmente si potrà manifestare nuovamente in maniera così gravosa nelle successive attività.
\end{description}


\myparagraph{Cost Variance}
Vengono riportati di seguito i valori ottenuti calcolando la Cost Variance sui tempi di stesura di ogni documento:
\begin{tabella}{l!{\VRule}c!{\VRule}c!{\VRule}}
	
	\color{white} \bold{Processo} & \color{white} \bold{CV} &\color{white} \bold{Giudizio} \\
	\endfirsthead
	Processo di documentazione & 4 \% & Accettabile\\
	\rowcolor{white}  
	\caption{Esiti della Cost Variance - Attività di Progettazione architetturale}	   	
\end{tabella}

\bold{Considerazioni finali:}
\\Per il processo di documentazione il gruppo ha necessitato lo sfruttamento di un maggior numero di risorse (umane) a causa della malattia di alcuni membri del team affinchè venissero ricoperte anche le loro mansioni. Ciò nonostante il valore di cost variance rientra nei parametri di accettazione definiti dal tale metrica.


\myparagraph{Capability Maturity }
L'attità di Progettazione architetturale è iniziata con un livello di Maturity pari a 2.
\\ Attraverso l'incremento dei documenti con ad esempio l'individuazione di nuove metriche che possano migliorare la qualità dei processi e grazie all'esperienza accumulata dai componenti del team, i processi e la loro organizzazione sono migliorati portando a raggiungere il livello 3 del lo CMM.
\\ Tale risultato è accettabile secondo quanto riportato nella corrispettiva metrica, ma il gruppo cercherà di migliorare ulteriormente per le attività successive.

\subsubsection{Verificazione}
\myparagraph{Schedule Variance}
Vengono riportati di seguito i valori ottenuti calcolando la Schedule Variance sui tempi di verifica per ogni documento:
\begin{tabella}{l!{\VRule}c!{\VRule}c!{\VRule}}
	
	\color{white} \bold{Documento} & \color{white} \bold{SV in giorni} &\color{white} \bold{Giudizio} \\
	\endfirsthead
	Analisi dei Requisiti v3.0 & +1 & Accettabile \\
	Norme di Progetto v3.0 & 0 & Ottimale \\
    Piano di Progetto v3.0 &  0 &  Ottimale\\
    Piano di Qualifica v3.0 & +1 & Accettabile \\
    Glossario v2.0 & -2 & Ottimale\\	
   	Specifica Tecnica v1.0 &  +4 &  Accettabile \\
	\rowcolor{white}  
	\caption{Esiti della Schedule Variance - Attività di Progettazione architetturale}	    	
\end{tabella}

\bold{Considerazioni finali:}
\begin{description}
\item{\bold{Schedule Variance finale:}} +4 giorni.
\\ Per il processo di verificazione il team ha ritardato di 4 giorni rispetto a quanto pianificato, a causa del ritardo provocato dalla stesura della documentazione, come riportato sopra. 
\\ Il valore rientra in ogni caso nel range accettabile definito per tale metrica, per cui non si necessitano di misure correttive da definire.
\end{description}

\myparagraph{Cost Variance}
Vengono riportati di seguito i valori ottenuti calcolando la Cost Variance sui tempi impiegati nel processo di verificazione:
\begin{tabella}{l!{\VRule}c!{\VRule}c!{\VRule}}
	
	\color{white} \bold{Processo} & \color{white} \bold{CV} &\color{white} \bold{Giudizio} \\
	\endfirsthead
	Processo di verifica & 0\% & Ottimale\\
	\rowcolor{white}  
	\caption{Esiti della Cost Variance - Attività di Progettazione architetturale}	  
\end{tabella}

\bold{Considerazioni finali:}
\\Per il processo di verificazione il team ha usufruito delle risorse in proprio possesso senza alcuna necessità aggiuntiva. In tal modo i costi pianificati sono stati rispettati perfettamente.

\myparagraph{Capability Maturity }
All'inizio dell'attività di Progettazione architetturale il livello di Maturity del processo di verificazione era pari a 2.
\\ Il team ha migliorato il proprio processo di verifica dei documenti attraverso l'utilizzo delle metriche definite in questo documento, consolidando quanto appreso nell'attività precedente e apportandone valore aggiunto, arrivando in tal modo a raggiungere il livello 3 del lo CMM.

\subsection{Verifica dei prodotti}
\subsubsection{Documenti}
\myparagraph{Errori ortografici}
Il gruppo, attraverso un processo di verifica automatica e manuale, ha riscontrato pochi errori ortografici nei documenti che hanno subito solo un incremento, mentre parecchi errori nella stesura del documento \STdoc.
\newline Vengono riportati di seguito i valori ottenuti riguardanti la verifica degli errori ortografici:
\begin{tabella}{l!{\VRule}c!{\VRule}c!{\VRule}}
	
	\color{white} \bold{Documento} & \color{white} \bold{Errori ortografici} &\color{white} \bold{Giudizio} \\
	\endfirsthead
	Analisi dei Requisiti v3.0 &  1 \% & Accettabile\\
	Norme di Progetto v3.0 & 0 \% & Ottimale\\
    Piano di Progetto v3.0 & 1 \% & Accettabile \\
    Piano di Qualifica 3.0 & 1 \% & Accettabile\\
    Glossario v2.0 & 0 \% & Ottimale\\
    Specifica Tecnica v1.0 & 3 \% &  Accettabile \\	
	\rowcolor{white}  
	\caption{Esiti degli Errori Ortografici - Attività di Progettazione architetturale}	    	
\end{tabella}

\bold{Considerazioni finali:}
\\ Per ogni documento che abbia subito un incremento rispetto alla versione precedente, il team ha svolto un buon lavoro nella correzione degli errori ortografici; solamente nella stesura del documento \STdoc si sono riscontrati maggiori errori che non sono stati corretti subito. Ciò nonostante la percentuale di errori individuati in ogni documento rientra nella soglia di accettabilità prevista da tale metrica.

\myparagraph{Indice Gulpease}
Ogni documento è stato sottoposto ad uno \gl{script} che ne calcolasse l'\gl{Indice Gulpease} per valutarne il grado di leggibilità.
\newline Di seguito vengono presentati i dati raccolti:
\begin{tabella}{l!{\VRule}c!{\VRule}c!{\VRule}}
	
	\color{white} \bold{Documento} & \color{white} \bold{\gl{Indice Gulpease}} &\color{white} \bold{Giudizio} \\
	\endfirsthead
	Analisi dei Requisiti v3.0 & ???? & ???? \\
	Norme di Progetto v3.0 & ???? & ????\\
    Piano di Progetto v3.0 & ???? & ???? \\
    Piano di Qualifica v3.0 & ???? & ????\\
    Glossario v3.0 & ???? & ????\\	
    Specifica Tecnica v1.0 & ???? & ???? \\
	\rowcolor{white}  
	\caption{Esiti dell'\gl{Indice Gulpease} - Attività di Progettazione architetturale}	    	
\end{tabella}
\bold{Considerazioni finali:}
\\ \bold{SPIEGAZIONE}

\myparagraph{Errori Concettuali}
Durante il processo di verificazione dei documenti non è stato riscontrato nessun tipo di errore concettuale all'interno dei documenti. La soglia di accettabilità in tal modo viene rispettata pienamente, ragggiungendo il valore ottimale per tale metrica.

\myparagraph{Errori di forma}
Avendo consolidato le norme descritte nel documento \doc{Norme di progetto} il gruppo si è attenuto perfettamente alla forma nelle stesura e nell'incremento dei documenti. Pochissimi sono stati gli errori di forma segnalati, che sono stati prontamente corretti da chi di dovere. 	
\\ La percentuale di errori di forma pertanto è pari allo 0\%, valore ottimale per tale metrica.


\subsection {Sommario del resoconto delle sottoattività di verifica}

\begin{tabella}{l!{\VRule}>{\centering\arraybackslash}p{6 cm}!{\VRule}>{\centering\arraybackslash}p{2 cm}!{\VRule}>{\centering\arraybackslash}p{2 cm}}

		
	
	\color{white} \bold{Verifica} & \color{white} \bold{Metrica} & \color{white} \bold{Giudizio finale} \\
	\endfirsthead
	
	\cellcolor{P} & Schedule Variance & Negativo\\
	\cellcolor{P} & Cost Variance & Accettabile \\
	\multirow{-3}{*}{\cellcolor{P}Processo di Documentazione}	& Capability Maturity  & Accettabile \\
	\hline
	
	\cellcolor{D} & Schedule Variance & Accettabile \\
	\cellcolor{D} & Cost Variance & Ottimale \\
	\multirow{-2}{*}{\cellcolor{D}Processo di Verificazione} & Capability Maturity  & Accettabile \\
	\hline
	
	\cellcolor{P} & Errori Ortografici & Accettabile\\
	\cellcolor{P} & \gl{Indice Gulpease} & ??????? \\
	\multirow{-2}{*}{\cellcolor{P} Documenti} & Errori Concettuali & Ottimale \\ & Errori di Forma & Ottimale \\
	\hline
		

	\caption{Riassunto del Resoconto delle sottoattività di verifica - Attività di Progettazione architetturale}	     	
	
\end{tabella}


\newpage
\appendix
\section{Design Pattern}
In questa sezione vengono presentati i design pattern che al momento sono stati stabiliti dal team per l'architettura del nostro prodotto.

\subsection{Model-View-Controller (MVC)}

Il Model-View-Controller (MVC) è un design pattern architetturale utilizzato per dividere il codice in blocchi di funzionalità ben distinte, e viene utilizzato molto frequentemente nelle applicazioni in cui un insieme di informazioni deve essere rappresentato attraverso una interfaccia grafica.

\subsubsection{Componenti}
MVC è basato sul principio del disaccoppiamento dei tre oggetti di cui è composto, ovvero sulla riduzione del loro grado di dipendenza reciproca, allo scopo di fornire una maggiore robustezza, modularità e manutenibilità al software.
\newline Di seguito viene riportata una breve descrizione delle sue componenti e delle loro caratteristiche.

\myparagraph{Model}
Il Model è il nucleo dell’applicazione. Definisce il modello dei dati realizzando la business logic definendo gli oggetti secondo la logica di utilizzo dell’applicazione e indicando inoltre le possibili operazioni effettuabili su di essi.
\newline Questa componente viene generalmente progettata attraverso tecniche object oriented.
\newline Nella struttura del pattern MVC, il Model è una componente passiva, che non ha relazioni uscenti forti, ma si occupa di notificare alla View eventuali aggiornamenti attraverso il Controller.

\myparagraph{View}
La View si occupa di visualizzare i dati contenuti nel Model e fornisce l’interfaccia di interazione con gli utenti che utilizzeranno il sistema.
Nella struttura del pattern MVC, la View cattura gli input dell’utente e li passa al Controller affinché esegua le corrette operazioni sul Model.
\newline La View può essere soggetta a due tipi di aggiornamento:
\begin{description}
\item{Push model} 
\\Il Push model viene implementato attraverso il pattern Observer e soltanto quando MVC è usato in un solo ambiente di esecuzione; consiste nel fatto che sia il Model a emettere, senza essere sollecitato, la notifica in seguito alla quale la View verrà aggiornata.
\item{Pull model}
\\Il Pull model viene invece implementato quando MVC è usato su diversi ambienti di esecuzione; in questo caso la View richiede al Model se sia necessario un aggiornamento solo al verificarsi di particolari eventi.
\end{description}

\myparagraph{Controller}
Il Controller è la componente fondamentale di comunicazione tra la View ed il Model. Si occupa infatti di implementare l’application logic, l’insieme di operazioni eseguibili sul modello dei dati attraverso una particolare vista.
\newline Poiché potenzialmente possono esistere più View sarà necessario attribuire per ogni View un corrispondente Controller.

\subsubsection{Vantaggi}
L'utilizzo del design pattern MVC porta la riduzione delle dipendenze tra le componenti dell'architettura, offrendo diversi vantaggi. La modularità ottenuta inoltre permette il riutilizzo del Model in applicazioni con diverse View, oltre a rendere più semplice e rapida l’esecuzione dei test.

\subsubsection{Svantaggi}
L’utilizzo di MVC nello sviluppo di un software presenta anche degli svantaggi in qualità di costi: la complessità risulta maggiore e si necessita un maggior  numero di aggiornamenti.
\newline La maggiore complessità è causata dai livelli di indirezione introdotti dal pattern e dall’implementazione ad eventi necessaria per far comunicare le tre componenti.
\newline La maggiore frequenza degli aggiornamenti, invece, è un effetto collaterale della modularizzazione dal momento che frequenti cambiamenti alla struttura del modello comportano spesso dei cambiamenti nelle View ed anche nei Controller.

\subsection{Model-View-ViewModel (MVVM)}
L'utilizzo di AngularJS all'interno del progetto porta a non implementare unicamente il design pattern MVC, ma 
necessita dell'implementazione di una sua variante, il Model-View-ViewModel (MVVM).

\subsubsection{Componenti}
Segue una breve descrizione delle componenti che caratterizzano il pattern MVVM.

\begin{description}
\item{Model} 
\\ Il Model è il dato nell’applicazione, che viene rappresentato tramite un normale oggetto JavaScript.
\item{ViewModel}
\\ Il ViewModel fa da intermediario tra la vista e il modello, ed è responsabile per la gestione della logica della vista. In generale interagisce con il modello invocando i metodi presenti nelle sue classi.
Il ViewModel fornisce quindi i dati dal modello in una forma che la vista possa interpretare facilmente.
\newline Con Angular il ViewModel sarà rappresentato dall’oggetto \$scope, una specie di variabile globale JavaScript che grazie alle proprie API permetterà di rilevare e trasmette i cambiamenti di stato.
\item{Controller}
\\ Il controller è il responsabile dell’impostare lo stato iniziale e aumentare lo \$scope con metodi che controllano il comportamento.
\item{View}
\\ La view è l’interfaccia grafica che dovrà visualizzare i dati e su cui interagirà l'utente.
\end{description}

Lo \$scope ha un riferimento ai dati, il controller definisce il comportamento, la view gestisce il layout e consegna l’interazione al controller per rispondere di conseguenza.

\subsection{Data-Access-Object (DAO)}
Il Data-Access-Object è un design pattern strutturale che permette di svolgere delle specifiche operazioni sui dati senza conoscere i dettagli del database. Questa separazione supporta il Principio di singola responsabilità: separa l'accesso a quali dati l'applicazione necessita dal come questi dati vengono reperiti attraverso il database.
Pertanto questo pattern viene utilizzato per disaccoppiare la logica di business dalla logica di accesso ai dati. 

\subsubsection{Vantaggi}
Il vantaggio di usare questo pattern è rappresentato dalla sua semplicità e dalla rigorosa separazione tra le due principali parti dell'applicazione che non necessitano di conoscere ogni aspetto l'una dell'altra e che permettono di essere espanse in modo indipendente.
Tutti i dettagli del salvataggio dei dati sono nascosti dal resto dell'applicazione, in questo modo possono essere cambiati senza influenzare tutto il resto dell'applicazione, ma modificando soltanto una implementazione del DAO.
\newline Il DAO pertanto funge da intermediario tra l'applicazione e il database.

\subsubsection{Svantaggi}
Questo pattern, pur essendo pochi, presenta comunque degli svantaggi. Il più importante riguarda la duplicazione di codice che comparirà in ogni singolo DAO presente nell'applicazione. Infatti viene creata un DAO per ogni entità presente nel sistema, che conterrà i metodi necessari per interagire con il database, ad esempio metodi per ricercare, inserire, modificare,rimuovere e aggiornare i dati.

\subsection{Microservices Architecture}
Il Microservice Architecture è un pattern architetturale ancora giovane, il cui pregio maggiore consiste nello sviluppare un'applicazione come una famiglia di piccoli services, ognuno che esegue il proprio processo in modo indipendente l'uno dall'altro, pur mantenendo comunque un minimo di gestione centralizzata di questi services.
\\ Il difetto principale di tale architettura è la duplicazione di codice che interesserà tutti i diversi services che vengono generati durante la cosifica del sistema.

\subsection{Builder}
Il Builder è un design pattern creazionale utilizzato per separare la costruzione di un oggetto complesso dalla sua effettiva rappresentazione. In tal modo l'algoritmo di costruzione di oggetti di tipi diversi sarà indipendente da i vari step necessari alla loro realizzazione.
\\ Il vantaggio di ciò è rendere la classe più semplice assegnando il compito ad una classe builder di costruire effettivamente l'oggetto complesso; in tal modo la classe potrà focalizzarsi sul corretto funzionamento degli oggetti creati.
Inoltre le modifiche apportate alla rappresentazione interna di un prodotto saranno semplificate.
\\ Tale design pattern verrà successivamente utilizzato per la costruzione del menù nel lato client del sistema.

\subsubsection{Componenti}
\begin{description}\item{Builder}
\\ Specifica l'interfaccia astratta che crea le parti dell'oggetto Product.
\item {ConcreteBuilder} 
\\Costruisce e assembla le parti del prodotto implementando l'interfaccia Builder; definisce e tiene inoltre traccia della rappresentazione che crea.
\item{Director} 
\\Costruisce un oggetto utilizzando l'interfaccia Builder.
\item{Product} 
\\Rappresenta l'oggetto complesso e include le classi che definiscono le parti che lo compongono, includendo le interfacce per assemblare le parti nel risultato finale.
\end{description}


\end{document}