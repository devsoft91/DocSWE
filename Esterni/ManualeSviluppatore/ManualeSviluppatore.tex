% Nome del file: ManualeSviluppatore.tex
% Percorso: \gl{template}
% Autore: Vault-Tech
% Data creazione: 10.05.2016
% E-mail: vaulttech.swe@gmail.comcom
%
% Diario delle modifiche: interno al file.

\documentclass[a4paper, titlepage]{article}

\usepackage[margin=3cm]{geometry}
\usepackage{../../Stile}
\usepackage{../../Comandi}

\setcounter{secnumdepth}{5}
\setcounter{tocdepth}{5}

\def\NOME{Manuale Sviluppatore}
\def\VERSIONE{1.0}
\def\DATA{13.05.2016}
\def\REDATTORE{Filippo Tesser \\ & Giacomo Beltrame}
\def\VERIFICATORE{Miki Violetto}
\def\RESPONSABILE{Giacomo Beltrame}
\def\USO{Esterno}
\def\DISTRIBUZIONE{\COMMITTENTE \\ & \CARDIN \\ & \PROPONENTE}


\begin{document}
	
	\pagestyle{fancy}	
	\pagenumbering{Roman}
	\rfoot{Pagina \thepage{} di \pageref{lastromanpage}}
	
	\maketitle
	
	\begin{diario}
	\recap{Approvazione del documento}{Michela De Bortoli}{Responsabile}{06.04.2016}{3.0}
	\recap{Correzione errori individuati}{Michela De Bortoli}{Analista}{06.04.2016}{2.10}
	\recap{Verifica dell'intero documento}{Rudy Berton}{Verificatore}{05.04.2016}{2.9}
	\recap{Stesura appendice D}{Giacomo Beltrame}{Analista}{04.04.2016}{2.8}
	\recap{Verifica appendici A e B}{Giacomo Beltrame}{Verificatore}{03.04.2016}{2.7}
	\recap{Stesura test di integrazione}{Rudy Berton}{Amministratore}{02.04.2016}{2.6}
	\recap{Stesura test di sistema}{Vassilikì Menarin}{Progettista}{02.04.2016}{2.5}
	\recap{Modifica della sezione A.3.3 dell'appendice}{Filippo Tesser}{Analista}{01.04.2016}{2.4}
	\recap{Incremento test di accettazione}{Michela De Bortoli}{Progettista}{01.04.2016}{2.3}
	\recap{Inizio stesura specifica dei test (appendice B)}{Michela De Bortoli}{Progettista}{31.03.2016}{2.2}
	\recap{Incremento dell'appendice A}{Filippo Tesser}{Analista}{31.03.2016}{2.1}
	\recap{Approvazione documento}{Miki Violetto}{Responsabile}{23.02.2016}{2.0}
	\recap{Verifica delle sezioni modificate}{Rudy Berton}{Verificatore}{22.02.2016}{1.2}
	\recap{Revisione correttiva dei contenuti rispetto alle segnalazioni del committente}{Giacomo Beltrame}{Analista}{20.02.2016}{1.1}
	\recap{Approvazione documento}{Vassilikì Menarin}{Responsabile}{20.01.2016}{1.0}
	\recap{Verifica del documento}{Simone Boccato}{Verificatore}{19.01.2016}{0.9}
	\recap{Stesura appendice D}{Rudy Berton}{Analista}{18.01.2016}{0.8}
	\recap{Correzione errori segnalati}{Rudy Berton}{Analista}{16.01.2016}{0.7}
	\recap{Verifica del documento}{Filippo Tesser}{Verificatore}{15.01.2016}{0.6}
	\recap{Stesura appendici A, B e C}{Rudy Berton}{Analista}{11.01.2016}{0.5}
	\recap{Fine stesura Gestione della qualità e stesura sezione Gestione amministrativa della revisione}{Rudy Berton}{Analista}{08.01.2016}{0.4}
	\recap{Inizio stesura Gestione della qualità}{Rudy Berton}{Analista}{05.01.2016}{0.3}
	\recap{Stesura sezione Obiettivi di qualità}{Rudy Berton}{Analista}{03.01.2016}{0.2}
	\recap{Stesura sezione Introduzione}{Rudy Berton}{Analista}{02.01.2016}{0.1}
\end{diario}
	
	\newpage
	\tableofcontents
%	\newpage
%	\listoffigures
%	\newpage
%	\listoftables
\label{lastromanpage}
	
	\newpage
	\clearpage	
	\pagenumbering{arabic}
	\rfoot{Pagina \thepage{} di \pageref*{LastPage}}
	%Deve esserci per permettere i riferimenti incrociati di colore blu
	\hypersetup{linkcolor=blue}
	
	\section{Introduzione}
	\subsection{Scopo del documento}
	Questo documento ha lo scopo di indicare e spiegare quali sono i comandi da eseguire per installare correttamente l'applicativo Quizzipedia.
	
	\subsection{Scopo del prodotto}
	\SCOPO
	
	\subsection{Riferimenti}	
	\subsubsection{Riferimenti informativi}
	\begin{itemize}
		\item \textbf{Manuale utente \gl{Git}:} \url{https://git-scm.com/docs/user-manual.html};
		\item \textbf{Manuale installazione \gl{Node.js}:} \url{https://docs.npmjs.com/getting-started/installing-node};
		\item \textbf{Manuale installazione \gl{MongoDB}:} \url{https://docs.mongodb.com/master/installation/};
		\item \textbf{Front-end package manager Bower}: \url{http://bower.io/};
		\item \textbf{Grunt JavaScript task runner}: \url{http://gruntjs.com/}.
	\end{itemize}
	\newpage
	
	\section{Prerequisiti software}
	Per poter avviare Quizzipedia è richiesta l'installazione dei seguenti software.
	
	\subsection{Node.js e npm}
	Scaricare e installare \gl{Node.js} e il package manager \gl{npm} da \url{https://nodejs.org/en/download/current/}.
	
	Verificare che sia visibile la versione installata con i seguenti comandi:
	
	\texttt{\$ node -v}
	
	\texttt{\$ npm -v}

	\subsection{MongoDB}
	Scaricare e installare \gl{MongoDB} da \url{https://www.mongodb.com/download-center}.
	
	Per avviare \gl{MongoDB} sarà necessario dare il seguente comando:
	
	\texttt{\$ mongod}

	\subsection{Forever}
	Scaricare e installare globalmente \gl{forever} utilizzando \gl{npm} con il seguente comando:

	\texttt{\$ npm install -g forever}
	
	\subsection{Angular.js}
	Scaricare e installare \gl{Angular.js} utilizzando \gl{npm} con il seguente comando:

	\texttt{\$ npm install angular}
	
	\section{Requisiti hardware}
	Il prodotto lato server richiede come requisiti minimi:
	\begin{itemize}
		\item processore dual core;
		\item 2GB di memoria RAM;
		\item 1GB di spazio libero su disco.
	\end{itemize}
	
	\section{Installazione e Avvio}
	
	\subsection{Ottenere Quizzipedia}
	\subsubsection{Download da browser}
	Reperire il progetto Quizzipedia è possibile recandosi con qualunque browser all'indirizzo \url{http://github.com/vault-tech/Quizzipedia}.
	A questo punto è possibile scaricare il codice sorgente con l'apposito tasto \texttt{Download ZIP}.

	\subsubsection{Download da riga di comando}
	Dopo aver aperto un terminale eseguire il seguente comando:

	\texttt{\$ git clone http://github.com/vault-tech/Quizzipedia dest}

	dove "dest" va sostituita con il percorso alla cartella nella quale si vuole installare Quizzipedia.

	\subsection{Installazione dipendenze}
	
	Una volta ottenuto il codice sorgente, spostarsi all'interno della root del repository, si controlli che sia presente il file package.json.
	
	Sempre nella directory corrente eseguire poi il seguente comando da terminale:
	
	\texttt{\$ npm install}
	
	in questo modo verrano installati automaticamente i moduli di \gl{Node.js} necessari per l'esecuzione del prodotto.
	
	\subsection{Avvio}
	
	Per avviare Quizzipedia e mantenerlo in esecuzione, utilizzare il modulo \gl{npm} \gl{forever} precedentemente installato digitando il seguente comando dopo essersi spostati nella root di Quizzipedia:

	\texttt{\$ PORT=porta forever start server.js}

	Sostituire "porta" con il numero di porta sul quale si vuole rendere reperibile Quizzipedia via browser;\\
	se non viene specificata verrà usata di default la 8080;\\
	se viene specificata una porta inferiore a 1024 allora sarà necessario digitare il comando \texttt{sudo} seguito dal resto per autorizzare \gl{Node.js} ad utilizzare tale porta.
	
\end{document}