% Nome del file: ManualeSviluppatore.tex
% Percorso: \gl{template}
% Autore: Vault-Tech
% Data creazione: 10.05.2016
% E-mail: vaulttech.swe@gmail.comcom
%
% Diario delle modifiche: interno al file.

\documentclass[a4paper, titlepage]{article}

\usepackage[margin=3cm]{geometry}
\usepackage{../../Stile}
\usepackage{../../Comandi}

\setcounter{secnumdepth}{5}
\setcounter{tocdepth}{5}

\def\NOME{Manuale Sviluppatore}
\def\VERSIONE{1.0}
\def\DATA{13.05.2016}
\def\REDATTORE{Filippo Tesser \\ & Giacomo Beltrame}
\def\VERIFICATORE{Miki Violetto}
\def\RESPONSABILE{Giacomo Beltrame}
\def\USO{Esterno}
\def\DISTRIBUZIONE{\COMMITTENTE \\ & \CARDIN \\ & \PROPONENTE}


\begin{document}
	
	\pagestyle{fancy}	
	\pagenumbering{Roman}
	\rfoot{Pagina \thepage{} di \pageref{lastromanpage}}
	
	\maketitle
	
	\begin{diario}
	\recap{Approvazione del documento}{Michela De Bortoli}{Responsabile}{06.04.2016}{3.0}
	\recap{Correzione errori individuati}{Michela De Bortoli}{Analista}{06.04.2016}{2.10}
	\recap{Verifica dell'intero documento}{Rudy Berton}{Verificatore}{05.04.2016}{2.9}
	\recap{Stesura appendice D}{Giacomo Beltrame}{Analista}{04.04.2016}{2.8}
	\recap{Verifica appendici A e B}{Giacomo Beltrame}{Verificatore}{03.04.2016}{2.7}
	\recap{Stesura test di integrazione}{Rudy Berton}{Amministratore}{02.04.2016}{2.6}
	\recap{Stesura test di sistema}{Vassilikì Menarin}{Progettista}{02.04.2016}{2.5}
	\recap{Modifica della sezione A.3.3 dell'appendice}{Filippo Tesser}{Analista}{01.04.2016}{2.4}
	\recap{Incremento test di accettazione}{Michela De Bortoli}{Progettista}{01.04.2016}{2.3}
	\recap{Inizio stesura specifica dei test (appendice B)}{Michela De Bortoli}{Progettista}{31.03.2016}{2.2}
	\recap{Incremento dell'appendice A}{Filippo Tesser}{Analista}{31.03.2016}{2.1}
	\recap{Approvazione documento}{Miki Violetto}{Responsabile}{23.02.2016}{2.0}
	\recap{Verifica delle sezioni modificate}{Rudy Berton}{Verificatore}{22.02.2016}{1.2}
	\recap{Revisione correttiva dei contenuti rispetto alle segnalazioni del committente}{Giacomo Beltrame}{Analista}{20.02.2016}{1.1}
	\recap{Approvazione documento}{Vassilikì Menarin}{Responsabile}{20.01.2016}{1.0}
	\recap{Verifica del documento}{Simone Boccato}{Verificatore}{19.01.2016}{0.9}
	\recap{Stesura appendice D}{Rudy Berton}{Analista}{18.01.2016}{0.8}
	\recap{Correzione errori segnalati}{Rudy Berton}{Analista}{16.01.2016}{0.7}
	\recap{Verifica del documento}{Filippo Tesser}{Verificatore}{15.01.2016}{0.6}
	\recap{Stesura appendici A, B e C}{Rudy Berton}{Analista}{11.01.2016}{0.5}
	\recap{Fine stesura Gestione della qualità e stesura sezione Gestione amministrativa della revisione}{Rudy Berton}{Analista}{08.01.2016}{0.4}
	\recap{Inizio stesura Gestione della qualità}{Rudy Berton}{Analista}{05.01.2016}{0.3}
	\recap{Stesura sezione Obiettivi di qualità}{Rudy Berton}{Analista}{03.01.2016}{0.2}
	\recap{Stesura sezione Introduzione}{Rudy Berton}{Analista}{02.01.2016}{0.1}
\end{diario}
	
	\newpage
	\tableofcontents\label{lastromanpage}
%	\newpage
%	\listoffigures
%	\newpage
%	\listoftables
	
	\newpage
	\clearpage	
	\pagenumbering{arabic}
	\rfoot{Pagina \thepage{} di \pageref*{LastPage}}
	%Deve esserci per permettere i riferimenti incrociati di colore blu
	\hypersetup{linkcolor=blue}
	
	\section{Introduzione}
	\subsection{Scopo del documento}
	Questo documento ha lo scopo di indicare e spiegare quali sono i comandi da eseguire per installare correttamente l'applicativo Quizzipedia.
	
	\subsection{Scopo del prodotto}
	\SCOPO
	
	\subsection{Glossario}
	Al fine di evitare ogni ambiguità nel linguaggio e massimizzare la comprensione dei documenti, i termini tecnici, gli acronimi e le abbreviazioni che necessitano di definizione sono riportati nell'\hyperref[gl]{appendice A}.
	Inoltre ogni occorrenza di un vocabolo presente nel “Glossario” sarà posta in corsivo e seguita da
	una ‘g’ maiuscola a pedice (p.es. \gl{Parola}).
		
	\subsection{Riferimenti}	
	\subsubsection{Riferimenti informativi}
	\begin{itemize}
		\item \textbf{Manuale utente \gl{Git}:} \url{https://git-scm.com/docs/user-manual.html};
		\item \textbf{Manuale installazione \gl{Node.js}:} \url{https://docs.npmjs.com/getting-started/installing-node};
		\item \textbf{Manuale installazione \gl{MongoDB}:} \url{https://docs.mongodb.com/master/installation/};
		\item \textbf{Front-end package manager Bower}: \url{http://bower.io/};
		\item \textbf{Grunt JavaScript task runner}: \url{http://gruntjs.com/}.
	\end{itemize}
	\newpage
	
	\section{Prerequisiti software}
	Per poter avviare Quizzipedia è richiesta l'installazione dei seguenti \gl{software}.
	
	\subsection{Node.js e npm}
	Scaricare e installare \gl{Node.js} e il package manager \gl{npm} da \url{https://nodejs.org/en/download/current/}.
	
	Verificare che sia visibile la versione installata con i seguenti comandi:
	
	\texttt{\$ node -v}
	
	\texttt{\$ npm -v}

	\subsection{MongoDB}
	Scaricare e installare \gl{MongoDB} da \url{https://www.mongodb.com/download-center}.
	
	Per avviare \gl{MongoDB} sarà necessario dare il seguente comando:
	
	\texttt{\$ mongo}

	\subsection{Forever}
	Scaricare e installare globalmente il modulo 'forever' utilizzando \gl{npm} con il seguente comando:

	\texttt{\$ npm install -g forever}
	
	\subsection{Angular.js}
	Scaricare e installare \gl{Angular.js} utilizzando \gl{npm} con il seguente comando:

	\texttt{\$ npm install angular}
	
	\section{Requisiti hardware}
	Il prodotto lato \gl{server} richiede come requisiti minimi:
	\begin{itemize}
		\item processore dual core;
		\item 2GB di memoria RAM;
		\item 1GB di spazio libero su disco.
	\end{itemize}
	
	\section{Installazione e Avvio}
	
	\subsection{Ottenere Quizzipedia}
	\subsubsection{Download da browser}
	Reperire il progetto Quizzipedia è possibile recandosi con qualunque \gl{browser} all'indirizzo \url{http://github.com/vault-tech/Quizzipedia}.
	A questo punto è possibile scaricare il codice sorgente con l'apposito tasto \texttt{Download ZIP}.

	\subsubsection{Download da riga di comando}
	Dopo aver aperto un terminale eseguire il seguente comando:

	\texttt{\$ git clone http://github.com/vault-tech/Quizzipedia dest}

	dove "dest" va sostituita con il percorso alla cartella nella quale si vuole installare Quizzipedia.

	\subsection{Installazione dipendenze}
	
	Una volta ottenuto il codice sorgente, spostarsi all'interno della root del \gl{repository}, si controlli che sia presente il file package.json.
	
	Sempre nella directory corrente eseguire poi il seguente comando da terminale:
	
	\texttt{\$ npm install}
	
	in questo modo verranno installati automaticamente i moduli di \gl{Node.js} necessari per l'esecuzione del prodotto.
	
	\subsection{Avvio}
	
	Per avviare Quizzipedia e mantenerlo in esecuzione, utilizzare il modulo \gl{npm} 'forever' precedentemente installato digitando il seguente comando dopo essersi spostati nella root di Quizzipedia:

	\texttt{\$ PORT=porta forever start server.js}

	Sostituire "porta" con il numero di porta sul quale si vuole rendere reperibile Quizzipedia via \gl{browser};\\
	se non viene specificata verrà usata di default la 8080;\\
	se viene specificata una porta inferiore a 1024 allora sarà necessario digitare il comando \texttt{sudo} seguito dal resto per autorizzare \gl{Node.js} ad utilizzare tale porta.
	
	\newpage
	
	\section{Creazione nuova tipologia di domande}
	
	\subsection{Sintassi QML}
	Il QML (Quiz Markup Language) è un linguaggio con una sintassi definita da utilizzare per il salvataggio delle domande utilizzate nei quiz.
	Vengono qui illustrate le regole da adottare per utilizzare il QML per definire nuove tipologie di domande:
	\begin{itemize}
		\item deve iniziare e finire con i caratteri "|";
		\item la tipologia di domanda è una stringa lunga a piacere ed univoca, deve essere preceduta dalla sequenza di caratteri "q?";
		\item la parte relativa al testo della domanda è una stringa lunga a piacere, deve essere preceduta dalla sequenza di caratteri "\#t\#";
		\item la parte relativa a/alle risposta/e è una stringa lunga a piacere, deve essere preceduta dalla sequenza di caratteri "\#a\#";
		\item nel caso le due suddette stringhe siano multiple, è possibile utilizzare il carattere "§" per separare la varie parti;
		\item per aggiungere una validità o un id a una parte di testo si utilizzano i caratteri "[" e "]";
		\item eventuali allegati vanno racchiusi tra i caratteri "{" e "}", si veda la sottosezione seguente per i dettagli;
		\item alla fine delle stringhe relative al testo e alle risposte, deve essere utilizzata la sequenza di caratteri "\#££\#";
		\item è possibile inserire nuove parti aggiuntive nella stringa QML qualora vengano rispettate queste condizioni e il QMLAgent venga esteso in modo opportuno.
	\end{itemize}
	Segue un esempio di QML per i tipi di domande Vero/Falso e Risposta Multipla.
	\begin{verbatim}|q?trfs#t#Parigi è in Germania#a#false#££#|\end{verbatim}
	\begin{verbatim}|q?mult#t#Quali sono città europee?#a#Londra[true]§Mosca[false]#££#|\end{verbatim}
	
	\subsubsection{Allegati}
	Gli allegati possono essere aggiunti sia per il testo della domanda, che per ogni risposta.
	Si possono utilizzare immagini, video o audio, con un limite di ???.
	La sintassi specifica per allegare un file a una domanda o risposta è lo stesso ed è il seguente:
	\begin{itemize}
		\item deve essere preceduto dal tipo (img, vid, aud) e il carattere ":";
		\item deve contenere il path del file;
		\item opzionalmente può contenere le coordinate X e Y del media, preceduti dai caratteri ":x." e ":y.".
	\end{itemize}
	Segue un esempio di QML con allegato per la domanda.
	\begin{verbatim}|q?open#t#Chi è l'autore di quest'opera?{img:laPass.png:x.5:y.5}#a#Monet#££#|\end{verbatim}
	
	\subsection{QMLAgent}
	Il QMLAgent è il componente che si occupa di tradurre il QML in JSON, per poter poi creare effettivamente la domanda in modo visuale,
	e da JSON a QML per il salvataggio di una domanda nel database.
	Questo dopo aver rilevato il tipo di domanda chiama la funzione opportuna che genera il JSON o la stringa QML.
	Qualora si volesse implementare una nuova tipologia è quindi fondamentale scegliere una stringa univoca per identificare il tipo di domanda e inserire il caso che riconosca tale stringa.
	Successivamente è necessario creare una funzione di generazione e una di parsing per questo nuovo tipo, seguendo le regole definite in precedenza ed eventualmente creandone di nuove.
	
	\subsection{createQuestionBase}
	
	\subsection{??? }
	
	\newpage
	\appendix
	
	\section{Glossario}
	\label{gl} 
	
	\subsection{Angular.js}
	AngularJS (o semplicemente Angular o Angular.js) è un framework web open source principalmente sviluppato da Google e dalla comunità di sviluppatori individuali che ruotano intorno al framework nato per affrontare le molte difficoltà incontrate nello sviluppo di applicazioni singola pagina. Ha l'obiettivo di semplificare lo sviluppo e il test di questa tipologia di applicazioni fornendo un framework lato client con architettura MVC (Model View Controller) e Model–View–ViewModel (MVVM) insieme a componenti comunemente usate nelle applicazioni RIA.
	
	\subsection{Browser}
	Programma che consente di usufruire dei servizi di connettività in Internet, o di una rete di computer.
	
	\subsection{CSS3}
	Il CSS (Cascading Style Sheets) è un linguaggio usato per definire la formattazione di documenti HTML, XHTML e XML. CSS3 è l’ultimo standard approvato al momento.
	
	\subsection{Git}
	Sistema software di controllo di versione distribuito, utilizzabile da riga di comando.
	
	\subsection{HTML5}
	L’HTML5 è un linguaggio di markup per la strutturazione delle pagine web divenuto standard W3C nell’ottobre 2014.
	
	\subsection{JavaScript}
	Linguaggio di programmazione interpretato, generalmente utilizzato nella gestione degli eventi.
	Implementa un paradigma basato sia sugli oggetti che sulla programmazione funzionale.
	
	\subsection{MongoDB}
	MongoDB è un Data Base Management System non relazionale, orientato ai documenti. Classificato come un database di tipo NoSQL.
	
	\subsection{Node.js}
	Node.js è un framework event-driven per il motore JavaScript V8, su piattaforme UNIX like, si tratta cioè di un framework relativo all'utilizzo server-side di JavaScript.
	
	\subsection{Npm}
	Npm è il package manager di default per Node.js. Permette di installare moduli extra per Node.js da riga di comando.
	
	\subsection{Repository}
	È un ambiente di un sistema informativo, in cui vengono gestiti i dati, attraverso tabelle relazionali. In questo caso il sistema informativo è gestito con Git.
	
	\subsection{Server}
	Un server in informatica è un componente o sottosistema informatico di elaborazione e gestione del traffico di informazioni che fornisce, a livello logico e fisico, un qualunque tipo di servizio ad altre componenti (tipicamente chiamate clients, cioè clienti) che ne fanno richiesta attraverso una rete di computer, all'interno di un sistema informatico o anche direttamente in locale su un computer.
	
	\subsection{Software}
	Un software, in informatica, è l’informazione o le informazioni utilizzate da uno o più sistemi informatici
	e memorizzate su uno o più supporti informatici. Tali informazioni possono essere quindi rappresentate da uno o più programmi, oppure da uno o più dati, oppure da una combinazione
	delle due.
	
	\subsection{Web}
	Il Web, abbreviazione di World Wide Web, è uno dei principali servizi di Internet che permette di
	navigare e usufruire di un insieme vastissimo di contenuti (multimediali e non) collegati tra loro
	attraverso legami (link), e di ulteriori servizi accessibili a tutti o ad una parte selezionata degli
	utenti di Internet.
	
\end{document}
