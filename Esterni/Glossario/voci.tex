\newpage

\begin{center}
\Huge\section{\uppercase{A}}
\end{center}

\subsection{Akka}
Akka è un tool open source che semplifica la creazione di applicazioni concorrenti e distribuite sulla JVM.

\subsection{Amazon Web Services}
Amazon Web Services (AWS) è una collezione di servizi di cloud computing services, altresì chiamati servizi web.

\subsection{Android}
È un sistema operativo open source per dispositivi mobili sviluppato da Google.

\subsection{AngularJS}
AngularJS (o semplicemente Angular o Angular.js) è un framework web open source principalmente sviluppato da Google e dalla comunità di sviluppatori individuali che ruotano intorno al framework nato per affrontare le molte difficoltà incontrate nello sviluppo di applicazioni singola pagina. Ha l'obiettivo di semplificare lo sviluppo e il test di questa tipologia di applicazioni fornendo un framework lato client con architettura MVC (Model View Controller) e Model–view–viewmodel (MVVM) insieme a componenti comunemente usate nelle applicazioni RIA.

\newpage

\begin{center}
\Huge\section{\uppercase{B}}
\end{center}

\subsection{Back-end}
Nel campo della progettazione software il back end è la parte che elabora i dati generati dal front end.

\subsection{Baseline}
Una baseline è uno stato di avanzamento consolidato dal quale non si retrocede. 

\subsection{Bitbucket}
Bitbucket è un servizio di hosting web-based per i progetti.

\subsection{Bootstrap}
Bootstrap è una raccolta di strumenti liberi per la creazione di siti e applicazioni per il Web. Essa contiene modelli di progettazione basati su HTML e CSS, sia per la tipografia, che per le varie componenti dell'interfaccia, come moduli, bottoni e navigazione, e altri componenti dell'interfaccia, così come alcune estensioni opzionali di JavaScript.

\subsection{Browser}
Programma che consente di usufruire dei servizi di connettività in Internet, o di una rete di computer.

\subsection{Builder}
Nella programmazione ad oggetti, il Builder è uno dei pattern fondamentali, definiti originariamente dalla gang of four (banda dei quattro). Il design pattern Builder, nella programmazione ad oggetti, separa la costruzione di un oggetto complesso dalla sua rappresentazione cosicché il processo di costruzione stesso possa creare diverse rappresentazioni.

\newpage

\begin{center}
\Huge\section{\uppercase{C}}
\end{center}

\subsection{Canvas}
Canvas è una estensione dell'HTML standard che permette il rendering dinamico di immagini bitmap gestibili attraverso un linguaggio di scripting.

\subsection{Client}
Un client (in lingua italiana detto anche cliente), in informatica, indica una componente che accede ai servizi o alle risorse di un'altra componente detta server. In questo contesto si può quindi parlare di client riferendosi all'hardware oppure al software. Esso fa parte dunque dell'architettura logica di rete detta client-server.

\subsection{Cloud}
In informatica con il termine inglese cloud computing (in italiano nuvola informatica) si indica un paradigma di erogazione di risorse informatiche, come l'archiviazione, l'elaborazione o la trasmissione di dati, caratterizzato dalla disponibilità on demand attraverso Internet a partire da un insieme di risorse preesistenti e configurabili.

\subsection{CMM}
Il Capability Maturity Model (CMM) è un modello di sviluppo creato in seguito allo
studio dei dati collezionati dalle organizzazioni di ricerca sovvenzionate dal Dipartimento
della Difesa Statunitense. Il termine “capability” si riferisce alla misura dell’adeguatezza
del processo rispetto agli scopi per cui è stato definito. Il termine “maturity”, invece, è
legato al grado di formalità e ottimizzazione dei processi, proveniente da best practice, e
riguarda in particolare la definizione formale delle fasi, la gestione dei risultati delle misure
e l’ottimizzazione attiva dei processi.

\subsection{Cookies}
I cookie HTTP (più comunemente denominati cookie web, o per antonomasia cookie) sono un tipo particolare di magic cookie, una sorta di gettone identificativo, usato dai server web per poter riconoscere i browser durante comunicazioni con il protocollo HTTP usato per la navigazione web. Tale riconoscimento permette di realizzare meccanismi di autenticazione, conservazione di dati utili alla sessione di navigazione, di associare dati memorizzati dal server e di tracciare la navigazione dell'utente.

\subsection{Cross Site Scripting}
Il Cross Site Scripting (XSS) è una vulnerabilità che affligge siti web dinamici che impiegano un insufficiente controllo dell'input nei form. Un XSS permette ad un cracker di inserire o eseguire codice lato client al fine di attuare un insieme variegato di attacchi quali ad esempio: raccolta, manipolazione e reindirizzamento di informazioni riservate, visualizzazione e modifica di dati presenti sui server, alterazione del comportamento dinamico delle pagine web ecc. Nell'accezione odierna, la tecnica ricomprende l'utilizzo di qualsiasi linguaggio di scripting lato client tra i quali JavaScript, VBScript, Flash, HTML.

\subsection{CSS}
Il CSS (Cascading Style Sheets, in italiano fogli di stile a cascata), in informatica, è un linguaggio usato per definire la formattazione di documenti HTML, XHTML e XML ad esempio i siti web e relative pagine web. Le regole per comporre il CSS sono contenute in un insieme di direttive (Recommendations) emanate a partire dal 1996 dal W3C.

\subsection{CSS3}
Il CSS (Cascading Style Sheets) è un linguaggio usato per definire la formattazione di documenti HTML, XHTML e XML.
CSS3 è l'ultimo standard approvato al momento.

\newpage

\begin{center}
\Huge\section{\uppercase{D}}
\end{center}

\subsection{DAO}
In informatica, nell'ambito della programmazione Web, il DAO (Data Access Object) è un pattern architetturale per la gestione della persistenza: si tratta fondamentalmente di una classe con relativi metodi che rappresenta un'entità tabellare di un RDBMS, usata principalmente in applicazioni web sia di tipo Java EE sia di tipo EJB, per stratificare e isolare l'accesso ad una tabella tramite query (poste all'interno dei metodi della classe) ovvero al data layer da parte della business logic creando un maggiore livello di astrazione ed una più facile manutenibilità. I metodi del DAO con le rispettive query dentro verranno così richiamati dalle classi della business logic. Il vantaggio relativo all'uso del DAO è dunque il mantenimento di una rigida separazione tra le componenti di un'applicazione, le quali potrebbero essere il "Modello" e il "Controllo" in un'applicazione basata sul paradigma MVC.

\subsection{Dashboard}
È il termine inglese per "cruscotto", in ambito di sviluppo indica uno strumento
tramite il quale il Responsabile riesce a revisionare lo stato del progetto.

\subsection{Database}
È una base di dati, cioè un archivio o un insieme di archivi contenenti informazioni
strutturate e collegate tra loro secondo un particolare modello logico (relazionale, gerarchico,
a oggetti, etc.), offrendo una gestione efficiente dei dati che vi sono contenuti.

\subsection{Design pattern}
Soluzione progettuale generale per la risoluzione di un problema ricorrente.

\subsection{Diagramma di Gantt}
Il diagramma di Gantt è uno strumento di supporto alla gestione dei
progetti. Usato principalmente nelle attività di project management, è costruito partendo da
un asse orizzontale, usato per rappresentare l’arco temporale totale del progetto, suddiviso
in fasi incrementali, e da un asse verticale, usato per rappresentare delle mansioni o attività
che costituiscono il progetto. Un diagramma di Gantt permette dunque la rappresentazione
grafica di un calendario di attività, utile al fine di pianificare, coordinare e tracciare specifiche
attività in un progetto dando una chiara illustrazione dello stato d’avanzamento del progetto
rappresentato.

\subsection{DigitalOcean}
DigitalOcean è un ISP (internet service provider) americano con sede a New York City e possiede svariati data center in tutto il mondo. Il sistema operativo installabile è Linux mentre la memoria di massa è basata su SSD (solid-state drive).

\subsection{Domain Specific Language}
Il "Domain-Specific Language" o "Linguaggio specifico di Dominio" è un linguaggio
di programmazione o un linguaggio di specifica dedicato a particolari
problemi di un dominio nello sviluppo software, relativi a una particolare tecnica
di rappresentazione e/o a una particolare soluzione tecnica.

\subsection{DSL}
Si veda la voce Domain Specific Language.

\newpage

\begin{center}
\Huge\section{\uppercase{E}}
\end{center}

\subsection{Express.js}
Express è un framework dell'applicazione web Node.js flessibile e leggero che fornisce una serie di funzioni avanzate per le applicazioni web e per dispositivi mobili.

\newpage

\begin{center}
\Huge\section{\uppercase{F}}
\end{center}

\subsection{Fabric.js}
Fabric.js è una libreria che fornisce un modello a oggetti che manca al canvas, come anche un parser SVG, un livello di interattività, e una suite completa di altri strumenti indispensabili. È un progetto completamente open-source, sotto licenza MIT con numerosi contribuiti nel corso degli anni.

\subsection{Facade}
Il Facade è un design pattern strutturale che fornisce un' unica interfaccia semplice per un sottosistema complesso in modo tale che la complessità dell'intero sistema risulti minore e siano
semplificate le dipendenze tra le componenti che interagiscono.

\subsection{Framework}
Un framework, termine della lingua inglese che può essere tradotto come intelaiatura o struttura, in informatica e specificatamente nello sviluppo software, è un'architettura logica di supporto (spesso un'implementazione logica di un particolare design pattern) su cui un software può essere progettato e realizzato, spesso facilitandone lo sviluppo da parte del programmatore.

\subsection{Front-end}
Nel campo della progettazione software il front end è la parte di un sistema software che gestisce l'interazione con l'utente o con sistemi esterni che producono dati di ingresso (es. interfaccia utente con un form).

\newpage

\begin{center}
\Huge\section{\uppercase{G}}
\end{center}

\subsection{Git}
Sistema software di controllo di versione distribuito, utilizzabile da riga di comando.

\subsection{GitHub}
Servizio web di hosting per lo sviluppo di progetti software che usa il sistema di controllo
di versione Git . GitHub offre diversi piani per repository privati sia a pagamento, sia
gratuiti, molto utilizzati per lo sviluppo di progetti open source.

\subsection{Google Chrome}
Browser web sviluppato da Google.

\subsection{Google Drive}
Servizio di cloud storage multipiattaforma prodotto da Google, che offre un servizio di file hosting e
sincronizzazione automatica di file tramite web.

\newpage

\begin{center}
\Huge\section{\uppercase{H}}
\end{center}

\subsection{Hardware}
In ingegneria elettronica e informatica con il termine hardware si indica la parte fisica
di un computer, ovvero tutte quelle parti elettroniche, elettriche, meccaniche, magnetiche,
ottiche che ne consentono il funzionamento (dette anche strumentario). Più in generale il
termine si riferisce a qualsiasi componente fisico di una periferica o di una apparecchiatura
elettronica.

\subsection{Host}
Indica ogni terminale collegato, attraverso link di comunicazione, ad una rete informatica (es. Internet).

\subsection{HTML}
L'HyperText Markup Language (HTML) (traduzione letterale: linguaggio a marcatori per ipertesti), in informatica è il linguaggio di markup solitamente usato per la formattazione e impaginazione di documenti ipertestuali disponibili nel World Wide Web sotto forma di pagine web.

\subsection{HTML5}
L’HTML5 è un linguaggio di markup per la strutturazione delle pagine web divenuto
standard W3C nell’ottobre 2014.

\subsection{Hunspell}
Hunspell è un correttore ortografico utilizzato per eseguire le attività di verifica dei documenti.

\newpage

\begin{center}
\Huge\section{\uppercase{I}}
\end{center}

\subsection{IDE}
Acronimo di "Integrated Development Environment". Letteralmente "Ambiente
di Sviluppo Integrato". Esso fornisce un insieme di strumenti di ausilio
alla programmazione, solitamente comprendenti un editor, un compilatore e/o
interprete e un debugger.

\subsection{Indice Gulpease}
L'Indice Gulpease è un indice di leggibilità di un testo tarato sulla lingua
italiana. Rispetto ad altri ha il vantaggio di utilizzare la lunghezza delle parole in lettere
anziché in sillabe, semplificandone il calcolo automatico.

\subsection{Inspection }
Tecnica di analisi statica basata sulla lettura mirata dei documenti/codice in cerca di
errori specifici.

\subsection{Internet Explorer}
Browser web sviluppato da Microsoft Windows. La nuova versione del browser è la Edge, a differenza delle precedenti che portavano il nome di Internet Explorer.

\newpage

\begin{center}
\Huge\section{\uppercase{i}}
\end{center}

\subsection{iOS}
È un sistema operativo sviluppato da Apple.

\newpage

\begin{center}
\Huge\section{\uppercase{J}}
\end{center}

\subsection{Java}
Java è un linguaggio di programmazione che utilizza il paradigma ad oggetti simile al C++. È stato
progettato per rendere l'esecuzione il più possibile indipendente dalla piattaforma di
esecuzione.

\subsection{JavaScript}
Linguaggio di programmazione interpretato, generalmente utilizzato nella gestione
degli eventi. Implementa un paradigma basato sia sugli oggetti che sulla programmazione
funzionale.

\subsection{JetBrains}
JetBrains (ex IntelliJ) è un'azienda di sviluppo software ceca nata nel 2000 i cui strumenti hanno come mercato gli sviluppatori software e i project manager. Nel 2015 hanno oltre 500 dipendenti in 5 uffici tra Praga, San Pietroburgo, Mosca, Monaco di Baviera e Boston.

\subsection{JSON}
JSON, acronimo di JavaScript Object Notation, è un formato adatto all'interscambio di dati fra applicazioni client-server. È basato sul linguaggio JavaScript Standard ECMA-262 3ª edizione dicembre 1999, ma ne è indipendente. Viene usato in AJAX come alternativa a XML/XSLT.

\subsection{JVM}
La JVM, conosciuta come Java Virtual Machine, è il componente della piattaforma Java che esegue i programmi tradotti in bytecode dopo una prima compilazione.

\newpage

\begin{center}
\Huge\section{\uppercase{L}}
\end{center}

\subsection{LaTex}
Linguaggio di markup basato sul principio WYSIWYM (What You See Is What You
Mean), e utilizzato per la composizione di testi. Si presta particolarmente alla stesura di
documenti tecnici molto lunghi, in modo concorrente, da parte di persone diverse.

\subsection{Linguaggio di markup}
Insieme di regole che descrivono i meccanismi di rappresentazione (strutturali, semantici o presentazionali) di un testo.

\subsection{LoopBack}
LoopBack è un framework open source molto estensibile di Node.js. 

\newpage

\begin{center}
\Huge\section{\uppercase{M}}
\end{center}

\subsection{Middleware}
In informatica con middleware si intende un insieme di programmi informatici che fungono da intermediari tra diverse applicazioni e componenti software. Sono spesso utilizzati come supporto per sistemi distribuiti complessi.

\subsection{Milestone}
In italiano “Pietra miliare”. Questo termine indica importanti traguardi in termini di
tempo o di attività, che sanciscono una metrica di avanzamento del progetto.

\subsection{Modello incrementale}
Per modello incrementale o modello iterativo, nell'ambito dell'ingegneria
del software, si intende un modello di sviluppo di un progetto software basato sulla successione
dei seguenti passi principali: pianificazione, analisi dei requisiti, progetto, implementazione,
verifica, valutazione. Questo ciclo può essere ripetuto diverse volte fino a che la valutazione
del prodotto diviene soddisfacente rispetto ai requisiti richiesti.

\subsection{MongoDB}
MongoDB è un Data Base Management System non relazionale, orientato ai documenti. Classificato come un database di tipo NoSQL.

\subsection{Mozilla Firefox}
Browser web open source sviluppato da Mozilla Foundation. Risulta essere il terzo
browser più diffuso.

\subsection{MVC}
Il Model-View-Controller (MVC, talvolta tradotto in italiano Modello-Vista-Controllo), in informatica, è un pattern architetturale molto diffuso nello sviluppo di sistemi software, in particolare nell'ambito della programmazione orientata agli oggetti, in grado di separare la logica di presentazione dei dati dalla logica di business.

\subsection{MVVM}
Il Model–View–ViewModel (MVVM) è un pattern software architetturale o schema di progettazione software. É una variante del pattern "Presentation Model design" di Martin Fowler. Lo MVVM astrae lo stato di "view" (visualizzazione) e il comportamento. Sebbene, dove il modello di "presentazione" astrae una vista (crea un view model ) in una maniera che non dipende da una specifica piattaforma interfaccia utente.

\newpage

\begin{center}
\Huge\section{\uppercase{N}}
\end{center}

\subsection{Node.js}
Node.js è un framework event-driven per il motore JavaScript V8, su piattaforme UNIX
like, si tratta cioè di un framework relativo all'utilizzo server-side di JavaScript.

\newpage

\begin{center}
\Huge\section{\uppercase{O}}
\end{center}

\subsection{Open source}
Software di cui gli autori (più precisamente i detentori dei diritti) rendono pubblico
il codice sorgente, permettendo a programmatori indipendenti di apportarvi modifiche.
Questa possibilità è regolata tramite l'applicazione di apposite licenze d'uso.

\subsection{Opera}
Opera è un browser web freeware e multipiattaforma prodotto da Opera Software.

\subsection{OrientDB}
È una base di dati orientata al documento e le relazioni sono gestite come in un database a grafo con connessioni dirette tra i record.

\newpage

\begin{center}
\Huge\section{\uppercase{P}}
\end{center}

\subsection{Package}
In un sistema ad oggetti, un package è una collezione di classi.

\subsection{PDCA}
Il “Plan-Do-Check-Act”, detto anche Ciclo di Deming o Ciclo di Miglioramento Continuo,
è un metodo che permette di perseguire un continuo miglioramento della qualità nei processi.

\subsection{PDF}
Il Portable Document Format è un formato sviluppato da Adobe Systems per la rappresentazione di documenti in modo indipendente dall'hardware e dal software utilizzati per
visualizzarli o generarli.

\subsection{PhoneGap}
È un framework cross-platform mobile che consente di sviluppare delle applicazioni native attraverso l'utilizzo di tecnologie web quali HTML, CSS e JavaScript

\subsection{Play framework}
Play è un framework open source, scritto in Java e Scala, che implementa il pattern model-view-controller.

\subsection{PNG}
Il Portable Network Graphics è un formato per la memorizzazione di immagini.

\subsection{PostgreSQL}
PostgreSQL è un completo DBMS ad oggetti rilasciato con licenza libera (stile Licenza BSD). Spesso viene abbreviato come "Postgres", sebbene questo sia un nome vecchio dello stesso progetto. PostgreSQL è una reale alternativa sia rispetto ad altri prodotti liberi come MySQL, Firebird SQL e MaxDB che a quelli a codice chiuso come Oracle, IBM Informix o DB2 ed offre caratteristiche uniche nel suo genere che lo pongono per alcuni aspetti all'avanguardia nel settore dei database.

\newpage

\begin{center}
\Huge\section{\uppercase{Q}}
\end{center}

\subsection{QML}
Linguaggio di markup richiesto per la realizzazione del progetto.

\newpage

\begin{center}
\Huge\section{\uppercase{R}}
\end{center}

\subsection{Redmine}
Redmine è un software gestionale basato su web con funzioni di Project Management completo e flessibile con una struttura modulare e personalizzabile tramite campi definiti dall'utente, definizione workflow e molto altro.
Si tratta di un'applicazione gratuita e Open Source basata su framework Ruby on Rails, cross-platform e cross-database (MySQL, PostgreSQL or SQLite) accessibile via browser da qualsiasi computer connesso ad internet.

\subsection{Repository}
È un ambiente di un sistema informativo, in cui vengono gestiti i dati, attraverso
tabelle relazionali. In questo caso il sistema informativo è gestito con Git.

\subsection{Requisito desiderabile}
Requisiti non irrinunciabili ma con un valore aggiunto riconoscibile.

\subsection{Requisito obbligatorio}
Requisiti irrinunciabili per il corretto funzionamento del sistema.

\subsection{Requisito opzionale}
Requisiti di importanza relativa.

\newpage

\begin{center}
\Huge\section{\uppercase{S}}
\end{center}

\subsection{Safari}
Safari è un browser web sviluppato da Apple Inc. 

\subsection{Scala}
È un linguaggio di programmazione di tipo general-purpose multi-paradigma studiato per integrare le caratteristiche e funzionalità dei linguaggi orientati agli oggetti e dei linguaggi funzionali.

\subsection{Script}
Parola inglese che significa "copione". In informatica rappresenta un piccolo
programma, solitamente sequenziale e scritto in un linguaggio interpretato.
Spesso ha complessità bassa e realizza un singolo task.

\subsection{Server}
Un server in informatica è un componente o sottosistema informatico di elaborazione e gestione del traffico di informazioni che fornisce, a livello logico e fisico, un qualunque tipo di servizio ad altre componenti (tipicamente chiamate clients, cioè clienti) che ne fanno richiesta attraverso una rete di computer, all'interno di un sistema informatico o anche direttamente in locale su un computer.

\subsection{Slack}
Slack è un moderno strumento di messaggistica real-time che permette di concentrare tutte le comunicazioni del team in un unico luogo.
Consente di creare canali di comunicazione associati alle singole attività da svolgere permettendo di farne parte solo ai membri ai quali tali attività competono.

\subsection{Social}
Social media, in italiano media sociali , è un termine generico che indica tecnologie e pratiche online che le persone adottano per condividere contenuti testuali, immagini, video e audio.

\subsection{Socket.IO}
Socket.IO è una libreria JavaScript per le applicazioni web in real-time. Esso consente la comunicazione real-time bidirezionale tra client e server web. Ha due parti: un lato client che viene eseguito dal browser , e un lato server per Node.js.

\subsection{Software}
Un software, in informatica, è l'informazione o le informazioni utilizzate da uno o più
sistemi informatici e memorizzate su uno o più supporti informatici. Tali informazioni possono
essere quindi rappresentate da uno o più programmi, oppure da uno o più dati, oppure
da una combinazione delle due.

\subsection{StarUML}
StarUML è un'applicazione completa e Open Source per la realizzazione di modelli e progetti di sviluppo software. Permette in pratica di creare diagrammi sintetici e analitici per descrivere i blocchi di codice necessari allo sviluppo di un programma. Come dice il nome stesso, StarUML è pensato soprattutto per la realizzazione di grafici con le specifiche UML, le più utilizzate in questo campo.

\newpage

\begin{center}
\Huge\section{\uppercase{T}}
\end{center}

\subsection{Team}
Col termine Team, o con la sua traduzione italiana (Gruppo), nei documenti del progetto
Quizzipedia si intendono le persone che fanno parte del Vault-Tech.


\subsection{Telegram}
Telegram è un servizio di messaggistica istantanea erogato senza fini di lucro dalla società Telegram LLC. Viene utilizzata dal gruppo per le comunicazioni interne.

\subsection{Template}
Il termine inglese template (letteralmente "sagoma" o "calco") in informatica
indica un documento o programma nel quale, come in un foglio semicompilato
cartaceo, su una struttura generica o standard esistono spazi da riempire
successivamente.

\subsection{TeXstudio}
TeXstudio è un ottimo editor multipiattaforma per LaTeX.

\subsection{Ticket}
Associazione tra una persona e un obbligo o attività.

\subsection{Tomcat}
Apache Tomcat (o semplicemente Tomcat) è un application server nella forma di contenitore servlet open source sviluppato dalla Apache Software Foundation. Implementa le specifiche JavaServer Pages (JSP) e Servlet, fornendo quindi una piattaforma software per l'esecuzione di applicazioni Web sviluppate in linguaggio Java. 

\subsection{Tracker}
Tracker è lo strumento web creato dal team Vault-Tech. Il suo scopo è quello di automatizzare la generazione della documentazione relativa al progetto da realizzare e di tracciare tutti gli elementi e le attività ad esso correlate.

\newpage

\begin{center}
\Huge\section{\uppercase{U}}
\end{center}

\subsection{Ubiika}
È una piattaforma di prossimità che gestisce contenuti contestualizzati nelle
aree coperte da tecnologia BLE (beacon)​

\subsection{UML}
Acronimo per "Unified Modeling Language". È un linguaggio di modellazione e
specifica basato sul paradigma object-oriented. UML 2.0 riorganizza molti degli
elementi della versione precedente (1.5) in un quadro di riferimento ampliato e
introduce molti nuovi strumenti, inclusi alcuni nuovi tipi di diagrammi.

\subsection{Unity}
Framework di sviluppo multipiattaforma.

\subsection{URL}
La locuzione Uniform Resource Locator (in acronimo URL), nella terminologia delle telecomunicazioni e dell'informatica è una sequenza di caratteri che identifica univocamente l'indirizzo di una risorsa in Internet, tipicamente presente su un host server, come ad esempio un documento, un'immagine, un video, rendendola accessibile ad un client che ne faccia richiesta attraverso l'utilizzo di un web browser.

\subsection{UTF-8}
UTF-8 (Unicode Transformation Format, 8 bit) è una codifica dei caratteri Unicode in sequenze di lunghezza variabile di byte.

\newpage

\begin{center}
\Huge\section{\uppercase{V}}
\end{center}

\subsection{Versionamento}
È la gestione di versioni multiple di un insieme di informazioni.

\newpage

\begin{center}
\Huge\section{\uppercase{W}}
\end{center}

\subsection{W3C}
Il W3C, noto come World Wide Web Consortium, è un'organizzazione non governativa internazionale che ha come scopo quello di sviluppare tutte le potenzialità del World Wide Web.

\subsection{Walkthrough }
Walkthrough è una forma di analisi statica di un prodotto non compilabile, o del
quale non si vuole o non si può fare un'analisi dinamica, che prevede un esame accurato
dell'oggetto dell'analisi alla ricerca di ogni possibile errore. Questa tecnica viene applicata
quando non si ha un'idea precisa della tipologia di errori che possono verificarsi.

\subsection{Web}
Il Web, abbreviazione di World Wide Web, è uno dei principali servizi di Internet che permette di navigare e usufruire di un insieme vastissimo di contenuti (multimediali e non) collegati tra loro attraverso legami (link), e di ulteriori servizi accessibili a tutti o ad una parte selezionata degli utenti di Internet.

\subsection{WebStorm}
WebStorm è un ambiente di sviluppo integrato (IDE), il quale fornisce un ampio set di strumenti rivolti alla programmazione ed al supporto di diverse tecnologie web. È utilizzabile in ambiti di sviluppo di applicazioni basate su linguaggi quali XML, CSS, HMTL e JavaScript, rivelando una buona versatilità d'uso. Il tool è disponibile per i sistemi operativi Mac e Windows e prevede le opzioni tipicamente richieste nella stesura di codice. Il correttore ortografico integrato si adatta al contesto ed in modo automatico controlla commenti, stringhe e tags, al fine di ridurre possibili errori.

\subsection{Wikipedia}
Wikipedia è un'enciclopedia online a contenuto aperto, collaborativa, multilingue e gratuita, nata nel 2001, sostenuta e ospitata dalla Wikimedia Foundation, un'organizzazione non a scopo di lucro statunitense.

\subsection{Windows Phone}
Windows Phone è una famiglia di sistemi operativi per smartphone di Microsoft.

