% Nome del file: SpecificaTecnica.tex
% Percorso: \gl{template}
% Autore: Vault-Tech
% Data creazione: 17.03.2016
% E-mail: vaulttech.swe@gmail.comcom
%
% Diario delle modifiche: interno al file.

\documentclass[a4paper, titlepage]{article}


\usepackage[margin=3cm]{geometry}
\usepackage{float}
\usepackage{grffile}
\usepackage{../../Stile}
\usepackage{../../Comandi}

\setcounter{secnumdepth}{5}
\setcounter{tocdepth}{5}

\def\NOME{Specifica Tecnica}
\def\VERSIONE{0.1}
\def\DATA{05.03.2016}
\def\REDATTORE{Giacomo Beltrame \\ & Simone Boccato \\ & Michela De Bortoli \\ & Vassilikì Menarin \\ & Miki Violetto}
\def\VERIFICATORE{Rudy Berton \\ & Filippo Tesser}
\def\RESPONSABILE{Michela De Bortoli \\ & Filippo Tesser}
\def\USO{Esterno}
\def\DISTRIBUZIONE{\AUTORE}

\begin{document}
\pagestyle{fancy}	
\pagenumbering{Roman}
\rfoot{Pagina \thepage{} di \pageref{lastromanpage}}

\maketitle

\begin{diario}
	\recap{Stesura della struttura del documento}{Giacomo Beltrame}{Progettista}{05.03.2016}{0.1}
\end{diario}

\newpage
\tableofcontents

\newpage
\listoffigures \label{lastromanpage}

\newpage
\clearpage	
\pagenumbering{arabic}
\rfoot{Pagina \thepage{} di \pageref*{LastPage}}
\hypersetup{linkcolor=blue}

\section{Introduzione}
\subsection{Scopo del documento}
In tale documento verrà definita la progettazione ad alto livello del prodotto Quizzipedia.
Per tale scopo vengono descritte le componenti, le classi e i design pattern utilizzati per la realizzazione del prodotto. Inoltre viene presentato il tracciamento tra le componenti e i requisiti individuati.

\subsection{Scopo del prodotto}
\SCOPO

\subsection{Glossario}
\GLOSSARIO

\subsection{Riferimenti}
\subsubsection{Riferimenti normativi}
\begin{itemize}
\item \bold{Norme di Progetto:} \NdPdoc.
\end{itemize}

\subsubsection{Riferimenti informativi}
\begin{itemize}
\item \bold{Capitolato d'appalto C5:} \italics{Quizzipedia: \gl{software} per la gestione di questionari} \url{http://www.math.unipd.it/~tullio/IS-1/2015/Progetto/C5.pdf};

\item \bold{Glossario:} \Gldoc;

\item \bold{Analisi dei requisiti: } \ARdoc;

\item \bold{HTML5: } \url{https://www.w3.org/TR/html5/};

\item \bold{CSS3: } \url{https://www.w3.org/TR/CSS/};

\item \bold{Javascript: } \url{http://www.ecma-international.org/publications/files/ECMA-ST/Ecma-262.pdf};

\item \bold{PostgreSQL: } \url{http://www.postgresql.org/docs/};

\item \bold{AngularJS: } \url{https://angular.io/docs/ts/latest/};

\item \bold{Node.js: } \url{https://nodejs.org/api/};

\item \bold{Express.js: } \url{http://expressjs.com/en/api.html};

\item \bold{Bootstrap: } \url{http://getbootstrap.com/css/};

\item \bold{JSON: } \url{http://www.ecma-international.org/publications/files/ECMA-ST/ECMA-404.pdf};

\item \bold{Fabric.js: } \url{http://fabricjs.com/docs/};

\item \bold{Socket.IO: } \url{http://socket.io/docs/}.

\end{itemize}

\newpage

\section{Tecnologie utilizzate}
In questa sezione vengono presentate le tecnologie scelte per sviluppare il progetto. Per ognuna di esse verrà fornita una breve descrizione, i principali punti a favore e sfavore e, ove necessario, la versione utilizzata.

\subsection{AngularJS 2.0}

\subsection{Node.js}

\subsection{Express.js}

\subsection{PostgreSQL}

\subsection{HTML5}

\subsection{CSS3}

\subsection{Bootstrap}

\subsection{Javascript}

\subsection{JSON}

\subsection{Fabric.js}

\subsection{Socket.IO}

\end{document}