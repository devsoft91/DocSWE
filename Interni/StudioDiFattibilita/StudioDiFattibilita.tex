% Nome del file: Studio di fattibilità.tex
% Percorso: \gl{template}
% Autore: Vault-Tech
% Data creazione: 19.12.2015
% E-mail: vaulttech.swe@gmail.comcom
%
% Diario delle modifiche: interno al file.

\documentclass[a4paper, titlepage]{article}

\usepackage[margin=3cm]{geometry}
\usepackage{Stile}
\usepackage{Comandi}

\setcounter{secnumdepth}{5}
\setcounter{tocdepth}{5}

\def\NOME{Studio di fattibilità}
\def\VERSIONE{1.0}
\def\DATA{24.12.2015}
\def\REDATTORE{Simone Boccato}
\def\VERIFICATORE{Filippo Tesser}
\def\RESPONSABILE{Vassilikì Menarin}
\def\USO{Interno}
\def\DISTRIBUZIONE{\AUTORE}

\begin{document}
\pagestyle{fancy}	
\pagenumbering{Roman}
\rfoot{Pagina \thepage{} di \pageref{lastromanpage}}

\maketitle

\begin{diario}
	\recap{Approvazione del documento}{Vassilikì Menarin}{Responsabile}{24.12.2015}{1.0}
	\recap{Correzione errori}{Simone Boccato}{Analista}{23.12.2015}{0.3}
	\recap{Verifica del documento}{Filippo Tesser}{Verificatore}{22.12.2015}{0.2}
	\recap{Stesura del documento}{Simone Boccato}{Analista}{20.12.2015}{0.1}
\end{diario}

\newpage

\tableofcontents\label{lastromanpage}

\newpage
\clearpage
\pagenumbering{arabic}
\rfoot{Pagina \thepage{} di \pageref*{LastPage}}
\hypersetup{linkcolor=blue}

\section{Introduzione}
\subsection{Scopo del Documento}
Lo scopo di questo documento è descrivere le motivazioni che hanno portato
alla scelta del capitolato C5. Vengono inoltre riportate le considerazioni
e le motivazioni per cui non sono stati scelti gli altri capitolati d'appalto.

\subsection{Scopo del prodotto}
\SCOPO

\subsection{Glossario}
\GLOSSARIO

\subsection{Riferimenti}

\subsubsection{Normativi}
\begin{itemize}
	\item \textbf{Norme di Progetto: }\textit{"Norme di Progetto v1.0"}.
\end{itemize}

\subsubsection{Informativi}
\begin{itemize}
	\item \textbf{Capitolato d'appalto C1:} \textit{Actorbase: a NoSQL DB based on the Actor model} (\href{http://www.math.unipd.it/~tullio/IS-1/2015/Progetto/C1.pdf}{http://www.math.unipd.it/~tullio/IS-1/2015/Progetto/C1.pdf})
	
	\item \textbf{Capitolato d'appalto C2:} \textit{CLIPS: Communication \& Localisation with Indoor Positioning System} (\href{http://www.math.unipd.it/~tullio/IS-1/2015/Progetto/C2.pdf}{http://www.math.unipd.it/~tullio/IS-1/2015/Progetto/C2.pdf})
	
	\item \textbf{Capitolato d'appalto C3:} \textit{UMAP: un motore per l'analisi predittiva in ambiente Internet of Things}
	(\href{http://www.math.unipd.it/~tullio/IS-1/2015/Progetto/C3.pdf}{http://www.math.unipd.it/~tullio/IS-1/2015/Progetto/C3.pdf})
	
	\item \textbf{Capitolato d'appalto C4:} \textit{MaaS: MondoDB as an admin Service} (\href{http://www.math.unipd.it/~tullio/IS-1/2015/Progetto/C4.pdf}{http://www.math.unipd.it/~tullio/IS-1/2015/Progetto/C4.pdf})
	
	\item \textbf{Capitolato d'appalto C5:} \textit{Quizzipedia: \gl{software} per la gestione di questionari} (\href{http://www.math.unipd.it/~tullio/IS-1/2015/Progetto/C5.pdf}{http://www.math.unipd.it/~tullio/IS-1/2015/Progetto/C5.pdf})
	
	\item \textbf{Capitolato d'appalto C6:} \textit{SiVoDiM: Sintesi Vocale per Dispositivi Mobili} (\href{http://www.math.unipd.it/~tullio/IS-1/2015/Progetto/C6.pdf}{http://www.math.unipd.it/~tullio/IS-1/2015/Progetto/C6.pdf})
\end{itemize}

\newpage

\section{Capitolato C5 - Quizzipedia}
\subsection{Descrizione del capitolato}
Si è scelto il capitolato proposto dall'azienda Zucchetti S.p.a, chiamato Quizzipedia, riguardante la creazione di un \gl{software} per la produzione di questionari specifici per diversi argomenti. Viene richiesto che tale \gl{software} sia sviluppato in tecnologia \gl{HTML5} e che funzioni su ogni dispositivo desktop, mobile e tablet.
Le funzionalità minime che il \gl{software} deve soddisfare sono:
\begin{itemize}
	\item archiviazione di quiz in un server, suddivisi per argomento;
	\item creazione di un linguaggio \gl{QML} (Quiz Markup Language) per la realizzazione delle domande;
	\item traduzione delle domande archiviate da \gl{QML} a HTML;
	\item gestione, da parte del linguaggio \gl{QML}, delle risposte vero/falso, risposte a scelta multipla, testi e immagini;
	\item archiviazione di questionari contenenti domande dal primo punto;
	\item proposta di questionari preconfezionati;
	\item valutazione delle risposte fornite dall'utente.
\end{itemize}

\subsection{Studio del dominio}
Per lo sviluppo del capitolato scelto sono necessarie alcune competenze tecnologiche e la conoscenza del contesto nel quale si inserisce l’applicazione. Di seguito vengono illustrati i domini tecnologici e applicativi ai quali si riferisce il capitolato.
\subsubsection{Dominio tecnologico}
Al gruppo viene richiesta la conoscenza nei seguenti campi:
\begin{itemize}
	\item \textit{\gl{HTML5}:} il gruppo dispone di una conoscenza basilare di questa tecnologia;
	\item \textit{\gl{CSS3}:} il gruppo dispone di una conoscenza basilare di questa tecnologia;
	\item \textit{\gl{JavaScript}:} il gruppo dispone di una conoscenza basilare di questa tecnologia;
	\item \textit{\gl{Tomcat}:} il gruppo non ha conoscenze riguardanti questa tecnologia;
	\item \textit{\gl{Node.js}:} il gruppo non ha conoscenze riguardanti questa tecnologia.
\end{itemize}

Sono state inoltre valutate e considerate le seguenti librerie:
\begin{itemize}
	\item \textit{bootstrap:} \href{http://getbootstrap.com/}{http://getbootstrap.com/}.
\end{itemize}

\subsubsection{Dominio applicativo}
Per la comprensione del dominio applicativo il gruppo si è basato su alcuni \gl{software} \gl{open source} suggeriti dal Proponente:

\begin{itemize}
	\item \textit{Moodle:} \href{https://hotpot.uvic.ca/}{https://hotpot.uvic.ca/}
	\item \textit{Hot Potatoes:} \href{https://hotpot.uvic.ca}{https://hotpot.uvic.ca}
\end{itemize}

\subsection{Potenziali criticità}
Le principali criticità provengono dalla mancata esperienza di creazione di un \gl{linguaggio di markup} e dalla scarsa conoscenza di alcune tecnologie di cui si necessita l'utilizzo.
\subsection{Analisi di mercato}
La scarsa presenza sul mercato di questi prodotti ha incentivato la partecipazione del gruppo a questo progetto poiché oggigiorno il \gl{web} è molto utilizzato e ci sono pochi \gl{software} di questa categoria, anche se usati ampiamente.
\subsection{Fattibilità del progetto}
I principali punti che hanno portato il gruppo a scegliere questo capitolato sono:
\begin{itemize}
	\item Requisiti obbligatori e opzionali espressi molto chiaramente;
	\item Basi di \textit{\gl{HTML5}}, \textit{\gl{JavaScript}} e \textit{\gl{CSS3}} provenienti dagli studi universitari;
	\item Interesse nel realizzare un'applicazione che funzioni su tutti i dispositivi;
	\item Interesse nell'apprendere nuove tecnologie ritenute utili per il mondo del lavoro.
	
\end{itemize}


\newpage
\section{Altri Capitolati}

\subsection{Capitolato C1 - Actorbase: a NoSQL DB based on the Actor model}

\subsubsection{Descrizione del capitolato}
Il capitolato proposto dal professor Riccardo Cardin chiede la realizzazione di un \gl{Database} NoSQL Key-value  (p.es. \gl{MongoDB}) utilizzando il modello ad attori.
Al gruppo viene richiesto lo sviluppo di un insieme di attori:
\begin{itemize}
	\item STOREKEEPER
	\item STOREFINDER
	\item WAREHOUSEMEN
\end{itemize}
e l'implementazione di alcune operazioni sul \gl{database}:
\begin{itemize}
	\item Inserimento;
	\item Cancellazione;
	\item Aggiornamento (caso particolare di un inserimento con chiave già presente).
\end{itemize}

Si richiede inoltre la definizione di un \gl{domain specific language} (\gl{DSL}) da utilizzare da riga di comando per poter interagire con il \gl{database}.
Il progetto dovrà essere sviluppato utilizzando la libreria \gl{Akka} (\href{http://akka.io}{http://akka.io}) per l'implementazione del modello ad attori su \gl{JVM} e il linguaggio da utilizzare può essere scelto tra \gl{Scala} (\href{http://www.scala-lang.org/}{http://www.scala-lang.org/}) o \gl{Java}.

Il progetto dev'essere pubblicato su \gl{GitHub} e far uso delle issue per la segnalazione di eventuali bug.

\subsubsection{Valutazione generale}
I requisiti del progetto sono stati esposti in modo sufficientemente chiaro. Il gruppo valuta positivamente questo progetto in quanto i \gl{database} NoSQL sono diffusi e sarebbe utile averne un approccio diretto. Inoltre si avrebbe la possibilità di imparare anche un nuovo linguaggio \gl{Scala} e l'utilizzo della libreria \gl{Akka}.

\subsubsection{Potenziali criticità}
Le principali criticità vengono dalla mancanza di conoscenze riguardati i \gl{database} NoSQL e il modello ad attori. Inoltre nessun elemento del \gl{team} ha esperienza con il linguaggio \gl{Scala} e la libreria \gl{Akka}.

\newpage

\subsection{Capitolato C2 - CLIPS: Communication  Localisation with Indoor Positioning Systems}

\subsubsection{Descrizione del capitolato}

Il capitolato proposto da Miriade richiede la ricerca e la sperimentazione di nuovi scenari per l'implementazione della navigazione indoor applicata a più ambiti.
L'obiettivo è quello di riuscire ad allestire uno scenario funzionale (interazione, navigazione, broadcasting) e definire l'area indoor che verrà coperta (un'aula, un percorso specifico, etc.). Inoltre sarà necessario scegliere i componenti HW(beacon) e SW (SDK, framework, componenti \gl{open source}, etc.) da utilizzare ed effettuare alcuni test di fattibilità prima di procedere allo sviluppo vero e proprio del prototipo.

Quando sarà sviluppato il \gl{software} il gruppo deve individuare le problematiche lato \gl{hardware} (problemi lato mobile e beacon) e anche \gl{software} (navigazione e comunicazione). Infine è richiesta la presentazione all'utente in modo immediato ed intuitivo dello spazio coperto dalla funzionalità IPS.

Il proponente mette a disposizione diversi strumenti:
\begin{itemize}
	\item \gl{Repository} per \gl{versionamento} di sorgenti;
	\item \gl{Redmine} per il tracciamento dei requisiti;
	\item Accesso al \gl{database} \gl{Ubiika};
	\item Un numero di beacons (massimo 50).
	
\end{itemize}

Infine il proponente intende che gli sviluppatori del prodotto di questo capitolato ne ritengano il copyright intellettuale e materiale.

\subsubsection{Valutazione generale}

Interessante dominio applicativo, è emerso un interesse sia per l'ambiente di sviluppo \gl{Android}/\gl{iOS}, sia per l'utilizzo dei beacon.

\subsubsection{Potenziali criticità}

Le principali criticità provengono dalle scarse conoscenze del dominio applicativo e tecnologico.
La partecipazione di altri due gruppi alla gara d'appalto ha scoraggiato la scelta di questo capitolato.

\newpage

\subsection{Capitolato C3 - UMAP: un motore per l'analisi predittiva in ambiente Internet of Things}

\subsubsection{Descrizione del capitolato}
Il capitolato proposto da Zero12 Innovation Company richiede di creare un algoritmo predittivo in grado di analizzare i dati provenienti da “oggetti”, inseriti in diversi contesti, e fornire delle previsioni su possibili guasti, interazioni con nuovi utenti ed identificare dei pattern di comportamento degli utenti per prevedere le azioni degli stessi su altri oggetti o altri contesti.

L'applicazione \gl{software} sarà composta da:
\begin{itemize}
	\item Console \gl{Web} amministrativa per definire regole di apprendimento a seconda del contesto e del tipo di dati;
	\item Console \gl{Web} amministrativa per le singole aziende;
	\item Servizi \gl{Web} Restful JSON interrogabili.
\end{itemize}

Tecnologie da utilizzare:
\begin{itemize}
	\item \gl{Amazon \gl{Web} Services};
	\item \gl{MongoDB} e/o \gl{OrientDB};
	\item \gl{Java} o \gl{Scala};
	\item \gl{HTML5}, \gl{CSS3} e \gl{JavaScript} per lo sviluppo dell'interfaccia \gl{Web}.
\end{itemize}

\subsubsection{Valutazione generale}
I requisti del progetto non sono stati esposti in modo sufficientemente chiaro. Il gruppo ha familiarità con alcune tecnologie che si dovranno utilizzare.

\subsubsection{Potenziali criticità}
Le principali criticità provengono dalla scarsa conoscenza del linguaggio \gl{Scala} e dall'assenza di conoscenze sull'utilizzo di \gl{Amazon \gl{Web} Services}, \gl{MongoDB} e \gl{Play framework}.
Il gruppo ha inoltre ritenuto che lo sviluppo di algoritmi predittivi sia molto oneroso (deve essere efficiente data la grande mole di dati da analizzare) e complesso.

\newpage

\subsection{Capitolato C4 - MaaS: MongoDB as an admin Service}

\subsubsection{Descrizione del capitolato}
Il capitolato proposto da RedBabel \href{http://redbabel.com/}{(http://redbabel.com/)} richiede di sviluppare un \gl{web} service \textit{Maas (\gl{MongoDB} as an admin Service)} che estenda MaaP in due punti significativi: \textit{SaaS (\gl{Software} as a Service)} e \textit{\gl{DSL}}.
Inoltre Maas deve essere reso disponibile direttamente via \gl{web} rendendolo usufruibile da molte aziende.

Per lo sviluppo del \gl{software} il proponente richiede:
\begin{itemize}
	\item Realizzazione di MaaS;
	
	\item Lo sviluppo del sistema utilizzando \gl{Node.js}, però si consiglia fortemente IBM Loopback (\href{http://loopback.io}{http://loopback.io});
	
	\item Il sistema deve essere sviluppato per Heroku (\href{http://www.heroku.com}{http://www.heroku.com}), una piattaforma \gl{cloud} che supporta diversi linguaggi di programmazione;
	
	\item Il codice sorgente deve essere pubblicato su \gl{GitHub}(\href{www.github.com}{www.github.com}) oppure su \gl{Bitbucket} (\href{www.bitbucket.com}{www.bitbucket.com}).
	
\end{itemize}


\subsubsection{Valutazione generale}
I requisti del capitolato non sono stati esposti in modo chiaro; forse risulta il capitolato meno comprensibile e maggiormente difficile rispetto agli altri.

\subsubsection{Potenziali criticità}
Le principali criticità vengono dalla mancanza di familiarità riguardante i \gl{database} NoSQL, e da alcune tecnologie che il proponente richiede. Inoltre il gruppo prova uno scarso interesse per il dominio applicativo.

\newpage

\subsection{Capitolato C6 - SiVoDiM: Sintesi Vocale per Dispositivi Mobili}

\subsubsection{Descrizione del capitolato}

Il capitolato proposto da MIVOQ S.R.L. richiede la realizzazione di un'applicazione per dispositivi mobili (smartphone e tablet), che sfrutti appieno le potenzialità offerte dal motore di sintesi vocale \gl{Open source} “Flexible and Adaptive Text To Speech” (FA-TTS), in particolare la possibilità di creare la propria voce sintetica e di applicare degli effetti a voci tradizionali.

Per quanto riguarda le tecnologie scelte, si richiede:
\begin{itemize}
	\item L'utilizzo del motore di sintesi FA-TTS;
	\item L'applicativo deve essere realizzato su una o più piattaforme mobili: \gl{Android}, \gl{iOS} o \gl{Windows Phone};
	\item Se lo si ritiene opportuno, un framework di sviluppo multipiattaforma: \gl{Unity}, \gl{PhoneGap}.
\end{itemize}

\subsubsection{Valutazione generale}

I requisti del capitolato sono stati esposti in modo chiaro con una moltitudine di esempi.
Il gruppo ha provato molto interesse per la sintetizzazione vocale, però il capitolato è stato reputato difficile in quanto c'è scarsa familiarità con la sintetizzazione vocale e nessuna esperienza con FA-TTS.

\subsubsection{Potenziali criticità}
Le principali criticità vengono dalla mancanza di conoscenze riguardanti FA-TTS e da possibili caratteristiche/tecnologie ulteriori che potrebbero rendersi necessarie. Inoltre, la partecipazione di altri due gruppi alla gara d'appalto ha scoraggiato la scelta di questo capitolato.

\end{document}
